\subsubsection{最適化アルゴリズム}

本システムでは、未利用資源の収集運搬経路を最適化するために、車両経路問題(Vehicle Routing Problem, VRP)の解法を実装している。VRPは、複数の地点を訪問する最適な経路を決定する組合せ最適化問題であり、物流分野において広く研究されている。

\paragraph{問題の定式化}

本システムが扱う問題は、以下のように定式化される:

\begin{itemize}
    \item \textbf{入力}:
    \begin{itemize}
        \item デポ(車両の出発地点・帰着地点)
        \item 複数の回収地点(未利用資源の発生場所)
        \item 集積場所(資源の最終目的地)
        \item 道路ネットワーク(ノードとエッジのグラフ構造)
        \item 車両の容量制約
        \item 各回収地点での資源量
    \end{itemize}
    \item \textbf{目的}:
    \begin{itemize}
        \item 総走行距離の最小化
        \item 運搬費用の最小化
        \item 車両台数の最小化
    \end{itemize}
    \item \textbf{制約}:
    \begin{itemize}
        \item すべての回収地点を訪問すること
        \item 車両の積載容量を超えないこと
        \item すべての車両がデポから出発し、デポに帰着すること
        \item 資源は最終的に集積場所に運搬されること
    \end{itemize}
\end{itemize}

\paragraph{最短経路探索アルゴリズム}

本システムでは、グラフ理論に基づく最短経路探索アルゴリズムを採用している。

\subparagraph{Dijkstra法}
Dijkstra法は、単一始点からすべてのノードへの最短経路を求める古典的なアルゴリズムである。計算量はO((V + E) log V)(Vはノード数、Eはエッジ数)であり、負の重みがない場合に正確な解を得られる。

本システムでは、NetworkXライブラリの\texttt{dijkstra\_path}関数を使用して実装している。デポから各回収地点、回収地点間、回収地点から集積場所への最短経路を事前計算し、これを基に全体の経路を構築する。

\subparagraph{A*アルゴリズム}
A*アルゴリズムは、ヒューリスティック関数を用いてDijkstra法を拡張したアルゴリズムである。目標ノードまでの推定距離(ヒューリスティック値)を利用することで、探索の効率を向上させる。

本システムでは、ユークリッド距離をヒューリスティック関数として使用し、NetworkXの\texttt{astar\_path}関数により実装している。特に目標地点が明確な場合、Dijkstra法よりも高速に解を得られる。

\paragraph{車両経路問題の解法}

最短経路探索を基礎として、以下の手法により車両経路問題を解決している。

\subparagraph{最近傍法(Nearest Neighbor Heuristic)}
初期解の構築に使用される貪欲法である:
\begin{enumerate}
    \item デポから最も近い未訪問の回収地点を訪問
    \item 現在地から最も近い未訪問の回収地点を訪問
    \item 積載容量に達したら集積場所へ運搬
    \item すべての回収地点を訪問するまで繰り返す
\end{enumerate}

計算量はO(n²)(nは回収地点数)であり、高速に解を得られるが、最適解を保証しない。

\subparagraph{2-opt法による改善}
最近傍法で得られた初期解を改善するための局所探索法である:
\begin{enumerate}
    \item 経路中の2つのエッジを選択
    \item エッジを入れ替えた場合の経路長を計算
    \item 改善があれば入れ替えを採用
    \item 改善がなくなるまで繰り返す
\end{enumerate}

この手法により、交差する経路を解消し、経路長を短縮できる。

\subparagraph{容量制約の処理}
各車両には積載容量の制約があり、これを考慮した経路構築が必要である。本システムでは、以下の方法で容量制約を処理している:

\begin{itemize}
    \item 回収地点を訪問する際、現在の積載量に資源量を加算
    \item 積載量が車両容量を超える場合、その回収地点は訪問せず、次の候補を選択
    \item 積載量が容量に達したら、集積場所へ運搬し積載量をリセット
    \item すべての回収地点を訪問するまでこのプロセスを繰り返す
\end{itemize}

\paragraph{計算の効率化}

大規模な問題に対しても現実的な時間で解を得るため、以下の効率化を実施している:

\begin{itemize}
    \item \textbf{距離行列の事前計算}:すべてのノード間の最短距離を事前に計算し、キャッシュする
    \item \textbf{候補地点の絞り込み}:ヒューリスティックにより、明らかに非効率な経路を探索対象から除外
    \item \textbf{並列計算}:複数の車両の経路を独立に計算できる場合、並列処理を適用
    \item \textbf{早期終了条件}:一定時間経過後、または十分に良い解が得られた時点で探索を終了
\end{itemize}

これらのアルゴリズムにより、数十から数百の回収地点を持つ実用的な問題を、数秒から数分で解決できる。
