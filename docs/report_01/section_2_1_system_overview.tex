\subsection{システムの全体構成}

本システムは、未利用資源の収集運搬計画を最適化するためのWebアプリケーションである。Streamlitフレームワークを基盤とし、以下の主要コンポーネントから構成される(図\ref{fig:system_architecture})。

\begin{figure}[H]
\centering
\fbox{
\begin{minipage}{0.9\textwidth}
\centering
\vspace{0.5cm}
\textbf{【入力層】} \\
道路ネットワークデータ、マスタデータ、ユーザー選択 \\
\vspace{0.3cm}
$\downarrow$ \\
\vspace{0.3cm}
\textbf{【データ格納層】} \\
NetworkXグラフ、地点レジストリ、車種カタログ、\\
距離行列、空間インデックス \\
\vspace{0.3cm}
$\downarrow$ \\
\vspace{0.3cm}
\textbf{【処理層】} \\
データ読込 → 地点選択 → 車種割当 → 距離計算 →\\
ルート最適化 → 結果生成 \\
\vspace{0.3cm}
$\downarrow$ \\
\vspace{0.3cm}
\textbf{【出力層】} \\
最適ルート情報、コスト詳細、エネルギー消費量、地図表示 \\
\vspace{0.5cm}
\end{minipage}
}
\caption{システムの全体構成}
\label{fig:system_architecture}
\end{figure}

\subsubsection{システム構成要素}

システムは4層構造で構成されており、図\ref{fig:system_architecture}に示す各層の要素を以下に説明する。

\paragraph{入力層}
入力層では、システムが最適化計算を実行するために必要な3種類のデータを受け取る。

\subparagraph{道路ネットワークデータ}
OpenStreetMap(OSM)またはカスタムJSON形式で提供される道路ネットワーク情報である。ノード(交差点・地点)とエッジ(道路区間)の接続関係、各道路区間の距離情報を含む。

\subparagraph{マスタデータ}
システムの基礎となる静的データであり、以下の3つのファイルから構成される:
\begin{itemize}
    \item \texttt{vehicles.json}:14種類の運搬車両の諸元(積載容積、燃費、購入費用、固定費、変動費)
    \item \texttt{resources.json}:11種類の未利用資源の特性(比重、取扱い注意事項)
    \item \texttt{compatibility.json}:車両と資源の適合性マトリックス(154組み合わせ)
\end{itemize}

\subparagraph{ユーザー選択}
Streamlitの対話的インターフェースを通じてユーザーが入力する情報である:
\begin{itemize}
    \item デポ(車庫)の位置:車両の出発地点・帰着地点の座標
    \item 回収地点:資源を回収する地点の座標、資源種別、回収量
    \item 集積場所:資源の最終目的地の座標
    \item 使用車両の選択:14種類から選択
\end{itemize}

\paragraph{データ格納層}
データ格納層では、入力層から受け取ったデータを最適化処理に適した形式に変換・保持する。

\subparagraph{NetworkXグラフ}
道路ネットワークをグラフ理論のデータ構造に変換したものである。ノード(頂点)とエッジ(辺)で構成され、最短経路探索アルゴリズムの基盤となる。

\subparagraph{地点レジストリ}
デポ、回収地点、集積場所を統合的に管理するデータ構造である。各地点のID、座標、タイプ(デポ/回収地点/集積場所)、資源情報を保持する。

\subparagraph{車種カタログ}
選択された車両の詳細情報を格納する。積載容積、燃費、コスト情報、資源適合性などを含む。

\subparagraph{距離行列}
すべての地点間の最短距離をN×N行列として事前計算・キャッシュしたものである。最適化計算の高速化に寄与する。

\subparagraph{空間インデックス}
地図上のクリック位置から最寄りの道路ネットワークノードを高速に検索するためのデータ構造である。KD-treeなどの空間分割アルゴリズムを使用する。

\paragraph{処理層}
処理層では、データ格納層のデータを用いて最適化計算を実行する。図\ref{fig:system_architecture}に示す6つのステップを順次処理する。

\subparagraph{データ読込}
入力層からのデータをパースし、データ格納層の各データ構造を初期化する。

\subparagraph{地点選択}
ユーザーの地図クリック操作を受け取り、空間インデックスを用いて最寄りの道路ネットワークノードを特定する。選択された地点を地点レジストリに登録する。

\subparagraph{車種割当}
資源種別に基づいて、車種カタログから適合する車両を抽出する。複数の車両が適合する場合、コスト評価関数により最適な車両を選択する。

\subparagraph{距離計算}
NetworkXグラフを用いて、選択された全地点間の最短経路をDijkstra法またはA*アルゴリズムで計算する。計算結果を距離行列に格納する。

\subparagraph{ルート最適化}
車両経路問題(VRP)を解決し、最適な訪問順序を決定する。容量制約と資源適合性を考慮したヒューリスティックアルゴリズム(最近傍法、2-opt法)を使用する。

\subparagraph{結果生成}
最適化されたルートに基づき、総走行距離、運搬費用、エネルギー消費量を計算する。地図表示用のデータ形式に変換する。

\paragraph{出力層}
出力層では、最適化結果をユーザーに分かりやすい形式で提供する。

\subparagraph{最適ルート情報}
車両ごとの訪問順序、各区間の距離、総走行距離、所要時間を表示する。

\subparagraph{コスト詳細}
変動費12項目と固定費13項目の内訳、項目別金額、総コストを詳細に表示する。

\subparagraph{エネルギー消費量}
燃料消費量、CO$_2$排出量を車両の燃費データから算出し表示する。

\subparagraph{地図表示}
Foliumライブラリを用いて、最適化されたルートをインタラクティブな地図上に可視化する。経路は色分けされ、クリックすると詳細情報がポップアップ表示される。

\subsubsection{データフロー}

システムのデータフローは以下の通りである:

\begin{enumerate}
    \item ユーザーが道路ネットワークを選択し、システムに読み込む
    \item 地図上で拠点、回収地点、集積場所を設定する
    \item 使用する車両タイプと運搬する資源タイプを選択する
    \item システムが車両・資源適合性を自動的に検証する
    \item 最適化エンジンが最短経路と最小コストのルートを計算する
    \item 結果が地図上に可視化され、詳細レポートが生成される
    \item ユーザーは結果をCSV、JSON、HTML形式でエクスポート可能
\end{enumerate}

\subsubsection{技術スタック}

本システムは以下の技術により構築されている:

\begin{itemize}
    \item \textbf{フロントエンド}:Streamlit(Python Webフレームワーク)
    \item \textbf{グラフ処理}:NetworkX(道路ネットワークのグラフ表現と経路探索)
    \item \textbf{地図可視化}:Folium(インタラクティブマップ生成)
    \item \textbf{データ処理}:Pandas、NumPy(データ管理と数値計算)
    \item \textbf{最適化}:OR-Tools(制約付き最適化問題の解決)
    \item \textbf{データ形式}:JSON、CSV(データの入出力)
\end{itemize}

\subsubsection{システムの特徴}

\begin{itemize}
    \item \textbf{使いやすさ}:専門知識不要の直感的なインターフェース
    \item \textbf{柔軟性}:様々な資源・車両の組み合わせに対応
    \item \textbf{拡張性}:新しい資源や車両の追加が容易
    \item \textbf{透明性}:計算過程とコスト内訳を詳細に表示
    \item \textbf{実用性}:実際のコストデータに基づく現実的な推計
\end{itemize}
