\subsubsection{運搬車両}

本システムでは、未利用資源の収集運搬に使用される14種類の車両を対象としている。各車両の特徴と主な用途を以下に分類して示す。

\paragraph{小型車両(2-3tクラス)}

\subparagraph{軽トラック}
最大積載量350kg程度、車両総重量2t未満の小型貨物車である。狭い道路での機動性に優れ、普通免許で運転可能である。少量の資源回収や農村部での巡回収集に適している。

\subparagraph{2tトラック(平ボディ)}
最大積載量2t、荷台がフラットな汎用トラックである。荷台寸法は約3.1m×1.6m×0.4m(長さ×幅×高さ)であり、様々な形状の資源を積載可能で、側面からの積み下ろしが容易である。小規模事業所からの資源回収や市街地での収集業務に使用される。

\subparagraph{2tダンプ}
油圧装置により荷台を傾斜させて荷下ろしする車両である。土砂、廃棄物など流動性のある資源に適しており、建設廃材、剪定枝、家畜糞尿などの運搬に使用される。

\paragraph{中型車両(4tクラス)}

\subparagraph{4t平ボディ}
最大積載量4t程度、車両総重量8t未満の中型トラックである。荷台寸法は約5.0-6.2m×2.1-2.2m×0.6mであり、運搬効率と機動性のバランスが良く、最も汎用性が高い。中規模事業所からの資源回収や中距離輸送に適している。

\subparagraph{4tダンプ}
2tダンプよりも大容量で、建設現場などでの大量運搬に適する。建設廃材、解体材、土砂類の大量運搬に使用される。

\subparagraph{4tユニック車(クレーン付きトラック)}
小型クレーンを搭載し、重量物の積み下ろしが可能な車両である。クレーン搭載により荷台容積は減少するが、重量物の単独作業が可能である。金属スクラップや建設廃材など重量物の運搬に使用される。

\subparagraph{4tウイング車}
側面が翼のように開閉する箱型トラックである。側面全開により積み下ろし作業性が極めて良好で、雨風から荷物を保護できる。古紙、廃プラスチック、食品廃棄物など多様な資源の運搬に使用される。

\subparagraph{4tパッカー車(塵芥車)}
圧縮機構を備えた廃棄物収集専用車両である。回転板やプレスにより廃棄物を圧縮し、積載効率が高い。食品廃棄物、廃プラスチック、古紙、剪定枝などの収集に使用される。

\subparagraph{4tアームロール車}
コンテナを脱着できる装置を備えた車両である。複数のコンテナを準備することで車両の稼働率を向上でき、様々な資源に対応可能である。建設廃材、産業廃棄物、多種類の資源の効率的収集に使用される。

\paragraph{大型車両(10tクラス)}

\subparagraph{10t平ボディ}
最大積載量10t程度、車両総重量25t未満の大型トラックである。荷台寸法は約9m×2.4m×0.5mであり、大量輸送に適し、長距離輸送でも効率的である。大規模施設からの資源回収、広域収集、長距離輸送に使用される。

\subparagraph{10tダンプ}
建設廃材や土砂の大量輸送に最適である。解体現場や大規模工事現場からの廃材運搬に使用される。

\subparagraph{10tウイング車}
大容量と作業性を両立し、最も効率的な大型運搬車両である。古紙、廃プラスチック、パレット化された資源の大量輸送に使用される。

\subparagraph{大型パッカー車}
圧縮機構を備えた大型廃棄物収集車である。大量の廃棄物を効率的に圧縮収集でき、広域での一般廃棄物収集や大規模イベントでの廃棄物回収に使用される。

\paragraph{特殊用途車両}

\subparagraph{バキューム車(4tクラス)}
真空ポンプにより液体・泥状物を吸引収集する専用車両である。液状・泥状の資源専用で、密閉タンク構造を持つ。家畜糞尿、下水汚泥、食品廃液の収集運搬に使用される。

\vspace{1em}

これらの車両の諸元(積載容積、燃費、購入費用など)は、実際のメーカーカタログおよび業界データに基づいて詳細に設定されている(付録A参照)。
