\subsubsection{資源・運搬車両適合性表}

車両と資源の適合性は、物理的適合性、性状適合性、作業効率、法令遵守、安全性の観点から総合的に判定されている。適合性の判定基準は以下の通りである。

\paragraph{適合(○)の判定基準}
\begin{itemize}
    \item \textbf{物理的適合性}:資源の形状・サイズが荷台に収まる
    \item \textbf{性状適合性}:資源の性状(固体/液体/粉体)が車両構造に適している
    \item \textbf{作業効率}:積み下ろし作業が現実的な時間・労力で実施可能
    \item \textbf{法令遵守}:廃棄物処理法等の関連法規に抵触しない
    \item \textbf{安全性}:運搬中の落下・飛散・漏洩リスクが許容範囲内
\end{itemize}

\paragraph{条件付き適合($\triangle$)の判定基準}
追加の設備または対策を講じることにより運搬が可能となる組み合わせである。例えば、密閉容器の使用、防風シートの設置、防臭対策の実施などが条件として設定される。

\paragraph{不適合(×)の判定基準}
\begin{itemize}
    \item \textbf{構造的不適合}:車両の構造上、当該資源の運搬が不可能
    \item \textbf{性状不適合}:資源の性状が車両に適さない(例:液体をダンプで運搬)
    \item \textbf{衛生上の問題}:食品廃棄物を開放型荷台で運搬など
    \item \textbf{経済性の欠如}:著しく非効率(例:農業残渣をユニック車で運搬)
    \item \textbf{環境汚染リスク}:運搬中の漏洩により環境汚染の恐れ
\end{itemize}

\paragraph{主要な判定例}
\begin{itemize}
    \item \textbf{ダンプ×食品廃棄物}:水分が多く汁漏れのリスク、開放型荷台では衛生上不適切
    \item \textbf{パッカー車×金属スクラップ}:圧縮機構が金属により破損する恐れ、重量物には不向き
    \item \textbf{ユニック車×農業残渣}:軽量でかさばる資源にクレーンは不要、経済性に欠ける
    \item \textbf{バキューム車○下水汚泥}:液状・泥状物専用車両として最適、密閉構造で臭気対策も万全
    \item \textbf{アームロール$\triangle$家畜糞尿}:密閉コンテナ使用により臭気を抑制、コンテナ交換で効率的
\end{itemize}

\vspace{1em}

表\ref{tab:compatibility}に、14種類の車両と11種類の資源の適合性マトリックスを示す。

\begin{sidewaystable}[p]
\centering
\caption{�ԗ��E�����K�����}�g���b�N�X}
\label{tab:compatibility}
\begin{tabular}{|l|c|c|c|c|c|c|c|c|c|c|c|}
\hline
\textbf{�ԗ�} & \rotatebox{90}{\textbf{���ݔp��}} & \rotatebox{90}{\textbf{�_�Ǝc��}} & \rotatebox{90}{\textbf{�ыƎc��}} & \rotatebox{90}{\textbf{�H�i�p����}} & \rotatebox{90}{\textbf{�p�v���X�`�b�N}} & \rotatebox{90}{\textbf{�����X�N���b�v}} & \rotatebox{90}{\textbf{�Î��E�i�{�[��}} & \rotatebox{90}{\textbf{����}�E��}} & \rotatebox{90}{\textbf{�ƒ{���A}} & \rotatebox{90}{\textbf{�������D}} & \rotatebox{90}{\textbf{�p�H�p��}} \\ \hline
�y�g���b�N & $\circ$ & $\circ$ & $\circ$ & $\times$ & $\triangle$ & $\circ$ & $\triangle$ & $\circ$ & $\times$ & $\times$ & $\triangle$ \\ \hline
2t���{�f�B & $\circ$ & $\circ$ & $\circ$ & $\times$ & $\triangle$ & $\circ$ & $\triangle$ & $\circ$ & $\times$ & $\times$ & $\triangle$ \\ \hline
2t�_���v & $\circ$ & $\circ$ & $\circ$ & $\times$ & $\times$ & $\circ$ & $\times$ & $\circ$ & $\triangle$ & $\times$ & $\times$ \\ \hline
4t���{�f�B & $\circ$ & $\circ$ & $\circ$ & $\times$ & $\triangle$ & $\circ$ & $\triangle$ & $\circ$ & $\times$ & $\times$ & $\triangle$ \\ \hline
4t�_���v & $\circ$ & $\circ$ & $\circ$ & $\times$ & $\times$ & $\circ$ & $\times$ & $\circ$ & $\triangle$ & $\times$ & $\times$ \\ \hline
4t���j�b�N & $\circ$ & $\times$ & $\circ$ & $\times$ & $\times$ & $\circ$ & $\times$ & $\times$ & $\times$ & $\times$ & $\times$ \\ \hline
4t�E�C���O & $\circ$ & $\circ$ & $\circ$ & $\circ$ & $\circ$ & $\times$ & $\circ$ & $\circ$ & $\circ$ & $\times$ & $\circ$ \\ \hline
4t�p�b�J�[ & $\times$ & $\circ$ & $\circ$ & $\circ$ & $\circ$ & $\times$ & $\circ$ & $\circ$ & $\times$ & $\times$ & $\times$ \\ \hline
4t�A�[�����[�� & $\circ$ & $\circ$ & $\circ$ & $\circ$ & $\circ$ & $\circ$ & $\circ$ & $\circ$ & $\circ$ & $\times$ & $\circ$ \\ \hline
10t���{�f�B & $\circ$ & $\circ$ & $\circ$ & $\times$ & $\triangle$ & $\circ$ & $\triangle$ & $\circ$ & $\times$ & $\times$ & $\triangle$ \\ \hline
10t�_���v & $\circ$ & $\circ$ & $\circ$ & $\times$ & $\times$ & $\circ$ & $\times$ & $\circ$ & $\triangle$ & $\times$ & $\times$ \\ \hline
10t�E�C���O & $\circ$ & $\circ$ & $\circ$ & $\circ$ & $\circ$ & $\times$ & $\circ$ & $\circ$ & $\circ$ & $\times$ & $\circ$ \\ \hline
��^�p�b�J�[ & $\times$ & $\circ$ & $\circ$ & $\circ$ & $\circ$ & $\times$ & $\circ$ & $\circ$ & $\times$ & $\times$ & $\times$ \\ \hline
�o�L���[����(4t) & $\times$ & $\times$ & $\times$ & $\times$ & $\times$ & $\times$ & $\times$ & $\times$ & $\circ$ & $\circ$ & $\times$ \\ \hline
�K��������: & $\times$ & $\times$ & $\times$ & $\times$ & $\times$ & $\times$ & $\times$ & $\times$ & $\times$ & $\times$ & $\times$ \\ \hline
1 = �K��: �ԗ��\���������ɓK���A�@�ߏ���E���S���E�o�ϐ��̊ϓ_������Ȃ� & $\times$ & $\times$ & $\times$ & $\times$ & $\times$ & $\times$ & $\times$ & $\times$ & $\times$ & $\times$ & $\times$ \\ \hline
0 = �s�K��: �ԗ��\����̐���A�q����̖��A�o�ϐ��̌��@�Ȃǂɂ��s�K�� & $\times$ & $\times$ & $\times$ & $\times$ & $\times$ & $\times$ & $\times$ & $\times$ & $\times$ & $\times$ & $\times$ \\ \hline
�����t���K��: & $\times$ & $\times$ & $\times$ & $\times$ & $\times$ & $\times$ & $\times$ & $\times$ & $\times$ & $\times$ & $\times$ \\ \hline
- ���—e��: ���—e��i�t���R���A�h�����ʓ��j�g�p�ɂ��K�����i�lj��R�X�g5-15�~/km�j & $\times$ & $\times$ & $\times$ & $\times$ & $\times$ & $\times$ & $\times$ & $\times$ & $\times$ & $\times$ & $\times$ \\ \hline
- �h���V�[�g: ��U�h�~�V�[�g�g�p�ɂ��K�����i�lj��R�X�g2-5�~/km�j & $\times$ & $\times$ & $\times$ & $\times$ & $\times$ & $\times$ & $\times$ & $\times$ & $\times$ & $\times$ & $\times$ \\ \hline
- �V�[�g: �J���h��V�[�g�g�p�ɂ��K�����i�lj��R�X�g2-5�~/km�j & $\times$ & $\times$ & $\times$ & $\times$ & $\times$ & $\times$ & $\times$ & $\times$ & $\times$ & $\times$ & $\times$ \\ \hline
- �h�L�΍�: ���L�܁E���ɂ��K�����i�lj��R�X�g3-8�~/km�j & $\times$ & $\times$ & $\times$ & $\times$ & $\times$ & $\times$ & $\times$ & $\times$ & $\times$ & $\times$ & $\times$ \\ \hline
- �o�ϐ���: �Z�p�I�ɂ͉”\�����o�ϓI�ɔ���� & $\times$ & $\times$ & $\times$ & $\times$ & $\times$ & $\times$ & $\times$ & $\times$ & $\times$ & $\times$ & $\times$ \\ \hline
��v�ȕs�K�����R: & $\times$ & $\times$ & $\times$ & $\times$ & $\times$ & $\times$ & $\times$ & $\times$ & $\times$ & $\times$ & $\times$ \\ \hline
- �_���v�~�H�i�p����: �`�R�ꃊ�X�N�A�q����s�K�� & $\times$ & $\times$ & $\times$ & $\times$ & $\times$ & $\times$ & $\times$ & $\times$ & $\times$ & $\times$ & $\times$ \\ \hline
- �p�b�J�[�~�����X�N���b�v: ���k�@�\�j�����X�N & $\times$ & $\times$ & $\times$ & $\times$ & $\times$ & $\times$ & $\times$ & $\times$ & $\times$ & $\times$ & $\times$ \\ \hline
- ���j�b�N�~�_�Ǝc��: �y�ʎ����ɃN���[���s�v�A�o�ϐ����@ & $\times$ & $\times$ & $\times$ & $\times$ & $\times$ & $\times$ & $\times$ & $\times$ & $\times$ & $\times$ & $\times$ \\ \hline
- ���{�f�B�~�ƒ{���A: �L�C�E�R�k���X�N�i���—e��g�p�ʼnj & $\times$ & $\times$ & $\times$ & $\times$ & $\times$ & $\times$ & $\times$ & $\times$ & $\times$ & $\times$ & $\times$ \\ \hline
- �o�L���[���ԁ~�ő̎���: �t��E�D���p�ԗ� & $\times$ & $\times$ & $\times$ & $\times$ & $\times$ & $\times$ & $\times$ & $\times$ & $\times$ & $\times$ & $\times$ \\ \hline
���L: & $\times$ & $\times$ & $\times$ & $\times$ & $\times$ & $\times$ & $\times$ & $\times$ & $\times$ & $\times$ & $\times$ \\ \hline
1. �����t���K���ł́A�΍���{�ɂ��s�K��(0)���K��(1)�ɕύX�”\ & $\times$ & $\times$ & $\times$ & $\times$ & $\times$ & $\times$ & $\times$ & $\times$ & $\times$ & $\times$ & $\times$ \\ \hline
2. �lj��R�X�g�͑΍�ɕK�v�ȗe��E���ށE��򓙂̔�p & $\times$ & $\times$ & $\times$ & $\times$ & $\times$ & $\times$ & $\times$ & $\times$ & $\times$ & $\times$ & $\times$ \\ \hline
3. �@�ߏ���i�p���������@���j�̊m�F���K�v & $\times$ & $\times$ & $\times$ & $\times$ & $\times$ & $\times$ & $\times$ & $\times$ & $\times$ & $\times$ & $\times$ \\ \hline
4. ���^�p�ł͒n��̋K���E�ڋq�v���ɂ�蔻�肪�ς��ꍇ���� & $\times$ & $\times$ & $\times$ & $\times$ & $\times$ & $\times$ & $\times$ & $\times$ & $\times$ & $\times$ & $\times$ \\ \hline
\end{tabular}
\end{sidewaystable}

\vspace{0.5em}
\noindent\textbf{�}��}�F
$\circ$: �K���A
$\triangle$: �����t���K���A
$\times$: �s�K��

\vspace{1em}
\noindent\textbf{�����t���K���̏ڍ�}�F

\begin{table}[H]
\centering
\caption{�����t���K���̗v��}
\begin{tabular}{|l|l|l|}
\hline
\textbf{�ԗ�} & \textbf{����} & \textbf{�K�v����} \\ \hline
�y�g���b�N & �p�v���X�`�b�N & �h���V�[�g \\ \hline
�y�g���b�N & ���E�i�{�[�� & �V�[�g \\ \hline
�y�g���b�N & �p�H�p�� & ���—e�� \\ \hline
2t���{�f�B & �p�v���X�`�b�N & �h���V�[�g \\ \hline
2t���{�f�B & ���E�i�{�[�� & �V�[�g \\ \hline
2t���{�f�B & �p�H�p�� & ���—e�� \\ \hline
2t�_���v & �ƒ{���A & �h�L�΍� \\ \hline
4t���{�f�B & �p�v���X�`�b�N & �h���V�[�g \\ \hline
4t���{�f�B & ���E�i�{�[�� & �V�[�g \\ \hline
4t���{�f�B & �p�H�p�� & ���—e�� \\ \hline
4t�_���v & �ƒ{���A & �h�L�΍� \\ \hline
10t���{�f�B & �p�v���X�`�b�N & �h���V�[�g \\ \hline
10t���{�f�B & ���E�i�{�[�� & �V�[�g \\ \hline
10t���{�f�B & �p�H�p�� & ���—e�� \\ \hline
10t�_���v & �ƒ{���A & �h�L�΍� \\ \hline
\end{tabular}
\end{table}


\vspace{1em}

この適合性マトリックスは、システム内で自動的に参照され、ユーザーが選択した車両と資源の組み合わせが適切かどうかを判定する。不適合な組み合わせの場合は警告が表示され、条件付き適合の場合は必要な条件が提示される。これにより、実現不可能な運搬計画の立案を未然に防ぐことができる。
