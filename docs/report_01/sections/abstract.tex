% 要旨
\section*{要旨}
\addcontentsline{toc}{section}{要旨}

本報告書は、交通安全環境研究所で開発中の「地域交通計画立案ツール」を改良し、群馬県内における未利用資源の収集運搬最適化システムを開発した研究成果を述べたものである。

本研究は、未利用資源の効果的な収集運搬が地域の循環型社会形成において重要な課題であることを踏まえ、運搬最適化を目的として実施された。システムは、Streamlitフレームワークを基盤としたWebアプリケーションとして実装され、以下の主要機能を提供する。

主な特徴として、以下の点が挙げられる:

\begin{itemize}
    \item 地図上での直感的な地点選択インターフェース
    \item 14種類の運搬車両と11種類の未利用資源に対応
    \item 車両・資源の適合性マトリックス(154組み合わせ)による自動検証
    \item 変動費12項目・固定費13項目の詳細なコスト内訳表示
    \item 最短経路探索アルゴリズム(Dijkstra法・A*アルゴリズム)の実装
    \item インタラクティブな地図可視化(Foliumライブラリ)
\end{itemize}

システムの核心技術として、容量制約付き車両経路問題(CVRP)を解くための貪欲法ベースのヒューリスティックアルゴリズムを採用した。これにより、実用的な時間内で最適に近い解を得ることが可能となった。

本システムは、委託仕様書に定められたすべての要件を達成した:
\begin{itemize}
    \item 地域交通計画立案ツールの改良による未利用資源運搬対応
    \item 11種類の未利用資源のシステム変数化
    \item 14種類の運搬車両(軽トラから10tトラックまで)への対応
    \item 単一収集場所から単一集積場所への最短経路と運搬費用の推計
\end{itemize}

本研究の成果は、未利用資源収集運搬業務の効率化に寄与するとともに、今後の人貨混載システムへの拡張基盤を提供するものである。

\vspace{1cm}

\noindent
\textbf{キーワード}:未利用資源、ルート最適化、VRP、車両経路問題、コスト推計、地域交通計画、Streamlit
