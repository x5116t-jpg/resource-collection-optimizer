% 第4章 使い方
\section{使い方}

本システムは、Webブラウザを通じて利用可能な対話的インターフェースを提供している。以下に、システムの基本的な使用方法を5つのステップで説明する。

\subsection{ステップ1: システムの起動}

\subsubsection{Windows環境での起動}
プロジェクトのルートディレクトリに配置された\texttt{run\_app.bat}をダブルクリックすることで、システムが起動する。コマンドプロンプトウィンドウが開き、数秒後にWebブラウザが自動的に起動する。

\subsubsection{その他の環境での起動}
Pythonがインストールされている環境では、ターミナルから以下のコマンドを実行する:

\begin{verbatim}
streamlit run app.py
\end{verbatim}

システムが正常に起動すると、ブラウザに「未利用資源収集運搬最適化システム」のタイトルが表示される。

\subsection{ステップ2: 道路ネットワークの選択}

システムの最初のステップとして、道路ネットワークを選択する。

\subsubsection{選択肢}
\begin{enumerate}
    \item \textbf{デフォルトネットワーク}:群馬県大泉町を中心とした事前準備済みの道路ネットワーク
    \item \textbf{カスタムネットワーク}:独自の道路ネットワークデータ(JSON形式)をアップロード
\end{enumerate}

\subsubsection{デフォルトネットワークの使用(推奨)}
初めて使用する場合は、デフォルトネットワークの使用を推奨する。「デフォルトネットワークを使用」ボタンをクリックすると、地図上に道路ネットワークが表示される。

\subsubsection{カスタムネットワークのアップロード}
独自の道路ネットワークを使用する場合、JSON形式のファイルをアップロードする。ファイル形式の詳細は付録を参照されたい。

\subsection{ステップ3: 拠点・地点の設定}

地図上で以下の3種類の地点を設定する。

\subsubsection{デポ(車両の出発地点・帰着地点)}
\begin{enumerate}
    \item 「デポを追加」ボタンをクリック
    \item 地図上の任意の場所をクリックしてデポを配置
    \item デポ名を入力(例:「車両基地」)
\end{enumerate}

デポは通常1箇所設定するが、複数のデポを設定することも可能である(マルチデポVRP)。

\subsubsection{回収地点(資源の発生場所)}
\begin{enumerate}
    \item 「回収地点を追加」ボタンをクリック
    \item 地図上の資源発生場所をクリック
    \item 回収地点名と資源量(kg)を入力
\end{enumerate}

回収地点は複数設定可能である。各地点での資源量を正確に入力することで、より精度の高い最適化が可能となる。

\subsubsection{集積場所(資源の最終目的地)}
\begin{enumerate}
    \item 「集積場所を追加」ボタンをクリック
    \item 地図上の集積場所をクリック
    \item 集積場所名を入力(例:「バイオマス発電所」)
\end{enumerate}

集積場所は、資源を最終的に運搬する先である。リサイクル施設、バイオマス発電所、処理施設などが該当する。

\subsection{ステップ4: 車両と資源の選択・最適化の実行}

\subsubsection{車両タイプの選択}
ドロップダウンメニューから使用する車両タイプを選択する(14種類から選択可能)。例:
\begin{itemize}
    \item 少量運搬・狭い道路:軽トラック、2t平ボディ
    \item 中量運搬・汎用性:4t平ボディ、4tウイング車
    \item 大量運搬・長距離:10t平ボディ、10tウイング車
    \item 液状資源:バキューム車
    \item 圧縮収集:4tパッカー車、大型パッカー車
\end{itemize}

\subsubsection{資源タイプの選択}
運搬する資源タイプを選択する(11種類から選択可能)。例:
\begin{itemize}
    \item 建設廃材、林業残材
    \item 農業残渣、剪定枝・草
    \item 食品廃棄物、廃食用油
    \item 廃プラスチック、古紙・段ボール
    \item 金属スクラップ
    \item 家畜糞尿、下水汚泥
\end{itemize}

\subsubsection{適合性の自動検証}
車両と資源を選択すると、システムが自動的に適合性を検証する:
\begin{itemize}
    \item \textbf{適合}:問題なく運搬可能
    \item \textbf{条件付き適合}:追加対策(密閉容器、シートなど)が必要
    \item \textbf{不適合}:この組み合わせでは運搬不可
\end{itemize}

不適合の場合、警告メッセージが表示され、別の車両または資源の選択を促される。

\subsubsection{最適化の実行}
すべての設定が完了したら、「最適化を実行」ボタンをクリックする。システムが最適な収集運搬経路を計算し、数秒から数十秒で結果を表示する。

\subsection{ステップ5: 結果の確認とエクスポート}

\subsubsection{地図上での結果表示}
最適化された経路が地図上に色分けされた線で表示される。各経路をクリックすると、詳細情報(距離、コストなど)がポップアップ表示される。

\subsubsection{詳細レポートの確認}
画面下部に表示される詳細レポートには、以下の情報が含まれる:
\begin{itemize}
    \item 総走行距離
    \item 総運搬費用(変動費・固定費の内訳)
    \item 各経路の詳細(訪問順序、距離、所要時間)
    \item 車両別の稼働状況
    \item 資源回収量の集計
\end{itemize}

\subsubsection{結果のエクスポート}
結果は以下の形式でエクスポート可能である:
\begin{itemize}
    \item \textbf{CSV形式}:表計算ソフトで分析可能
    \item \textbf{JSON形式}:プログラムで処理可能
    \item \textbf{HTML形式}:レポートとして保存・共有可能
\end{itemize}

「結果をエクスポート」ボタンをクリックし、希望する形式を選択してダウンロードする。

\subsection{応用的な使い方}

\subsubsection{複数シナリオの比較}
異なる車両や経路設定で複数回最適化を実行し、結果を比較することで、最も経済的な運搬計画を選択できる。

\subsubsection{季節変動への対応}
資源量が季節により変動する場合、各季節のデータで最適化を実行し、年間を通じた運搬計画を立案できる。

\subsubsection{新規拠点の評価}
新しい回収地点や集積場所を追加した場合の影響を事前にシミュレーションし、投資判断の材料とすることができる。
