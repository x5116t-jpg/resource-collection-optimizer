% 付録B 技術的補足
\section{技術的補足}

本付録では、システムの実装に関する技術的補足を記載する。

\subsection{最短経路アルゴリズムの詳細}

\subsubsection{Dijkstra法の実装}

Dijkstra法は、優先度付きキュー(ヒープ)を用いて効率的に実装される。以下に疑似コードを示す:

\begin{verbatim}
function dijkstra(graph, source):
    dist = {node: infinity for node in graph.nodes}
    dist[source] = 0
    priority_queue = [(0, source)]
    visited = set()

    while priority_queue is not empty:
        current_dist, current_node = pop_min(priority_queue)

        if current_node in visited:
            continue

        visited.add(current_node)

        for neighbor in graph.neighbors(current_node):
            edge_weight = graph.edge_weight(current_node, neighbor)
            distance = current_dist + edge_weight

            if distance < dist[neighbor]:
                dist[neighbor] = distance
                push(priority_queue, (distance, neighbor))

    return dist
\end{verbatim}

計算量は、ヒープ操作がO(log V)、すべてのエッジを走査するのがO(E)であるため、合計O((V + E) log V)である。

\subsubsection{A*アルゴリズムの実装}

A*アルゴリズムは、ヒューリスティック関数$h(n)$を用いてDijkstra法を拡張する:

\begin{equation}
    f(n) = g(n) + h(n)
\end{equation}

ここで、$g(n)$は始点からノード$n$までの実際の距離、$h(n)$はノード$n$から目標ノードまでの推定距離である。

本システムでは、ヒューリスティック関数としてユークリッド距離を使用する:

\begin{equation}
    h(n) = \sqrt{(x_n - x_{\text{goal}})^2 + (y_n - y_{\text{goal}})^2}
\end{equation}

ユークリッド距離は許容的ヒューリスティック(admissible heuristic)であり、常に真の距離以下となるため、A*アルゴリズムは最適解を保証する。

\subsection{容量制約付きVRPの定式化}

容量制約付き車両経路問題(CVRP)は、以下のように数理計画問題として定式化される。

\subsubsection{記号の定義}

\begin{itemize}
    \item $V$: ノードの集合($V = \{0, 1, ..., n\}$、0はデポ)
    \item $E$: エッジの集合
    \item $c_{ij}$: ノード$i$からノード$j$への移動コスト(距離)
    \item $q_i$: ノード$i$での需要量
    \item $Q$: 車両の容量
    \item $K$: 車両の台数
    \item $x_{ijk}$: 車両$k$がエッジ$(i,j)$を使用する場合1、それ以外0の決定変数
\end{itemize}

\subsubsection{目的関数}

総移動コストの最小化:
\begin{equation}
    \min \sum_{k=1}^{K} \sum_{i \in V} \sum_{j \in V} c_{ij} x_{ijk}
\end{equation}

\subsubsection{制約条件}

各顧客は1回だけ訪問される:
\begin{equation}
    \sum_{k=1}^{K} \sum_{j \in V} x_{ijk} = 1, \quad \forall i \in V \setminus \{0\}
\end{equation}

各車両の経路の流れ保存:
\begin{equation}
    \sum_{i \in V} x_{ihk} - \sum_{j \in V} x_{hjk} = 0, \quad \forall h \in V, \forall k \in K
\end{equation}

容量制約:
\begin{equation}
    \sum_{i \in V} \sum_{j \in V} q_j x_{ijk} \leq Q, \quad \forall k \in K
\end{equation}

各車両はデポから出発・帰着:
\begin{equation}
    \sum_{j \in V \setminus \{0\}} x_{0jk} = 1, \quad \forall k \in K
\end{equation}
\begin{equation}
    \sum_{i \in V \setminus \{0\}} x_{i0k} = 1, \quad \forall k \in K
\end{equation}

部分巡回路除去制約(サブツアー除去):
\begin{equation}
    \sum_{i \in S} \sum_{j \in S} x_{ijk} \leq |S| - 1, \quad \forall S \subset V \setminus \{0\}, |S| \geq 2, \forall k \in K
\end{equation}

\subsection{コスト計算の詳細}

\subsubsection{変動費の距離単価換算}

変動費の各項目を距離単価(円/km)に換算する計算式を以下に示す。

燃料費:
\begin{equation}
    C_{\text{fuel}} = \frac{P_{\text{fuel}}}{F}
\end{equation}
ここで、$P_{\text{fuel}}$は燃料単価(円/L)、$F$は燃費(km/L)である。

人件費:
\begin{equation}
    C_{\text{labor}} = \frac{W_{\text{hour}}}{V_{\text{avg}}}
\end{equation}
ここで、$W_{\text{hour}}$は時給(円/時)、$V_{\text{avg}}$は平均速度(km/時)である。

作業時間人件費:
\begin{equation}
    C_{\text{work}} = \frac{T_{\text{work}} \times W_{\text{hour}}}{D_{\text{avg}}}
\end{equation}
ここで、$T_{\text{work}}$は1回あたりの作業時間(時)、$D_{\text{avg}}$は平均運搬距離(km)である。

\subsubsection{固定費の距離単価換算}

年間固定費を距離単価に換算する計算式を以下に示す。

\begin{equation}
    C_{\text{fixed/km}} = \frac{C_{\text{fixed/year}}}{D_{\text{annual}}}
\end{equation}
ここで、$C_{\text{fixed/year}}$は年間固定費(円/年)、$D_{\text{annual}}$は年間走行距離(km/年)である。

年間走行距離は以下のように推定される:
\begin{equation}
    D_{\text{annual}} = D_{\text{daily}} \times N_{\text{days}}
\end{equation}
ここで、$D_{\text{daily}}$は1日あたり平均走行距離(km/日)、$N_{\text{days}}$は年間稼働日数(日/年)である。

\subsubsection{総コストの計算}

ある経路の総コストは、以下の式で計算される:

\begin{equation}
    C_{\text{total}} = D \times (C_{\text{var}} + C_{\text{fixed/km}})
\end{equation}
ここで、$D$は総走行距離(km)、$C_{\text{var}}$は変動費単価(円/km)、$C_{\text{fixed/km}}$は固定費の距離単価(円/km)である。

より詳細には:
\begin{equation}
    C_{\text{total}} = D \times \left( \sum_{i=1}^{12} C_{\text{var},i} + \frac{\sum_{j=1}^{13} C_{\text{fixed},j}}{D_{\text{annual}}} \right)
\end{equation}
ここで、$C_{\text{var},i}$は変動費の第$i$項目(12項目)、$C_{\text{fixed},j}$は固定費の第$j$項目(13項目)である。

\subsection{地図可視化の実装}

地図可視化は、Foliumライブラリを用いて実装されている。主要な機能は以下の通りである。

\subsubsection{ベースマップの作成}
\begin{verbatim}
import folium

map_center = [36.2345, 139.3456]  # 中心座標
m = folium.Map(location=map_center, zoom_start=13)
\end{verbatim}

\subsubsection{経路の描画}
\begin{verbatim}
route_coords = [(lat1, lon1), (lat2, lon2), ...]
folium.PolyLine(
    route_coords,
    color='blue',
    weight=5,
    opacity=0.7,
    popup='Route 1: 15.5 km, 3,250 yen'
).add_to(m)
\end{verbatim}

\subsubsection{マーカーの配置}
\begin{verbatim}
folium.Marker(
    location=[36.2345, 139.3456],
    popup='Depot',
    icon=folium.Icon(color='red', icon='home')
).add_to(m)
\end{verbatim}

\subsection{性能最適化}

システムの性能を向上させるため、以下の最適化を実施している:

\subsubsection{距離行列のキャッシュ}
頻繁に使用されるノード間の最短距離を事前計算しキャッシュすることで、繰り返し計算を回避する。

\begin{verbatim}
@lru_cache(maxsize=10000)
def get_shortest_distance(node1, node2):
    return nx.shortest_path_length(
        graph, node1, node2, weight='length'
    )
\end{verbatim}

\subsubsection{グラフの前処理}
道路ネットワークを読み込む際、不要なノードやエッジを削除し、グラフを簡略化する。

\subsubsection{並列計算}
複数の車両経路を独立に計算できる場合、Pythonの\texttt{multiprocessing}を使用して並列処理を実施する。

\subsection{エラー処理とバリデーション}

システムの堅牢性を確保するため、以下のエラー処理とバリデーションを実装している:

\begin{itemize}
    \item 入力データの妥当性チェック(座標範囲、資源量の正負など)
    \item 道路ネットワークの連結性確認(すべてのノードが到達可能か)
    \item 車両・資源適合性の事前検証
    \item 容量制約の実現可能性チェック(総需要が車両容量を超えていないか)
    \item 計算時間の上限設定と早期終了
\end{itemize}

これらの技術的詳細により、システムの正確性、効率性、堅牢性が確保されている。
