% 第3章 システムの詳細
\section{システムの詳細}

本章では、システムを構成する主要要素について詳細に説明する。

\subsection{資源}

本システムでは、群馬県内で発生する11種類の未利用資源を対象としている。各資源の特性と収集における注意点を以下に示す。

\subsubsection{建設廃材}
建設・解体工事から発生する木材、金属、コンクリートがらなどを含む。比重が大きく重量物が多いため、積載重量に注意が必要である。混合廃棄物の場合は分別処理が必要となる。

\subsubsection{農業残渣}
稲わら、もみ殻、麦わらなど農作物収穫後の残留物である。軽量でかさばる特性を持ち、乾燥状態と湿潤状態で性状が大きく異なる。運搬時は飛散防止対策が重要である。

\subsubsection{林業残材}
間伐材、枝葉、樹皮、製材端材などを指す。長尺物が多く、水分含有率により重量が大きく変動する。積載時は荷崩れ防止のための固定が必要である。

\subsubsection{食品廃棄物}
生ごみ、食品残渣、売れ残り食品などを含む。水分含有率が高く腐敗しやすいため、迅速な運搬が求められる。汁漏れ防止のため、密閉容器の使用または専用車両が必要である。

\subsubsection{廃プラスチック}
容器包装、フィルム、硬質プラスチックなどを含む。軽量でかさばり、風で飛散しやすいため、シート養生または密閉車両での運搬が推奨される。

\subsubsection{金属スクラップ}
鉄くず、非鉄金属、廃家電から回収した金属などを含む。比重が極めて大きいため、積載重量に特に注意が必要である。クレーン付き車両の使用が効率的な場合が多い。

\subsubsection{古紙・段ボール}
新聞紙、雑誌、段ボール箱などを含む。水濡れ厳禁であり、圧縮により積載効率が向上する。雨天時はシート養生または箱型車両の使用が必須である。

\subsubsection{剪定枝・草}
庭木剪定枝、草刈り残渣、落ち葉などを含む。かさばるため、破砕処理により積載効率が大幅に向上する。パッカー車による圧縮収集も有効である。

\subsubsection{家畜糞尿}
牛糞、豚糞、鶏糞などを含む。半固形から液状まで性状が多様であり、臭気対策が必須である。液状のものはバキューム車、半固形のものはダンプまたは密閉コンテナで運搬する。

\subsubsection{下水汚泥}
浄化槽汚泥、し尿などを含む。液状から泥状であり、専用車両(バキューム車)での運搬が法令により義務付けられている。

\subsubsection{廃食用油}
使用済み天ぷら油、業務用廃油などを含む。液体であるため、漏洩防止のため密閉容器(缶・ペットボトル)での回収を前提とする。容器ごと運搬することで、平ボディやウイング車でも運搬可能である。

\vspace{1em}

これらの資源は、その物理的・化学的特性に応じて適切な運搬車両を選択する必要がある。

\subsection{運搬車両}

本システムでは、未利用資源の収集運搬に使用される14種類の車両を対象としている。各車両の特徴と主な用途を以下に分類して示す。

\subsubsection{小型車両(2-3tクラス)}

\paragraph{軽トラック}
最大積載量350kg程度、車両総重量2t未満の小型貨物車である。狭い道路での機動性に優れ、普通免許で運転可能である。少量の資源回収や農村部での巡回収集に適している。

\paragraph{2tトラック(平ボディ)}
最大積載量2t、荷台がフラットな汎用トラックである。荷台寸法は約3.1m×1.6m×0.4m(長さ×幅×高さ)であり、様々な形状の資源を積載可能で、側面からの積み下ろしが容易である。小規模事業所からの資源回収や市街地での収集業務に使用される。

\paragraph{2tダンプ}
油圧装置により荷台を傾斜させて荷下ろしする車両である。土砂、廃棄物など流動性のある資源に適しており、建設廃材、剪定枝、家畜糞尿などの運搬に使用される。

\subsubsection{中型車両(4tクラス)}

\paragraph{4t平ボディ}
最大積載量4t程度、車両総重量8t未満の中型トラックである。荷台寸法は約5.0-6.2m×2.1-2.2m×0.6mであり、運搬効率と機動性のバランスが良く、最も汎用性が高い。中規模事業所からの資源回収や中距離輸送に適している。

\paragraph{4tダンプ}
2tダンプよりも大容量で、建設現場などでの大量運搬に適する。建設廃材、解体材、土砂類の大量運搬に使用される。

\paragraph{4tユニック車(クレーン付きトラック)}
小型クレーンを搭載し、重量物の積み下ろしが可能な車両である。クレーン搭載により荷台容積は減少するが、重量物の単独作業が可能である。金属スクラップや建設廃材など重量物の運搬に使用される。

\paragraph{4tウイング車}
側面が翼のように開閉する箱型トラックである。側面全開により積み下ろし作業性が極めて良好で、雨風から荷物を保護できる。古紙、廃プラスチック、食品廃棄物など多様な資源の運搬に使用される。

\paragraph{4tパッカー車(塵芥車)}
圧縮機構を備えた廃棄物収集専用車両である。回転板やプレスにより廃棄物を圧縮し、積載効率が高い。食品廃棄物、廃プラスチック、古紙、剪定枝などの収集に使用される。

\paragraph{4tアームロール車}
コンテナを脱着できる装置を備えた車両である。複数のコンテナを準備することで車両の稼働率を向上でき、様々な資源に対応可能である。建設廃材、産業廃棄物、多種類の資源の効率的収集に使用される。

\subsubsection{大型車両(10tクラス)}

\paragraph{10t平ボディ}
最大積載量10t程度、車両総重量25t未満の大型トラックである。荷台寸法は約9m×2.4m×0.5mであり、大量輸送に適し、長距離輸送でも効率的である。大規模施設からの資源回収、広域収集、長距離輸送に使用される。

\paragraph{10tダンプ}
建設廃材や土砂の大量輸送に最適である。解体現場や大規模工事現場からの廃材運搬に使用される。

\paragraph{10tウイング車}
大容量と作業性を両立し、最も効率的な大型運搬車両である。古紙、廃プラスチック、パレット化された資源の大量輸送に使用される。

\paragraph{大型パッカー車}
圧縮機構を備えた大型廃棄物収集車である。大量の廃棄物を効率的に圧縮収集でき、広域での一般廃棄物収集や大規模イベントでの廃棄物回収に使用される。

\subsubsection{特殊用途車両}

\paragraph{バキューム車(4tクラス)}
真空ポンプにより液体・泥状物を吸引収集する専用車両である。液状・泥状の資源専用で、密閉タンク構造を持つ。家畜糞尿、下水汚泥、食品廃液の収集運搬に使用される。

\vspace{1em}

これらの車両の諸元(積載容積、燃費、購入費用など)は、実際のメーカーカタログおよび業界データに基づいて詳細に設定されている(付録参照)。

\subsection{資源・運搬車両適合性}

車両と資源の適合性は、物理的適合性、性状適合性、作業効率、法令遵守、安全性の観点から総合的に判定されている。

\subsubsection{適合判定基準}

\paragraph{適合(○)の判定基準}
\begin{itemize}
    \item \textbf{物理的適合性}:資源の形状・サイズが荷台に収まる
    \item \textbf{性状適合性}:資源の性状(固体/液体/粉体)が車両構造に適している
    \item \textbf{作業効率}:積み下ろし作業が現実的な時間・労力で実施可能
    \item \textbf{法令遵守}:廃棄物処理法等の関連法規に抵触しない
    \item \textbf{安全性}:運搬中の落下・飛散・漏洩リスクが許容範囲内
\end{itemize}

\paragraph{条件付き適合($\triangle$)の判定基準}
追加の設備または対策を講じることにより運搬が可能となる組み合わせである。例えば、密閉容器の使用、防風シートの設置、防臭対策の実施などが条件として設定される。

\paragraph{不適合(×)の判定基準}
\begin{itemize}
    \item \textbf{構造的不適合}:車両の構造上、当該資源の運搬が不可能
    \item \textbf{性状不適合}:資源の性状が車両に適さない(例:液体をダンプで運搬)
    \item \textbf{衛生上の問題}:食品廃棄物を開放型荷台で運搬など
    \item \textbf{経済性の欠如}:著しく非効率(例:農業残渣をユニック車で運搬)
    \item \textbf{環境汚染リスク}:運搬中の漏洩により環境汚染の恐れ
\end{itemize}

\subsubsection{主要な判定例}
\begin{itemize}
    \item \textbf{ダンプ×食品廃棄物}:水分が多く汁漏れのリスク、開放型荷台では衛生上不適切
    \item \textbf{パッカー車×金属スクラップ}:圧縮機構が金属により破損する恐れ、重量物には不向き
    \item \textbf{ユニック車×農業残渣}:軽量でかさばる資源にクレーンは不要、経済性に欠ける
    \item \textbf{バキューム車○下水汚泥}:液状・泥状物専用車両として最適、密閉構造で臭気対策も万全
    \item \textbf{アームロール$\triangle$家畜糞尿}:密閉コンテナ使用により臭気を抑制、コンテナ交換で効率的
\end{itemize}

\vspace{1em}

この適合性マトリックス(14種類の車両×11種類の資源、計154組み合わせ)は、システム内で自動的に参照され、ユーザーが選択した車両と資源の組み合わせが適切かどうかを判定する。不適合な組み合わせの場合は警告が表示され、条件付き適合の場合は必要な条件が提示される。これにより、実現不可能な運搬計画の立案を未然に防ぐことができる。

\subsection{最適化アルゴリズム}

本システムでは、未利用資源の収集運搬経路を最適化するために、車両経路問題(Vehicle Routing Problem, VRP)の解法を実装している。

\subsubsection{問題の定式化}

本システムが扱う問題は、以下のように定式化される:

\begin{itemize}
    \item \textbf{入力}:
    \begin{itemize}
        \item デポ(車両の出発地点・帰着地点)
        \item 複数の回収地点(未利用資源の発生場所)
        \item 集積場所(資源の最終目的地)
        \item 道路ネットワーク(ノードとエッジのグラフ構造)
        \item 車両の容量制約
        \item 各回収地点での資源量
    \end{itemize}
    \item \textbf{目的}:
    \begin{itemize}
        \item 総走行距離の最小化
        \item 運搬費用の最小化
        \item 車両台数の最小化
    \end{itemize}
    \item \textbf{制約}:
    \begin{itemize}
        \item すべての回収地点を訪問すること
        \item 車両の積載容量を超えないこと
        \item すべての車両がデポから出発し、デポに帰着すること
        \item 資源は最終的に集積場所に運搬されること
    \end{itemize}
\end{itemize}

\subsubsection{最短経路探索アルゴリズム}

本システムでは、グラフ理論に基づく最短経路探索アルゴリズムを採用している。

\paragraph{Dijkstra法}
Dijkstra法は、単一始点からすべてのノードへの最短経路を求める古典的なアルゴリズムである。計算量はO((V + E) log V)(Vはノード数、Eはエッジ数)であり、負の重みがない場合に正確な解を得られる。

本システムでは、NetworkXライブラリの\texttt{dijkstra\_path}関数を使用して実装している。デポから各回収地点、回収地点間、回収地点から集積場所への最短経路を事前計算し、これを基に全体の経路を構築する。

\paragraph{A*アルゴリズム}
A*アルゴリズムは、ヒューリスティック関数を用いてDijkstra法を拡張したアルゴリズムである。目標ノードまでの推定距離(ヒューリスティック値)を利用することで、探索の効率を向上させる。

本システムでは、ユークリッド距離をヒューリスティック関数として使用し、NetworkXの\texttt{astar\_path}関数により実装している。

\subsubsection{車両経路問題の解法}

\paragraph{最近傍法(Nearest Neighbor Heuristic)}
初期解の構築に使用される貪欲法である:
\begin{enumerate}
    \item デポから最も近い未訪問の回収地点を訪問
    \item 現在地から最も近い未訪問の回収地点を訪問
    \item 積載容量に達したら集積場所へ運搬
    \item すべての回収地点を訪問するまで繰り返す
\end{enumerate}

計算量はO(n²)(nは回収地点数)であり、高速に解を得られるが、最適解を保証しない。

\paragraph{2-opt法による改善}
最近傍法で得られた初期解を改善するための局所探索法である:
\begin{enumerate}
    \item 経路中の2つのエッジを選択
    \item エッジを入れ替えた場合の経路長を計算
    \item 改善があれば入れ替えを採用
    \item 改善がなくなるまで繰り返す
\end{enumerate}

この手法により、交差する経路を解消し、経路長を短縮できる。

\paragraph{容量制約の処理}
各車両には積載容量の制約があり、これを考慮した経路構築が必要である。本システムでは、以下の方法で容量制約を処理している:

\begin{itemize}
    \item 回収地点を訪問する際、現在の積載量に資源量を加算
    \item 積載量が車両容量を超える場合、その回収地点は訪問せず、次の候補を選択
    \item 積載量が容量に達したら、集積場所へ運搬し積載量をリセット
    \item すべての回収地点を訪問するまでこのプロセスを繰り返す
\end{itemize}

\subsubsection{計算の効率化}

大規模な問題に対しても現実的な時間で解を得るため、以下の効率化を実施している:

\begin{itemize}
    \item \textbf{距離行列の事前計算}:すべてのノード間の最短距離を事前に計算し、キャッシュする
    \item \textbf{候補地点の絞り込み}:ヒューリスティックにより、明らかに非効率な経路を探索対象から除外
    \item \textbf{並列計算}:複数の車両の経路を独立に計算できる場合、並列処理を適用
    \item \textbf{早期終了条件}:一定時間経過後、または十分に良い解が得られた時点で探索を終了
\end{itemize}

これらのアルゴリズムにより、数十から数百の回収地点を持つ実用的な問題を、数秒から数分で解決できる。

\subsection{コスト計算}

本システムでは、未利用資源の運搬に関する総コストを詳細に推計する機能を実装している。コストは変動費と固定費に大別され、それぞれ複数の項目から構成される。

\subsubsection{変動費(走行距離に比例するコスト)}

変動費は走行距離に応じて変化する費用であり、距離単価(円/km)で表される。以下の12項目から構成される。

\paragraph{燃料費}
計算式:燃料費 = 燃料単価 ÷ 燃費

燃料単価は、軽油150円/L、ガソリン170円/L(2025年10月時点の全国平均)を使用している。燃費は車両タイプごとに設定されており、軽トラック13-16 km/L、2t-4tトラック5-11 km/L、10tトラック3.5-5 km/Lの範囲である。

例:4t平ボディ(燃費6 km/L)の場合、150円 ÷ 6 km/L = 25円/km

\paragraph{運転手人件費}
計算式:人件費 = 時給 ÷ 平均速度

時給は厚生労働省「賃金構造基本統計調査」に基づき、軽トラック1,500-1,800円、2t-4tトラック1,800-2,200円、10tトラック2,000-2,500円を設定している。平均速度は市街地走行を想定し35-50 km/hである。

その他の変動費項目として、高速道路料金、タイヤ交換費、修理費、作業時間人件費、補助員人件費、回収容器費、消耗品費、通信費、マニフェスト費用、処理費が含まれる(詳細は付録参照)。

\subsubsection{固定費(年間で発生するコスト)}

固定費は走行距離に関わらず年間で発生する費用である。総コスト算出時は、固定費を年間走行距離で除算して距離単価に換算する。以下の13項目から構成される。

\paragraph{車両購入費および減価償却費}
新車価格の市場相場(2025年10月時点)に基づき、軽トラック100-150万円、2tトラック300-500万円、4tトラック500-1,200万円、10tトラック2,000-2,800万円である。

減価償却費は定額法、耐用年数5年を前提として算出される(車両購入費 ÷ 5年)。

その他の固定費項目として、自動車税、重量税、保険(自賠責・任意)、車検費用および定期点検費用、車庫賃料、許認可費用、システム利用料、福利厚生費、社会保険料が含まれる(詳細は付録参照)。

\subsubsection{総コストの算出}

総コストは以下の式で計算される:

\begin{equation}
    \text{総コスト} = \text{変動費単価} \times \text{走行距離} + \frac{\text{年間固定費}}{\text{年間走行距離}} \times \text{走行距離}
\end{equation}

システムは、選択された車両タイプと走行距離に基づき、これらのコストを自動的に計算し、詳細な内訳とともにユーザーに提示する。これにより、運搬計画の経済性を事前に評価することが可能となる。
