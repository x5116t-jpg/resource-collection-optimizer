% 第5章 結論
\section{結論}

\subsection{達成事項のまとめ}

本研究では、群馬県内における未利用資源の収集運搬計画を支援するシステムを開発した。以下に主要な達成事項を総括する。

\subsubsection{システム開発の達成}

交通安全環境研究所で開発中の「地域交通計画立案ツール」を改良し、未利用資源運搬に特化したシステムを構築した。主な達成事項は以下の通りである:

\begin{enumerate}
    \item \textbf{包括的な車両・資源データベースの構築}
    \begin{itemize}
        \item 14種類の運搬車両の詳細諸元(積載容積、燃費、購入費用など)を整備
        \item 11種類の未利用資源の特性と取扱い要件を体系化
        \item 車両・資源の適合性マトリックス(154組み合わせ)を作成
        \item 実際のメーカーカタログおよび業界データに基づく信頼性の高いデータ
    \end{itemize}

    \item \textbf{高度な最適化アルゴリズムの実装}
    \begin{itemize}
        \item Dijkstra法およびA*アルゴリズムによる最短経路探索
        \item 容量制約付き車両経路問題(CVRP)の解決
        \item 最近傍法と2-opt法による効率的なヒューリスティック探索
        \item 数十から数百の回収地点を持つ実用的な問題を数秒から数分で解決
    \end{itemize}

    \item \textbf{詳細なコスト推計機能}
    \begin{itemize}
        \item 変動費12項目(燃料費、人件費、高速道路料金など)の詳細計算
        \item 固定費13項目(車両購入費、減価償却費、税金、保険など)の包括的算出
        \item 総コストの自動計算と内訳の明示
        \item 複数のシナリオ間でのコスト比較機能
    \end{itemize}

    \item \textbf{使いやすいWebアプリケーションの実現}
    \begin{itemize}
        \item Streamlitによる直感的なユーザーインターフェース
        \item Foliumによる地図ベースの対話的な可視化
        \item CSV、JSON、HTML形式での結果エクスポート機能
        \item 専門知識不要で利用可能な設計
    \end{itemize}
\end{enumerate}

\subsubsection{仕様要件の達成}

委託仕様書に定められた要件をすべて達成した:

\begin{itemize}
    \item ツールの改良:地域交通計画立案ツールを未利用資源運搬に対応するよう改良
    \item 資源の変数化:11種類の未利用資源をシステムに組み込み
    \item 車両の変数化:14種類の運搬車両(軽トラから10tトラックまで)に対応
    \item 経路・費用推計:単一収集場所から単一集積場所への最短経路と運搬費用を推計
\end{itemize}

\subsubsection{学術的・実用的貢献}

本研究は、以下の点で学術的・実用的な貢献をした:

\begin{itemize}
    \item 未利用資源運搬における車両・資源適合性の体系的整理
    \item 実データに基づく詳細なコストモデルの構築
    \item 地方自治体や事業者が実際に利用可能な実用的システムの提供
    \item 循環型社会形成に向けた意思決定支援ツールの実現
\end{itemize}

\subsection{今後の展望:人貨混載システムへの拡張}

本システムの成果を基盤として、今後は人貨混載(passenger-cargo mixed transportation)システムへの拡張を検討する。人貨混載は、過疎地域における公共交通の維持と物流効率化を同時に実現する革新的な輸送方式であり、地域の持続可能性向上に寄与する可能性がある。

\subsubsection{人貨混載システムの概念}

人貨混載システムとは、同一車両で旅客と貨物を同時に運搬するシステムである。具体的には以下のパターンが考えられる:

\begin{itemize}
    \item \textbf{バス・タクシーによる貨物運搬}:路線バスやデマンドタクシーの空きスペースを利用して、小口貨物や未利用資源を運搬
    \item \textbf{貨物車両による旅客運送}:宅配便や資源回収車両の帰路を利用した旅客運送
    \item \textbf{専用混載車両}:旅客と貨物の両方に対応した専用設計の車両
\end{itemize}

過疎地域では、旅客需要の減少により公共交通の維持が困難な一方、高齢者の移動手段確保は喫緊の課題である。同時に、少量分散型の貨物輸送も非効率的である。人貨混載により、これらの課題を統合的に解決できる可能性がある。

\subsubsection{技術的拡張課題}

人貨混載システムの実現には、以下の技術的拡張が必要である:

\begin{enumerate}
    \item \textbf{動的経路最適化}:旅客の予約状況に応じてリアルタイムに経路を再計算
    \item \textbf{旅客・貨物の優先度設定}:緊急性の高い移動を優先する仕組み
    \item \textbf{スペース管理アルゴリズム}:限られた車両スペースを旅客と貨物に最適配分
    \item \textbf{予約システム統合}:旅客予約と貨物配送依頼を統合的に管理
    \item \textbf{安全性確保}:旅客と貨物の物理的分離、貨物固定方法の最適化
\end{enumerate}

\subsubsection{社会的意義}

人貨混載システムの実現は、以下の社会的意義を持つ:

\begin{itemize}
    \item \textbf{過疎地域の移動手段確保}:公共交通空白地帯における高齢者等の移動支援
    \item \textbf{物流効率化}:小口貨物と未利用資源の効率的な収集運搬
    \item \textbf{CO$_2$排出削減}:旅客と貨物の統合輸送による走行距離削減
    \item \textbf{地域雇用創出}:運転手、予約管理者などの雇用機会提供
    \item \textbf{循環型社会形成}:未利用資源の効率的収集による地域資源循環促進
\end{itemize}

本研究で構築したシステムを基盤として、これらの拡張を段階的に実施することで、持続可能な地域社会の実現に貢献できると考えられる。

\vspace{1em}

\noindent
以上、本研究の成果と今後の展望について述べた。開発したシステムは、未利用資源運搬の効率化に寄与するとともに、人貨混載という新たな展開への基盤を提供するものである。
