\subsubsection{コスト計算}

本システムでは、未利用資源の運搬に関する総コストを詳細に推計する機能を実装している。コストは変動費と固定費に大別され、それぞれ複数の項目から構成される。

\paragraph{変動費(走行距離に比例するコスト)}

変動費は走行距離に応じて変化する費用であり、距離単価(円/km)で表される。以下の12項目から構成される。

\subparagraph{燃料費}
計算式:燃料費 = 燃料単価 ÷ 燃費

燃料単価は、軽油150円/L、ガソリン170円/L(2025年10月時点の全国平均)を使用している。燃費は車両タイプごとに設定されており、軽トラック13-16 km/L、2t-4tトラック5-11 km/L、10tトラック3.5-5 km/Lの範囲である。

例:4t平ボディ(燃費6 km/L)の場合、150円 ÷ 6 km/L = 25円/km

\subparagraph{運転手人件費}
計算式:人件費 = 時給 ÷ 平均速度

時給は厚生労働省「賃金構造基本統計調査」に基づき、軽トラック1,500-1,800円、2t-4tトラック1,800-2,200円、10tトラック2,000-2,500円を設定している。平均速度は市街地走行を想定し35-50 km/hである。

\subparagraph{高速道路料金}
車種区分別の基本料金を設定している:
\begin{itemize}
    \item 軽トラック・普通車:25-30円/km
    \item 中型車(2t-4t):30-50円/km
    \item 大型車(10t):60-75円/km
\end{itemize}

高速道路を使用しない場合は0円として計算される。ETC割引適用により実際の料金は変動する。

\subparagraph{タイヤ交換費}
計算式:タイヤ交換費 = (タイヤ本数 × 単価 + 工賃)÷ タイヤ寿命走行距離

軽トラック2円/km、2t-4tトラック3-5円/km、10tトラック10-15円/kmの範囲である。

\subparagraph{修理費}
年間平均修理費を年間走行距離で除算して算出する。軽トラック3-5円/km、2t-4tトラック6-12円/km、10tトラック15-25円/kmの範囲である。車両年式と使用状況により大きく変動する。

\subparagraph{作業時間人件費}
積み下ろし作業時間の人件費を平均運搬距離で除算して算出する。1回の運搬あたり積み下ろし各30分、計1時間を前提とし、軽トラック15-20円/km、2t-4tトラック25-35円/km、10tトラック30-45円/kmである。

\subparagraph{補助員人件費}
2名体制が必要な場合の追加人件費である。軽トラックは0円(1名で作業可能)、2t-4tトラックは0-30円/km(重量物の場合のみ2名体制)、10tトラックは30-45円/km(基本2名体制)である。

\subparagraph{回収容器費}
フレコンバッグ、プラスチックコンテナ等の償却費である。軽トラック5-8円/km、2t-4tトラック8-25円/km、10tトラック10-25円/kmである。アームロール車はコンテナ費用が高額となる。

\subparagraph{消耗品費}
作業用手袋、ロープ、シート、清掃用品等の費用である。軽トラック2-3円/km、2t-4tトラック3-8円/km、10tトラック8-12円/kmである。

\subparagraph{通信費}
携帯電話、業務無線、GPS利用料を走行距離で除算する。全車種で1-5円/kmである(月額通信費3,000-8,000円を想定)。

\subparagraph{マニフェスト費用}
産業廃棄物管理票の発行費用である。軽トラック3-5円/km、2t-4tトラック5-12円/km、10tトラック12-18円/kmである。産業廃棄物の場合のみ必要であり、一般廃棄物は不要である。

\subparagraph{処理費}
中間処理施設または最終処分場でのゲート料である。軽トラック20-30円/km、2t-4tトラック30-100円/km、10tトラック80-180円/kmである。資源の種類、地域、処理方法により大きく変動し、リサイクル可能資源は低額、焼却・埋立は高額となる。

\paragraph{固定費(年間で発生するコスト)}

固定費は走行距離に関わらず年間で発生する費用である。総コスト算出時は、固定費を年間走行距離で除算して距離単価に換算する。以下の13項目から構成される。

\subparagraph{車両購入費および減価償却費}
新車価格の市場相場(2025年10月時点)に基づき、軽トラック100-150万円、2tトラック300-500万円、4tトラック500-1,200万円、10tトラック2,000-2,800万円である。

減価償却費は定額法、耐用年数5年を前提として算出される(車両購入費 ÷ 5年)。税法上の貨物自動車の耐用年数は5年であるが、実際の使用年数は10-15年程度である。

\subparagraph{自動車税}
車両総重量に応じた年間税額であり、軽トラック5,000-11,000円、2tトラック15,000-22,000円、4tトラック25,000-35,000円、10tトラック40,000-60,000円である。

\subparagraph{重量税}
車検時(2年ごと)に納付する税金を年換算したものであり、軽トラック3,000-5,000円/年、2tトラック12,000-16,000円/年、4tトラック20,000-25,000円/年、10tトラック32,000-41,000円/年である。

\subparagraph{保険(自賠責・任意)}
法定義務の自賠責保険と任意保険の合計である。軽トラック5.8-20万円/年、2tトラック9.5-38万円/年、4tトラック17.5-60万円/年、10tトラック27.5-90万円/年である。任意保険は事故歴、運転者の年齢・経験により大きく変動する。

\subparagraph{車検費用および定期点検費用}
2年ごとの車検費用(法定費用除く)と法定点検(3ヶ月、6ヶ月、12ヶ月)の費用の合計を年換算したものである。軽トラック5-8万円/年、2tトラック8-13万円/年、4tトラック13-27万円/年、10tトラック22-40万円/年である。

\subparagraph{車庫賃料}
車両保管場所の年間賃料である。地域により大きく変動し、軽トラック12-24万円/年、2tトラック18-36万円/年、4tトラック24-48万円/年、10tトラック36-72万円/年である。都市部は高額、地方は低額であり、自社所有地の場合は0円である。

\subparagraph{許認可費用}
産業廃棄物収集運搬業許可等の申請・更新費用を年換算したものである。新規許可約10万円、更新約7万円(5年ごと)であり、講習会受講費も含む。

\subparagraph{システム利用料}
配車管理システム、GPS追跡システム、デジタルタコグラフ等の利用料であり、全車種で3-18万円/年である(月額3,000-15,000円)。事業規模により導入システムが異なる。

\subparagraph{福利厚生費}
健康診断、制服貸与、研修費用等であり、軽トラック15-25万円/年、2t-4tトラック20-50万円/年、10tトラック35-60万円/年である。

\subparagraph{社会保険料}
健康保険、厚生年金、雇用保険、労災保険の事業主負担分である。運転手の年収(300-500万円)の約25-30%を想定し、軽トラック80-120万円/年、2t-4tトラック100-200万円/年、10tトラック150-220万円/年である。

\paragraph{総コストの算出}

総コストは以下の式で計算される:

\begin{equation}
    \text{総コスト} = \text{変動費単価} \times \text{走行距離} + \frac{\text{年間固定費}}{\text{年間走行距離}} \times \text{走行距離}
\end{equation}

システムは、選択された車両タイプと走行距離に基づき、これらのコストを自動的に計算し、詳細な内訳とともにユーザーに提示する。これにより、運搬計画の経済性を事前に評価することが可能となる。
