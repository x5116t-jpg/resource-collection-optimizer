本付録では、システムの設計に関する技術的詳細を補足する。

\subsection{車両諸元データの詳細}

表\ref{tab:vehicle_specs}に、システムに登録されている14種類の車両の詳細諸元を示す。

\begin{table}[H]
\centering
\caption{車両諸元一覧(抜粋)}
\label{tab:vehicle_specs}
\begin{tabular}{|l|r|r|r|r|}
\hline
\textbf{車両タイプ} & \textbf{積載容積} & \textbf{燃費} & \textbf{購入費} & \textbf{年間固定費} \\
& \textbf{(m³)} & \textbf{(km/L)} & \textbf{(万円)} & \textbf{(万円)} \\
\hline
軽トラック & 1.0-1.2 & 13-16 & 100-150 & 120-180 \\
2t平ボディ & 2.0-3.0 & 6-11 & 300-500 & 180-300 \\
2tダンプ & 2.5-3.5 & 6-10 & 350-550 & 190-320 \\
4t平ボディ & 6-10 & 5-8 & 500-1,200 & 300-600 \\
4tダンプ & 7-11 & 5-7 & 600-1,300 & 320-650 \\
4tユニック & 5-8 & 4-7 & 1,000-1,800 & 400-800 \\
4tウイング & 18-25 & 5-8 & 800-1,500 & 350-700 \\
4tパッカー & 8-12 & 5-7 & 1,200-2,000 & 450-850 \\
4tアームロール & 8-15 & 5-8 & 900-1,600 & 380-750 \\
10t平ボディ & 10-13 & 3.5-5 & 2,000-2,800 & 600-1,200 \\
10tダンプ & 12-18 & 3.5-4.5 & 2,200-3,000 & 650-1,300 \\
10tウイング & 40-60 & 3.5-5 & 2,500-3,500 & 700-1,400 \\
大型パッカー & 20-30 & 3-4.5 & 2,800-4,000 & 800-1,500 \\
バキューム車(4t) & 4-6 & 4-6 & 1,500-2,500 & 500-900 \\
\hline
\end{tabular}
\end{table}

\subsection{道路ネットワークデータ構造}

システムで使用する道路ネットワークは、JSON形式で以下の構造を持つ:

\begin{verbatim}
{
  "nodes": [
    {
      "id": "node_001",
      "lat": 36.2345,
      "lon": 139.3456,
      "type": "intersection"
    },
    ...
  ],
  "edges": [
    {
      "id": "edge_001",
      "source": "node_001",
      "target": "node_002",
      "length": 1250.5,
      "road_type": "local",
      "speed_limit": 40
    },
    ...
  ]
}
\end{verbatim}

ノードは交差点や地点を表し、エッジは道路区間を表す。各エッジには長さ(m)、道路種別、速度制限が設定される。

\subsection{最適化パラメータ}

システムの最適化計算には、以下のパラメータが使用される:

\begin{itemize}
    \item \textbf{探索時間制限}:最大300秒(デフォルト60秒)
    \item \textbf{2-opt改善反復回数}:最大1000回(改善が無くなるまで)
    \item \textbf{距離行列キャッシュサイズ}:最大10,000ノード対
    \item \textbf{ヒューリスティック重み}:A*アルゴリズムで1.0(ユークリッド距離と等価)
    \item \textbf{容量制約余裕率}:95\%(安全マージン5\%)
\end{itemize}

\subsection{システム要件}

本システムを動作させるための推奨環境は以下の通りである:

\begin{itemize}
    \item \textbf{OS}:Windows 10以降、macOS 10.15以降、Linux(Ubuntu 20.04以降)
    \item \textbf{Python}:3.9以降
    \item \textbf{メモリ}:4GB以上(8GB推奨)
    \item \textbf{ストレージ}:500MB以上の空き容量
    \item \textbf{Webブラウザ}:Google Chrome、Firefox、Edge(最新版)
    \item \textbf{インターネット接続}:地図表示に必要(オフライン使用も可能)
\end{itemize}

\subsection{インストール手順}

\paragraph{Pythonパッケージのインストール}
\begin{verbatim}
pip install streamlit networkx folium pandas numpy ortools
\end{verbatim}

\paragraph{システムの起動}
\begin{verbatim}
streamlit run app.py
\end{verbatim}

または、Windows環境では\texttt{run\_app.bat}をダブルクリックする。

\subsection{データファイル構成}

システムは以下のディレクトリ構造を持つ:

\begin{verbatim}
project_root/
├── app.py                    # メインアプリケーション
├── data/
│   ├── processed/
│   │   ├── compatibility.json    # 適合性データ
│   │   └── vehicle_specs.json    # 車両諸元
│   └── networks/
│       └── default_network.json  # デフォルト道路ネットワーク
├── src/
│   ├── optimization/         # 最適化モジュール
│   ├── visualization/        # 可視化モジュール
│   └── cost/                 # コスト計算モジュール
└── claudedocs/
    └── quickstart_guide.md   # クイックスタートガイド
\end{verbatim}
