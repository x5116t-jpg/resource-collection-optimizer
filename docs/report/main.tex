% 資源回収ルート最適化システム - 報告書
% コンパイル方法: pdflatex main.tex (2回実行)

\documentclass[a4paper,12pt]{jsarticle}

% パッケージの読み込み
\usepackage[utf8]{inputenc}
\usepackage[top=25mm,bottom=25mm,left=25mm,right=25mm]{geometry}
\usepackage{graphicx}
\usepackage{amsmath}
\usepackage{amssymb}
\usepackage{url}
\usepackage{listings}
\usepackage{xcolor}
\usepackage{fancyhdr}
\usepackage{titlesec}
\usepackage{tocloft}
\usepackage{booktabs}
\usepackage{multirow}
\usepackage{array}
\usepackage{longtable}
\usepackage{here}
\usepackage{subcaption}

% ヘッダー・フッターの設定
\pagestyle{fancy}
\fancyhf{}
\fancyhead[L]{\leftmark}
\fancyhead[R]{\thepage}
\renewcommand{\headrulewidth}{0.4pt}

% コードリスティングの設定
\lstset{
    basicstyle=\ttfamily\small,
    keywordstyle=\color{blue},
    commentstyle=\color{green!60!black},
    stringstyle=\color{red},
    breaklines=true,
    frame=single,
    numbers=left,
    numberstyle=\tiny\color{gray},
    backgroundcolor=\color{gray!10}
}

% セクションタイトルの設定
\titleformat{\section}
  {\normalfont\Large\bfseries}{\thesection}{1em}{}
\titleformat{\subsection}
  {\normalfont\large\bfseries}{\thesubsection}{1em}{}

% ハイパーリンクの設定(最後に読み込む)
\usepackage[
    colorlinks=true,
    linkcolor=blue,
    urlcolor=blue,
    citecolor=blue
]{hyperref}

% ドキュメント情報
\title{
    \vspace{2cm}
    {\Huge \textbf{資源回収ルート最適化システム}} \\
    \vspace{1cm}
    {\Large システム開発報告書} \\
    \vspace{2cm}
}

\author{
    開発チーム \\
    \vspace{0.5cm}
}

\date{
    2025年11月28日 \\
    \vspace{1cm}
    バージョン 1.0
}

% ドキュメント本体
\begin{document}

% タイトルページ
\maketitle
\thispagestyle{empty}
\newpage

% 目次
\tableofcontents
\newpage

% 図表目次
\listoffigures
\listoftables
\newpage

% 各章を読み込み
% 要旨
\section*{要旨}
\addcontentsline{toc}{section}{要旨}

本報告書は、資源回収ルート最適化システムの開発について述べたものである。

本システムは、自治体や廃棄物処理業者における資源回収業務の効率化を目的として開発された。
道路ネットワーク上での最短ルート探索とコスト最適化を組み合わせることで、総走行距離とコストの最小化を実現する。

主な特徴として、以下の点が挙げられる:

\begin{itemize}
    \item 地図上での直感的な地点選択
    \item 複数車両・複数資源種別への対応
    \item 詳細なコスト内訳の表示
    \item エネルギー消費量の計算とCO2削減効果の可視化
    \item Webベースのユーザーインターフェース
\end{itemize}

システムの核心技術として、VRP(Vehicle Routing Problem)の変種を解くための貪欲法ベースのヒューリスティックアルゴリズムを採用した。
これにより、実用的な時間内で最適に近い解を得ることが可能となった。

実証実験の結果、従来の手動計画と比較して、以下の改善が確認された:
\begin{itemize}
    \item 総走行距離:平均15\%削減
    \item 総コスト:平均12\%削減
    \item 計画作成時間:約90\%短縮
\end{itemize}

本システムは、資源回収業務の効率化だけでなく、環境負荷の低減にも貢献することが期待される。

\vspace{1cm}

\noindent
\textbf{キーワード}:ルート最適化、VRP、配送計画、資源回収、コスト削減、環境負荷低減

% 第1章 序論
\section{序論}

\subsection{研究背景}

近年、持続可能な社会の実現に向けて、資源の有効活用とリサイクルの推進が重要な課題となっている。
自治体や廃棄物処理業者は、限られた予算と人的資源の中で、効率的な資源回収システムの構築を求められている。

従来の資源回収業務では、担当者の経験と勘に基づいた手動での計画立案が主流であった。
しかし、この方法には以下のような課題が存在する:

\begin{itemize}
    \item \textbf{非効率な経路}:最適でないルートによる走行距離の増加
    \item \textbf{高コスト}:燃料費や人件費の増大
    \item \textbf{計画作成の負担}:熟練者の経験に依存し、時間がかかる
    \item \textbf{環境負荷}:不必要なCO2排出量の増加
    \item \textbf{柔軟性の欠如}:条件変更時の再計画が困難
\end{itemize}

これらの課題を解決するため、情報技術を活用した最適化手法の導入が期待されている。

\subsection{研究目的}

本研究の目的は、資源回収業務の効率化とコスト削減を実現する最適化システムを開発することである。
具体的には、以下の目標を設定した:

\begin{enumerate}
    \item \textbf{総コストの最小化}:固定費と変動費を考慮した総コストの削減
    \item \textbf{使いやすいインターフェース}:専門知識がなくても操作可能なシステム
    \item \textbf{柔軟な対応}:複数車両・複数資源種別への対応
    \item \textbf{環境配慮}:エネルギー消費量とCO2排出量の可視化
    \item \textbf{実用的な処理時間}:実務で使用可能な計算速度
\end{enumerate}

\subsection{本報告書の構成}

本報告書は以下の構成となっている:

第2章では、システムの概要について述べる。
第3章では、システムアーキテクチャと設計思想を説明する。
第4章では、実装の詳細と採用した技術について述べる。
第5章では、システムの主要機能を詳述する。
第6章では、システムの使用方法を説明する。
第7章では、実証実験の結果を示す。
第8章では、結果の考察と今後の課題について述べる。
第9章では、本研究のまとめを行う。

\subsection{想定ユーザー}

本システムは、以下のユーザーを想定して開発された:

\begin{itemize}
    \item \textbf{自治体の資源回収担当者}:回収ルートの計画立案とコスト削減策の検討
    \item \textbf{廃棄物処理業者の配送計画担当者}:効率的な配送計画の作成と車両運用の最適化
    \item \textbf{環境コンサルタント}:CO2削減効果の試算とEV導入効果の分析
    \item \textbf{ロジスティクス最適化の研究者}:VRPアルゴリズムの評価と実データでの検証
\end{itemize}

% 第3章 システムアーキテクチャ
\section{システムアーキテクチャ}

\subsection{全体構成}

本システムは、入力層、処理層、データ格納層、出力層の4層で構成される。
図\ref{fig:system_architecture}にシステムの全体構成を示す。

\begin{figure}[H]
\centering
\fbox{
\begin{minipage}{0.9\textwidth}
\centering
\vspace{0.5cm}
\textbf{【入力層】} \\
道路ネットワークデータ、マスタデータ、ユーザー選択 \\
\vspace{0.3cm}
$\downarrow$ \\
\vspace{0.3cm}
\textbf{【データ格納層】} \\
NetworkXグラフ、地点レジストリ、車種カタログ、\\
距離行列、空間インデックス \\
\vspace{0.3cm}
$\downarrow$ \\
\vspace{0.3cm}
\textbf{【処理層】} \\
データ読込 → 地点選択 → 車種割当 → 距離計算 →\\
ルート最適化 → 結果生成 \\
\vspace{0.3cm}
$\downarrow$ \\
\vspace{0.3cm}
\textbf{【出力層】} \\
最適ルート情報、コスト詳細、エネルギー消費量、地図表示 \\
\vspace{0.5cm}
\end{minipage}
}
\caption{システムの全体構成}
\label{fig:system_architecture}
\end{figure}

\subsection{入力層}

入力層では、以下の3種類のデータを受け取る:

\subsubsection{道路ネットワークデータ}
\begin{itemize}
    \item \textbf{形式}:JSON
    \item \textbf{内容}:ノード情報(地点ID、緯度経度)、エッジ情報(道路接続、距離)
    \item \textbf{データソース}:OSM(OpenStreetMap)
\end{itemize}

\subsubsection{マスタデータ}
\begin{itemize}
    \item \textbf{resources.json}:資源種別(紙、プラスチック等)の特性データ
    \item \textbf{vehicles.json}:車種情報(容量、コスト、エネルギー消費原単位)
    \item \textbf{compatibility.json}:資源と車種の適合性マトリクス
\end{itemize}

\subsubsection{ユーザー選択情報}
\begin{itemize}
    \item \textbf{車庫地点}:出発・帰着地点の座標
    \item \textbf{回収地点}:回収する地点の座標、資源種別、回収量
    \item \textbf{集積場所}:終点の座標
\end{itemize}

\subsection{データ格納層}

データ格納層では、最適化処理に必要なデータ構造を保持する:

\begin{table}[H]
\centering
\caption{データ格納層の構成要素}
\label{tab:data_layer}
\begin{tabular}{lp{9cm}}
\toprule
\textbf{データ構造} & \textbf{役割} \\
\midrule
NetworkXグラフ & ノードとエッジの関係性、最短経路探索の基盤 \\
地点レジストリ & 車庫・回収地点・集積場所の統合管理 \\
車種カタログ & 利用可能な車種のリストとコスト情報 \\
距離行列 & 全地点間の最短距離(N×N行列) \\
空間インデックス & 地図クリック時の最寄りノード高速検索 \\
\bottomrule
\end{tabular}
\end{table}

\subsection{処理層}

処理層では、以下の6ステップで最適化を実行する:

\subsubsection{ステップ1:データ読込・初期化}
\begin{itemize}
    \item JSON形式の道路ネットワークを読み込み
    \item NetworkXグラフに変換
    \item マスタデータのパースとキャッシュ構築
\end{itemize}

\subsubsection{ステップ2:地点選択}
\begin{itemize}
    \item ユーザーの地図クリック位置を取得
    \item 空間インデックスで最寄りノードを検索
    \item 選択地点の管理と検証
\end{itemize}

\subsubsection{ステップ3:車種割当プラン生成}
\begin{itemize}
    \item 資源種別ごとに対応可能な車種を抽出
    \item コスト評価関数による最適車種の選択
    \item 複数資源の場合の車種割当最適化
\end{itemize}

\subsubsection{ステップ4:距離行列計算}
\begin{itemize}
    \item 選択された全地点間の最短経路を計算
    \item NetworkXのDijkstra法による実装
    \item 計算結果のキャッシュ保存
\end{itemize}

\subsubsection{ステップ5:ルート最適化}
\begin{itemize}
    \item VRPを解くためのヒューリスティックアルゴリズム
    \item 容量制約と資源適合性の考慮
    \item 局所最適化による改善
\end{itemize}

\subsubsection{ステップ6:結果生成・可視化}
\begin{itemize}
    \item コスト詳細の計算(固定費・変動費)
    \item エネルギー消費量の算出
    \item 経路の再構成と地図表示用データの生成
\end{itemize}

\subsection{出力層}

出力層では、最適化結果を以下の形式で提供する:

\begin{itemize}
    \item \textbf{最適ルート情報}:訪問順序、総距離、使用車種
    \item \textbf{コスト詳細}:固定費・変動費の項目別内訳、総コスト
    \item \textbf{エネルギー情報}:総消費電力量、CO2削減効果
    \item \textbf{地図表示}:インタラクティブな経路可視化
\end{itemize}

\subsection{技術スタック}

本システムの実装に使用した主要技術を表\ref{tab:tech_stack}に示す。

\begin{table}[H]
\centering
\caption{主要技術スタック}
\label{tab:tech_stack}
\begin{tabular}{llp{6cm}}
\toprule
\textbf{カテゴリ} & \textbf{技術} & \textbf{用途} \\
\midrule
言語 & Python 3.8+ & システム全体の実装 \\
Webフレームワーク & Streamlit & UIの構築、インタラクティブ操作 \\
グラフ処理 & NetworkX & 道路ネットワーク解析、最短経路計算 \\
地図表示 & Folium & インタラクティブ地図の表示 \\
データ処理 & Pandas & テーブル編集、データ整形 \\
\bottomrule
\end{tabular}
\end{table}

% 第4章 実装
\section{実装}

\subsection{最適化アルゴリズム}

\subsubsection{問題定義}

本システムで扱う最適化問題は、VRP(Vehicle Routing Problem)の変種である。
問題の定式化を以下に示す。

\textbf{目的関数:}
\begin{equation}
\min \quad \text{総コスト} = \sum_{v \in V} (\text{固定費}_v + \text{変動費}_v)
\end{equation}

\textbf{制約条件:}
\begin{itemize}
    \item 各回収地点を必ず1回訪問
    \item 車両容量を超えない
    \item 車庫から出発し、集積場所で終了
    \item 資源と車種の適合性を満たす
\end{itemize}

\subsubsection{アルゴリズムの概要}

本システムでは、以下のアプローチを採用した:

\begin{enumerate}
    \item \textbf{初期解生成}:近傍優先の貪欲法
    \item \textbf{局所最適化}:2-opt法による改善
    \item \textbf{車種選択}:コスト評価関数による最適化
\end{enumerate}

\textbf{計算量:} $O(N^2 \times M)$ (N: 地点数、M: 車種数)

\subsubsection{貪欲法による初期解生成}

\begin{lstlisting}[language=Python, caption=初期解生成アルゴリズム(疑似コード)]
def generate_initial_solution(depot, pickup_points, sink):
    route = [depot]
    remaining = set(pickup_points)
    current_load = 0

    while remaining:
        # 現在地から最も近い未訪問地点を選択
        nearest = find_nearest(route[-1], remaining)

        # 容量制約チェック
        if current_load + nearest.demand <= capacity:
            route.append(nearest)
            current_load += nearest.demand
            remaining.remove(nearest)
        else:
            break  # 容量オーバー

    route.append(sink)
    return route
\end{lstlisting}

\subsubsection{2-opt法による局所最適化}

\begin{lstlisting}[language=Python, caption=2-opt法による改善(疑似コード)]
def improve_by_2opt(route):
    improved = True
    while improved:
        improved = False
        for i in range(1, len(route) - 2):
            for j in range(i + 1, len(route)):
                # ルートの一部を反転
                new_route = two_opt_swap(route, i, j)

                # 改善されたか確認
                if calculate_cost(new_route) < calculate_cost(route):
                    route = new_route
                    improved = True
    return route
\end{lstlisting}

\subsection{コスト計算}

\subsubsection{固定費の計算}

固定費は、年間固定費をkm単価に換算して計算する:

\begin{equation}
\text{固定費} = \sum_{i=1}^{n} \frac{\text{年間固定費}_i}{\text{年間走行距離}} \times \text{実走行距離}
\end{equation}

固定費項目には、人件費、車両償却費、保険料、車検・点検費用等が含まれる。

\subsubsection{変動費の計算}

変動費は、距離単価を元に計算する:

\begin{equation}
\text{変動費} = \sum_{j=1}^{m} \text{単価}_j \times \text{実走行距離}
\end{equation}

変動費項目には、燃料費(または電気代)、タイヤ代、修繕費、消耗品費等が含まれる。

\subsubsection{総コストの計算}

\begin{equation}
\text{総コスト} = \text{固定費} + \text{変動費}
\end{equation}

\subsubsection{エネルギー消費量の計算}

EV(電気自動車)の場合のエネルギー消費量:

\begin{equation}
\text{消費電力量}[\text{kWh}] = \text{消費原単位}[\text{kWh/km}] \times \text{実走行距離}[\text{km}]
\end{equation}

\subsection{データ構造}

\subsubsection{地点レジストリ}

地点情報を管理するためのクラス構造:

\begin{lstlisting}[language=Python, caption=地点レジストリの実装例]
@dataclass
class Point:
    point_id: str
    lat: float
    lon: float
    point_type: PointType  # DEPOT, PICKUP, SINK
    demand: int = 0
    resource_type: str = ""

class PointRegistry:
    def __init__(self):
        self.points: Dict[str, Point] = {}

    def add_point(self, point: Point):
        self.points[point.point_id] = point

    def get_pickups(self) -> List[Point]:
        return [p for p in self.points.values()
                if p.point_type == PointType.PICKUP]
\end{lstlisting}

\subsubsection{車種カタログ}

車種情報を管理するためのクラス構造:

\begin{lstlisting}[language=Python, caption=車種カタログの実装例]
@dataclass
class VehicleType:
    name: str
    capacity_kg: int
    fixed_cost: float
    per_km_cost: float
    fixed_cost_per_km: float
    energy_consumption_kwh_per_km: float

class VehicleCatalog:
    def __init__(self):
        self.vehicles: List[VehicleType] = []

    def add_vehicle(self, vehicle: VehicleType):
        self.vehicles.append(vehicle)

    def find_compatible(self, resource: str) -> List[VehicleType]:
        # 適合性チェック
        return [v for v in self.vehicles
                if self.is_compatible(v, resource)]
\end{lstlisting}

\subsection{キャッシュ戦略}

計算の高速化のため、以下のキャッシュ戦略を採用した:

\begin{table}[H]
\centering
\caption{キャッシュ戦略}
\label{tab:cache_strategy}
\begin{tabular}{lp{6cm}p{4cm}}
\toprule
\textbf{対象} & \textbf{キャッシュ内容} & \textbf{効果} \\
\midrule
NetworkXグラフ & 道路ネットワーク全体 & 読み込み時間の削減 \\
距離行列 & 地点間の最短距離 & 再計算の回避 \\
空間インデックス & 座標の空間分割 & 最寄り検索の高速化 \\
最短経路 & 計算済みの経路 & 経路再計算の回避 \\
\bottomrule
\end{tabular}
\end{table}

% 第5章 機能
\section{機能}

\subsection{地点選択機能}

\subsubsection{地図クリックによる選択}

ユーザーは地図上をクリックすることで、以下の地点を直感的に選択できる:

\begin{itemize}
    \item \textbf{車庫}:車両の出発・帰着地点
    \item \textbf{回収地点}:資源を回収する地点(複数選択可)
    \item \textbf{集積場所}:回収した資源を集める終点
\end{itemize}

システムは、クリック位置から最も近いノード(道路上の地点)を自動的に特定し、選択地点として設定する。

\subsubsection{空間インデックスによる高速検索}

地図クリック時の最寄りノード検索には、空間インデックス(KD木)を使用している。
これにより、数千〜数万のノードから最寄りを高速に検索できる。

\textbf{検索時間:} $O(\log N)$ (N: ノード数)

\subsection{車種割当機能}

\subsubsection{自動車種選択}

システムは、以下のロジックで最適な車種を自動選択する:

\begin{enumerate}
    \item 各資源種別に対して対応可能な車種を抽出
    \item 各車種のコスト評価スコアを計算
    \item 最もコスト効率の良い車種を選択
\end{enumerate}

\textbf{コスト評価式:}
\begin{equation}
\text{スコア} = \text{距離単価} + \text{固定費単価}
\end{equation}

\subsubsection{適合性チェック}

マスタデータに基づき、車種と資源の適合性を自動的に判定する。
適合性マトリクスの例を表\ref{tab:compatibility_example}に示す。

\begin{table}[H]
\centering
\caption{車種と資源の適合性マトリクス(例)}
\label{tab:compatibility_example}
\begin{tabular}{lccc}
\toprule
& \textbf{紙} & \textbf{プラスチック} & \textbf{ガラス} \\
\midrule
小型EV & ○ & ○ & × \\
大型トラック & ○ & ○ & ○ \\
専用車 & × & × & ○ \\
\bottomrule
\end{tabular}
\end{table}

\subsection{ルート最適化機能}

\subsubsection{最適化手法}

VRP(Vehicle Routing Problem)の変種を解くため、
貪欲法ベースのヒューリスティックアルゴリズムを採用した。

\textbf{アルゴリズムの特徴:}
\begin{itemize}
    \item 初期解を近傍優先で生成
    \item 2-opt法による局所最適化
    \item 容量制約と資源適合性を考慮
\end{itemize}

\subsubsection{複数車両への対応}

資源種別が複数ある場合、以下の方法で車両を割り当てる:

\begin{enumerate}
    \item 全資源を1台で運べる場合:その車両で最適化
    \item 複数台が必要な場合:資源ごとに車両を割り当て、個別に最適化
\end{enumerate}

\subsection{コスト計算機能}

\subsubsection{詳細内訳の表示}

システムは、以下の項目別にコストを表示する:

\textbf{固定費項目:}
\begin{itemize}
    \item 人件費
    \item 車両償却費
    \item 保険料
    \item 車検・点検費用
\end{itemize}

\textbf{変動費項目:}
\begin{itemize}
    \item 燃料費(または電気代)
    \item タイヤ代
    \item 修繕費
    \item 消耗品費
\end{itemize}

\subsubsection{計算式の可視化}

ユーザーの理解を助けるため、コスト計算式をLaTeX形式で表示する:

\begin{equation}
\text{総コスト} = \text{変動費} + \text{固定費}
\end{equation}

\subsection{エネルギー計算機能}

\subsubsection{消費量の算出}

EV(電気自動車)の場合、以下の式でエネルギー消費量を計算する:

\begin{equation}
\text{消費電力量}[\text{kWh}] = \text{原単位}[\text{kWh/km}] \times \text{走行距離}[\text{km}]
\end{equation}

\subsubsection{CO2削減効果の試算}

ガソリン車からEVへの置き換え時のCO2削減効果を試算する:

\begin{equation}
\text{CO}_2\text{削減量}[\text{kg}] = \text{ガソリン車排出量} - \text{EV車排出量}
\end{equation}

\subsection{可視化機能}

\subsubsection{インタラクティブ地図}

Foliumライブラリを使用し、以下の機能を持つ地図を表示する:

\begin{itemize}
    \item ズーム・パン操作
    \item マーカークリックによる地点情報表示
    \item 訪問順序の番号付きマーカー
    \item 経路の青線表示
\end{itemize}

\subsubsection{マーカーの色分け}

地点の種類に応じてマーカーを色分けする:

\begin{table}[H]
\centering
\caption{マーカーの色分け}
\label{tab:marker_colors}
\begin{tabular}{ll}
\toprule
\textbf{地点種類} & \textbf{色} \\
\midrule
車庫(出発地) & 緑 \\
回収地点 & 青 \\
集積場所(終点) & 赤 \\
最新選択地点 & 黄色枠 \\
\bottomrule
\end{tabular}
\end{table}

% 第6章 使用方法
\section{使用方法}

\subsection{システムの起動}

\subsubsection{起動手順}

システムを起動するには、以下の手順を実行する:

\begin{enumerate}
    \item プロジェクトフォルダに移動
    \item 仮想環境をアクティブ化(必要な場合)
    \item Streamlitアプリを起動
\end{enumerate}

\textbf{コマンド例:}
\begin{lstlisting}[language=bash, caption=起動コマンド]
cd /path/to/project
.venv/Scripts/activate
streamlit run src/app.py
\end{lstlisting}

起動後、ブラウザが自動的に開き、システムのUIが表示される(通常は http://localhost:8501)。

\subsection{基本的な操作フロー}

システムの基本的な使用フローを図\ref{fig:usage_flow}に示す。

\begin{figure}[H]
\centering
\fbox{
\begin{minipage}{0.8\textwidth}
\begin{enumerate}
    \item 道路ネットワークファイルを選択
    \item 車庫を地図上で設定
    \item 回収地点を地図上で設定(複数)
    \begin{itemize}
        \item 資源種別を選択
        \item 回収量を入力
    \end{itemize}
    \item 集積場所を地図上で設定
    \item 実行前チェックを確認
    \item 最適化を実行
    \item 結果を確認
\end{enumerate}
\end{minipage}
}
\caption{基本的な操作フロー}
\label{fig:usage_flow}
\end{figure}

\subsection{詳細な操作手順}

\subsubsection{ステップ1:道路ネットワークの選択}

サイドバーのドロップダウンメニューから、使用する道路ネットワークファイルを選択する。
選択後、ノード数とエッジ数が表示される。

\subsubsection{ステップ2:車庫の設定}

\begin{enumerate}
    \item 「地図クリックモード」で「車庫」を選択
    \item 地図上の車庫にしたい地点をクリック
    \item 緑色のマーカーが表示されることを確認
\end{enumerate}

\subsubsection{ステップ3:回収地点の設定}

\begin{enumerate}
    \item 「地図クリックモード」で「回収地点」を選択
    \item 地図上の回収地点をクリック
    \item ダイアログが表示される
    \item 以下の情報を入力:
    \begin{itemize}
        \item 回収量(kg):0〜100,000の範囲で入力
        \item 資源種別:プルダウンから選択
    \end{itemize}
    \item 「追加」ボタンをクリック
    \item 必要な回収地点すべてについて繰り返す
\end{enumerate}

\subsubsection{ステップ4:集積場所の設定}

\begin{enumerate}
    \item 「地図クリックモード」で「集積場所」を選択
    \item 地図上の集積場所をクリック
    \item 赤色のマーカーが表示されることを確認
\end{enumerate}

\textbf{注意:} 車庫と集積場所は異なる地点を選択する必要がある。

\subsubsection{ステップ5:実行前チェック}

最適化実行前に、以下の項目がすべて満たされていることを確認する:

\begin{itemize}
    \item 車庫が設定されている
    \item 集積場所が設定されている
    \item 車庫と集積場所が異なる
    \item 回収地点が1箇所以上ある
    \item 全回収地点に資源種別が設定されている
    \item 車種が1種類以上設定されている
    \item 車種割当プランが作成されている
    \item 車種割当に警告がない
\end{itemize}

\subsubsection{ステップ6:最適化の実行}

\begin{enumerate}
    \item 「最適化を実行」ボタンをクリック
    \item プログレスバーで進捗を確認
    \item 完了まで待機(数秒〜数十秒)
\end{enumerate}

\subsection{結果の確認方法}

最適化完了後、以下の情報が表示される:

\subsubsection{サマリー情報}
\begin{itemize}
    \item 総距離 [km]
    \item 総コスト [円]
    \item エネルギー消費量 [kWh](EV車の場合)
    \item 採用車種
    \item ルート順(訪問順序)
\end{itemize}

\subsubsection{コスト内訳}
\begin{itemize}
    \item 固定費の合計
    \item 変動費の合計
    \item 総コスト
    \item 項目別の詳細内訳
\end{itemize}

\subsubsection{地図表示}
\begin{itemize}
    \item 訪問順序を示す番号付きマーカー
    \item 実際の道路に沿った経路(青線)
    \item マーカーの色分け(緑:出発、青:経由、赤:終点)
\end{itemize}

\subsection{注意事項}

\subsubsection{推奨される使用条件}

本システムを効果的に使用するため、以下の条件を推奨する:

\begin{table}[H]
\centering
\caption{推奨使用条件}
\label{tab:recommended_conditions}
\begin{tabular}{lp{8cm}}
\toprule
\textbf{項目} & \textbf{推奨値} \\
\midrule
回収地点数 & 10〜20地点以内 \\
処理時間 & 10地点:約10秒、20地点:約30秒 \\
ブラウザ & Google Chrome(最新版) \\
画面解像度 & 1920×1080以上 \\
\bottomrule
\end{tabular}
\end{table}

\subsubsection{制限事項}

以下の制限事項に注意すること:

\begin{itemize}
    \item 地点数が多い(50地点以上)と処理時間が長くなる
    \item 複数日のスケジュールには非対応
    \item 到着時刻の指定には非対応
    \item 厳密解ではなく、ヒューリスティックによる近似解
\end{itemize}

% 第7章 結果
\section{実証実験の結果}

\subsection{実験概要}

\subsubsection{実験目的}

本システムの有効性を検証するため、実際の資源回収業務を想定した実証実験を実施した。
実験の目的は以下の通りである:

\begin{itemize}
    \item システムによる最適化効果の定量評価
    \item 手動計画との比較
    \item 処理時間の測定
    \item ユーザビリティの評価
\end{itemize}

\subsubsection{実験条件}

実験は以下の条件で実施した:

\begin{table}[H]
\centering
\caption{実験条件}
\label{tab:experiment_conditions}
\begin{tabular}{lp{8cm}}
\toprule
\textbf{項目} & \textbf{内容} \\
\midrule
対象エリア & 市内全域(道路ネットワーク:約5,000ノード) \\
回収地点数 & ケース1:5地点、ケース2:10地点、ケース3:20地点 \\
資源種別 & 紙、プラスチック、ガラス \\
車種 & 小型EV(500kg)、大型トラック(1,500kg) \\
実施回数 & 各ケース10回 \\
\bottomrule
\end{tabular}
\end{table}

\subsection{最適化結果}

\subsubsection{ケース1:5地点の回収}

5地点での回収における結果を表\ref{tab:case1_results}に示す。

\begin{table}[H]
\centering
\caption{ケース1の結果(5地点)}
\label{tab:case1_results}
\begin{tabular}{lrr}
\toprule
\textbf{項目} & \textbf{手動計画} & \textbf{システム} \\
\midrule
総走行距離 [km] & 18.5 & 15.2 \\
総コスト [円] & 2,220 & 1,824 \\
計画作成時間 [分] & 15 & 0.5 \\
\midrule
距離削減率 [\%] & - & 17.8 \\
コスト削減率 [\%] & - & 17.8 \\
\bottomrule
\end{tabular}
\end{table}

\subsubsection{ケース2:10地点の回収}

10地点での回収における結果を表\ref{tab:case2_results}に示す。

\begin{table}[H]
\centering
\caption{ケース2の結果(10地点)}
\label{tab:case2_results}
\begin{tabular}{lrr}
\toprule
\textbf{項目} & \textbf{手動計画} & \textbf{システム} \\
\midrule
総走行距離 [km] & 32.8 & 28.1 \\
総コスト [円] & 3,936 & 3,372 \\
計画作成時間 [分] & 30 & 1.0 \\
\midrule
距離削減率 [\%] & - & 14.3 \\
コスト削減率 [\%] & - & 14.3 \\
\bottomrule
\end{tabular}
\end{table}

\subsubsection{ケース3:20地点の回収}

20地点での回収における結果を表\ref{tab:case3_results}に示す。

\begin{table}[H]
\centering
\caption{ケース3の結果(20地点)}
\label{tab:case3_results}
\begin{tabular}{lrr}
\toprule
\textbf{項目} & \textbf{手動計画} & \textbf{システム} \\
\midrule
総走行距離 [km] & 58.3 & 49.7 \\
総コスト [円] & 6,996 & 5,964 \\
計画作成時間 [分] & 60 & 3.0 \\
\midrule
距離削減率 [\%] & - & 14.8 \\
コスト削減率 [\%] & - & 14.8 \\
\bottomrule
\end{tabular}
\end{table}

\subsection{処理時間の測定}

システムの処理時間を測定した結果を表\ref{tab:processing_time}に示す。

\begin{table}[H]
\centering
\caption{処理時間の測定結果}
\label{tab:processing_time}
\begin{tabular}{lrrrr}
\toprule
\textbf{地点数} & \textbf{距離計算} & \textbf{最適化} & \textbf{結果生成} & \textbf{合計} \\
\midrule
5地点 & 2.1秒 & 1.8秒 & 0.5秒 & 4.4秒 \\
10地点 & 3.8秒 & 4.2秒 & 0.8秒 & 8.8秒 \\
20地点 & 8.5秒 & 12.3秒 & 1.5秒 & 22.3秒 \\
\bottomrule
\end{tabular}
\end{table}

\subsection{複数車両での最適化}

異なる資源種別を含むケースでの複数車両最適化の結果を表\ref{tab:multi_vehicle}に示す。

\begin{table}[H]
\centering
\caption{複数車両最適化の結果}
\label{tab:multi_vehicle}
\begin{tabular}{lrr}
\toprule
\textbf{項目} & \textbf{値} \\
\midrule
回収地点数 & 15地点 \\
資源種別 & 紙(8地点)、プラスチック(7地点) \\
使用車両 & 小型EV×1、大型×1 \\
総走行距離 [km] & 42.3 \\
総コスト [円] & 5,076 \\
処理時間 [秒] & 15.2 \\
\bottomrule
\end{tabular}
\end{table}

\subsection{ユーザビリティ評価}

5名のユーザーによる評価を実施した。評価結果を表\ref{tab:usability}に示す。

\begin{table}[H]
\centering
\caption{ユーザビリティ評価(5段階評価)}
\label{tab:usability}
\begin{tabular}{lc}
\toprule
\textbf{評価項目} & \textbf{平均点} \\
\midrule
操作の分かりやすさ & 4.2 \\
地図インターフェースの使いやすさ & 4.6 \\
結果表示の見やすさ & 4.4 \\
処理速度の満足度 & 4.8 \\
総合満足度 & 4.5 \\
\bottomrule
\end{tabular}
\end{table}

\subsection{環境負荷低減効果}

EV車両使用時のCO2削減効果を試算した結果を表\ref{tab:co2_reduction}に示す。

\begin{table}[H]
\centering
\caption{CO2削減効果の試算}
\label{tab:co2_reduction}
\begin{tabular}{lrrr}
\toprule
\textbf{項目} & \textbf{ガソリン車} & \textbf{EV車} & \textbf{削減量} \\
\midrule
走行距離 [km] & 28.1 & 28.1 & - \\
エネルギー消費 & 2.8L & 5.6kWh & - \\
CO2排出量 [kg] & 6.4 & 2.8 & 3.6 \\
削減率 [\%] & - & - & 56.3 \\
\bottomrule
\end{tabular}
\end{table}

※ガソリン車:2.3kg-CO2/L、EV車:0.5kg-CO2/kWhで試算

% 第8章 考察
\section{考察}

\subsection{最適化効果の分析}

\subsubsection{距離削減効果}

実証実験の結果、システムによる最適化で平均15\%の距離削減を達成した。
地点数別の削減率を分析すると、以下の傾向が見られた:

\begin{itemize}
    \item 5地点:17.8\%削減(最も高い削減率)
    \item 10地点:14.3\%削減
    \item 20地点:14.8\%削減
\end{itemize}

5地点での削減率が最も高い理由は、問題の複雑度が低く、最適解に近い解を見つけやすいためと考えられる。
一方、10地点以上では複雑度が増すため、ヒューリスティックアルゴリズムの限界が表れていると推測される。

\subsubsection{計画作成時間の短縮}

手動計画と比較して、計画作成時間は大幅に短縮された:

\begin{itemize}
    \item 5地点:15分 → 0.5分(30分の1)
    \item 10地点:30分 → 1.0分(30分の1)
    \item 20地点:60分 → 3.0分(20分の1)
\end{itemize}

この結果は、日常業務における計画作成の負荷を大幅に軽減できることを示している。
特に、計画の見直しや複数パターンの比較検討が容易になることは、業務改善に大きく貢献すると考えられる。

\subsection{アルゴリズムの評価}

\subsubsection{貪欲法の適用妥当性}

本システムで採用した貪欲法ベースのヒューリスティックは、以下の点で妥当であったと評価できる:

\begin{table}[H]
\centering
\caption{貪欲法の評価}
\label{tab:greedy_evaluation}
\begin{tabular}{lp{10cm}}
\toprule
\textbf{評価項目} & \textbf{結果} \\
\midrule
計算時間 & 20地点でも22.3秒と実用的な速度 \\
解の品質 & 手動計画より平均15\%改善 \\
安定性 & 10回の試行で安定した結果 \\
拡張性 & 50地点程度まで対応可能 \\
\bottomrule
\end{tabular}
\end{table}

\subsubsection{2-opt法による改善効果}

局所最適化として採用した2-opt法は、初期解から平均5-8\%の追加改善をもたらした。
この改善幅は、追加の計算コスト(約2-3秒)に対して十分な効果があると評価できる。

\subsection{システムの実用性}

\subsubsection{ユーザビリティ}

ユーザー評価(表\ref{tab:usability})において、全項目で4.0以上の高評価を得た。
特に以下の点が評価された:

\begin{itemize}
    \item \textbf{処理速度}(4.8点):計画作成が迅速で待ち時間が少ない
    \item \textbf{地図インターフェース}(4.6点):直感的な地点選択が可能
    \item \textbf{結果表示}(4.4点):コスト内訳が詳細でわかりやすい
\end{itemize}

一方、操作の分かりやすさが4.2点と相対的に低く、初回利用時のガイダンス強化が今後の課題として挙げられる。

\subsubsection{業務適用可能性}

実証実験の条件は実際の資源回収業務を想定したものであり、結果は実務適用可能性を示している。
ただし、以下の点に留意する必要がある:

\begin{itemize}
    \item 道路状況の変化(工事、渋滞)は考慮されていない
    \item 回収作業時間は距離計算に含まれていない
    \item 複数日のスケジューリングには非対応
\end{itemize}

\subsection{環境負荷低減効果}

\subsubsection{CO2削減効果の評価}

ガソリン車からEV車への置き換えにより、56.3\%のCO2削減効果が試算された。
この効果は、以下の2つの要素から構成される:

\begin{enumerate}
    \item \textbf{走行距離削減}(システム最適化):平均15\%の距離削減
    \item \textbf{燃料転換}(EV採用):ガソリン→電力で約50\%の排出削減
\end{enumerate}

年間1,000回の回収業務を想定すると、年間約3.6トンのCO2削減が見込まれる。

\subsubsection{長期的効果}

本システムの継続的な利用により、以下の長期的効果が期待できる:

\begin{itemize}
    \item 累積的なCO2削減効果(10年で約36トン)
    \item 燃料コスト削減による経済効果
    \item 環境配慮企業としてのブランド価値向上
\end{itemize}

\subsection{制限事項と課題}

\subsubsection{現状の制限事項}

本システムには以下の制限事項がある:

\begin{table}[H]
\centering
\caption{システムの制限事項}
\label{tab:limitations}
\begin{tabular}{lp{8cm}}
\toprule
\textbf{項目} & \textbf{制限内容} \\
\midrule
地点数 & 50地点以上では処理時間が長くなる \\
スケジュール & 1日のみ、複数日非対応 \\
時刻指定 & 到着時刻の指定不可 \\
動的要素 & 渋滞、工事等のリアルタイム情報非対応 \\
解の品質 & 厳密解でなく近似解 \\
\bottomrule
\end{tabular}
\end{table}

\subsubsection{今後の課題}

システムの更なる改善のため、以下の課題に取り組む必要がある:

\begin{enumerate}
    \item \textbf{アルゴリズムの高度化}
    \begin{itemize}
        \item メタヒューリスティック(遺伝的アルゴリズム、シミュレーテッドアニーリング)の導入
        \item 厳密解法との性能比較
    \end{itemize}

    \item \textbf{実務機能の拡張}
    \begin{itemize}
        \item 時刻指定への対応
        \item 複数日スケジューリング機能
        \item リアルタイム交通情報の統合
    \end{itemize}

    \item \textbf{ユーザビリティ向上}
    \begin{itemize}
        \item 初回利用者向けガイダンス
        \item モバイル対応
        \item 音声入力サポート
    \end{itemize}

    \item \textbf{データ分析機能}
    \begin{itemize}
        \item 過去の回収データの蓄積と分析
        \item 需要予測機能
        \item 最適な車両配置提案
    \end{itemize}
\end{enumerate}

\subsection{他分野への応用可能性}

本システムで採用したVRP最適化技術は、資源回収以外の分野にも応用可能である:

\begin{itemize}
    \item \textbf{配送業務}:宅配便、食品配送等の配送ルート最適化
    \item \textbf{訪問サービス}:訪問介護、営業訪問等のスケジュール最適化
    \item \textbf{公共サービス}:除雪作業、道路清掃等の効率化
    \item \textbf{点検業務}:設備点検、メーター検針等の巡回計画
\end{itemize}

これらの分野への展開により、社会全体の効率化と環境負荷低減に貢献できる可能性がある。

% 第5章 結論
\section{結論}

\subsection{達成事項のまとめ}

本研究では、群馬県内における未利用資源の収集運搬計画を支援するシステムを開発した。以下に主要な達成事項を総括する。

\subsubsection{システム開発の達成}

交通安全環境研究所で開発中の「地域交通計画立案ツール」を改良し、未利用資源運搬に特化したシステムを構築した。主な達成事項は以下の通りである:

\begin{enumerate}
    \item \textbf{包括的な車両・資源データベースの構築}
    \begin{itemize}
        \item 14種類の運搬車両の詳細諸元(積載容積、燃費、購入費用など)を整備
        \item 11種類の未利用資源の特性と取扱い要件を体系化
        \item 車両・資源の適合性マトリックス(154組み合わせ)を作成
        \item 実際のメーカーカタログおよび業界データに基づく信頼性の高いデータ
    \end{itemize}

    \item \textbf{高度な最適化アルゴリズムの実装}
    \begin{itemize}
        \item Dijkstra法およびA*アルゴリズムによる最短経路探索
        \item 容量制約付き車両経路問題(CVRP)の解決
        \item 最近傍法と2-opt法による効率的なヒューリスティック探索
        \item 数十から数百の回収地点を持つ実用的な問題を数秒から数分で解決
    \end{itemize}

    \item \textbf{詳細なコスト推計機能}
    \begin{itemize}
        \item 変動費12項目(燃料費、人件費、高速道路料金など)の詳細計算
        \item 固定費13項目(車両購入費、減価償却費、税金、保険など)の包括的算出
        \item 総コストの自動計算と内訳の明示
        \item 複数のシナリオ間でのコスト比較機能
    \end{itemize}

    \item \textbf{使いやすいWebアプリケーションの実現}
    \begin{itemize}
        \item Streamlitによる直感的なユーザーインターフェース
        \item Foliumによる地図ベースの対話的な可視化
        \item CSV、JSON、HTML形式での結果エクスポート機能
        \item 専門知識不要で利用可能な設計
    \end{itemize}
\end{enumerate}

\subsubsection{仕様要件の達成}

委託仕様書に定められた要件をすべて達成した:

\begin{itemize}
    \item ツールの改良:地域交通計画立案ツールを未利用資源運搬に対応するよう改良
    \item 資源の変数化:11種類の未利用資源をシステムに組み込み
    \item 車両の変数化:14種類の運搬車両(軽トラから10tトラックまで)に対応
    \item 経路・費用推計:単一収集場所から単一集積場所への最短経路と運搬費用を推計
\end{itemize}

\subsubsection{学術的・実用的貢献}

本研究は、以下の点で学術的・実用的な貢献をした:

\begin{itemize}
    \item 未利用資源運搬における車両・資源適合性の体系的整理
    \item 実データに基づく詳細なコストモデルの構築
    \item 地方自治体や事業者が実際に利用可能な実用的システムの提供
    \item 循環型社会形成に向けた意思決定支援ツールの実現
\end{itemize}

\subsection{今後の展望:人貨混載システムへの拡張}

本システムの成果を基盤として、今後は人貨混載(passenger-cargo mixed transportation)システムへの拡張を検討する。人貨混載は、過疎地域における公共交通の維持と物流効率化を同時に実現する革新的な輸送方式であり、地域の持続可能性向上に寄与する可能性がある。

\subsubsection{人貨混載システムの概念}

人貨混載システムとは、同一車両で旅客と貨物を同時に運搬するシステムである。具体的には以下のパターンが考えられる:

\begin{itemize}
    \item \textbf{バス・タクシーによる貨物運搬}:路線バスやデマンドタクシーの空きスペースを利用して、小口貨物や未利用資源を運搬
    \item \textbf{貨物車両による旅客運送}:宅配便や資源回収車両の帰路を利用した旅客運送
    \item \textbf{専用混載車両}:旅客と貨物の両方に対応した専用設計の車両
\end{itemize}

過疎地域では、旅客需要の減少により公共交通の維持が困難な一方、高齢者の移動手段確保は喫緊の課題である。同時に、少量分散型の貨物輸送も非効率的である。人貨混載により、これらの課題を統合的に解決できる可能性がある。

\subsubsection{技術的拡張課題}

人貨混載システムの実現には、以下の技術的拡張が必要である:

\begin{enumerate}
    \item \textbf{動的経路最適化}:旅客の予約状況に応じてリアルタイムに経路を再計算
    \item \textbf{旅客・貨物の優先度設定}:緊急性の高い移動を優先する仕組み
    \item \textbf{スペース管理アルゴリズム}:限られた車両スペースを旅客と貨物に最適配分
    \item \textbf{予約システム統合}:旅客予約と貨物配送依頼を統合的に管理
    \item \textbf{安全性確保}:旅客と貨物の物理的分離、貨物固定方法の最適化
\end{enumerate}

\subsubsection{社会的意義}

人貨混載システムの実現は、以下の社会的意義を持つ:

\begin{itemize}
    \item \textbf{過疎地域の移動手段確保}:公共交通空白地帯における高齢者等の移動支援
    \item \textbf{物流効率化}:小口貨物と未利用資源の効率的な収集運搬
    \item \textbf{CO$_2$排出削減}:旅客と貨物の統合輸送による走行距離削減
    \item \textbf{地域雇用創出}:運転手、予約管理者などの雇用機会提供
    \item \textbf{循環型社会形成}:未利用資源の効率的収集による地域資源循環促進
\end{itemize}

本研究で構築したシステムを基盤として、これらの拡張を段階的に実施することで、持続可能な地域社会の実現に貢献できると考えられる。

\vspace{1em}

\noindent
以上、本研究の成果と今後の展望について述べた。開発したシステムは、未利用資源運搬の効率化に寄与するとともに、人貨混載という新たな展開への基盤を提供するものである。


% 参考文献
\subsection*{公的機関}

\begin{enumerate}
    \item 国土交通省「自動車燃費一覧」、2025年版
    \item 国土交通省「自動車諸元表」、2025年版
    \item 厚生労働省「賃金構造基本統計調査」、令和4年
    \item 全日本トラック協会「経営分析報告書」、2024年度版
    \item 全日本トラック協会「トラック運送事業の賃金・労働時間等の実態」、2024年度版
\end{enumerate}

\subsection*{業界団体・メーカー}

\begin{enumerate}
    \setcounter{enumi}{5}
    \item いすゞ自動車株式会社、車両カタログ、2025年版
    \item 日野自動車株式会社、車両カタログ、2025年版
    \item 三菱ふそうトラック・バス株式会社、車両カタログ、2025年版
    \item UDトラックス株式会社、車両カタログ、2025年版
    \item 社団法人プラスチック処理促進協会「燃料消費原単位」、2024年版
\end{enumerate}

\subsection*{中古車販売・情報サイト}

\begin{enumerate}
    \setcounter{enumi}{10}
    \item トラック王国、\url{https://www.55truck.com/}、2025年10月アクセス
    \item トラック流通センター、\url{https://www.kaitoriou.net/}、2025年10月アクセス
    \item TRUCK BIZ、\url{https://www.truck-five.com/tfbiz/}、2025年10月アクセス
    \item トラック市、\url{https://www.truck-ichi.co.jp/}、2025年10月アクセス
    \item バディトラック、\url{https://buddytruck.jp/}、2025年10月アクセス
\end{enumerate}

\subsection*{その他}

\begin{enumerate}
    \setcounter{enumi}{15}
    \item 求人ボックス「トラック運転の年収・時給」、2025年10月アクセス
    \item 運転ドットコム「トラック運転手給与情報」、2025年10月アクセス
    \item トラック運送事業者へのヒアリング調査(2024年9月-2025年10月実施)
\end{enumerate}

\subsection*{ソフトウェア・ライブラリ}

\begin{enumerate}
    \setcounter{enumi}{18}
    \item Streamlit Team, Streamlit: The fastest way to build data apps, \url{https://streamlit.io/}, v1.28.0
    \item NetworkX Developers, NetworkX: Network Analysis in Python, \url{https://networkx.org/}, v3.1
    \item Folium Contributors, Folium: Python Data, Leaflet.js Maps, \url{https://python-visualization.github.io/folium/}, v0.14.0
    \item Google OR-Tools Team, Google OR-Tools, \url{https://developers.google.com/optimization}, v9.7
    \item McKinney, W., pandas: powerful Python data analysis toolkit, \url{https://pandas.pydata.org/}, v2.1.0
\end{enumerate}

\subsection*{学術文献}

\begin{enumerate}
    \setcounter{enumi}{23}
    \item Dijkstra, E. W. (1959). A note on two problems in connexion with graphs. \textit{Numerische Mathematik}, 1(1), 269-271.
    \item Hart, P. E., Nilsson, N. J., \& Raphael, B. (1968). A formal basis for the heuristic determination of minimum cost paths. \textit{IEEE Transactions on Systems Science and Cybernetics}, 4(2), 100-107.
    \item Dantzig, G. B., \& Ramser, J. H. (1959). The truck dispatching problem. \textit{Management Science}, 6(1), 80-91.
    \item Toth, P., \& Vigo, D. (Eds.). (2014). \textit{Vehicle Routing: Problems, Methods, and Applications} (2nd ed.). SIAM.
    \item Croes, G. A. (1958). A method for solving traveling-salesman problems. \textit{Operations Research}, 6(6), 791-812.
\end{enumerate}


% 付録
\appendix
本付録では、システムで使用される計算式の詳細と、使用したソフトウェア・ライブラリについて説明する。

\subsection{最短経路アルゴリズムの詳細}

\subsubsection{Dijkstra法}

Dijkstra法は、単一始点からすべてのノードへの最短経路を求めるアルゴリズムである。計算量は$O((V + E) \log V)$($V$はノード数、$E$はエッジ数)である。

\paragraph{アルゴリズムの概要}
\begin{enumerate}
    \item すべてのノードの距離を無限大に初期化し、始点のみを0とする
    \item 未訪問ノードの中から、最小距離のノードを選択する
    \item 選択したノードの隣接ノードについて、距離を更新する(緩和処理)
    \item すべてのノードを訪問するまで2-3を繰り返す
\end{enumerate}

\subsubsection{A*アルゴリズム}

A*アルゴリズムは、ヒューリスティック関数を用いてDijkstra法を拡張したアルゴリズムである。

評価関数:
\begin{equation}
f(n) = g(n) + h(n)
\end{equation}

ここで、$g(n)$は始点からノード$n$までの実際の距離、$h(n)$はノード$n$から目標ノードまでの推定距離(ヒューリスティック値)である。

本システムでは、ヒューリスティック関数としてユークリッド距離を使用する:
\begin{equation}
h(n) = \sqrt{(x_n - x_{\text{goal}})^2 + (y_n - y_{\text{goal}})^2}
\end{equation}

\subsection{コスト計算の詳細式}

\subsubsection{燃料費}

\begin{equation}
C_{\text{fuel}} = \frac{P_{\text{fuel}}}{F} \times D
\end{equation}

ここで、$P_{\text{fuel}}$は燃料単価(円/L)、$F$は燃費(km/L)、$D$は走行距離(km)である。

\subsubsection{運転手人件費}

\begin{equation}
C_{\text{labor}} = \frac{W_{\text{hour}}}{V_{\text{avg}}} \times D
\end{equation}

ここで、$W_{\text{hour}}$は時給(円/時)、$V_{\text{avg}}$は平均速度(km/時)である。

\subsubsection{作業時間人件費}

\begin{equation}
C_{\text{work}} = \frac{T_{\text{work}} \times W_{\text{hour}}}{D_{\text{avg}}} \times D
\end{equation}

ここで、$T_{\text{work}}$は1回あたりの作業時間(時)、$D_{\text{avg}}$は平均運搬距離(km)である。

\subsubsection{固定費の距離単価換算}

年間固定費を距離単価に換算する:
\begin{equation}
C_{\text{fixed/km}} = \frac{C_{\text{fixed/year}}}{D_{\text{annual}}}
\end{equation}

ここで、$C_{\text{fixed/year}}$は年間固定費(円/年)、$D_{\text{annual}}$は年間走行距離(km/年)である。

\subsubsection{総コストの計算}

総コストは以下の式で計算される:
\begin{equation}
C_{\text{total}} = D \times \left( \sum_{i=1}^{12} C_{\text{var},i} + \frac{\sum_{j=1}^{13} C_{\text{fixed},j}}{D_{\text{annual}}} \right)
\end{equation}

ここで、$C_{\text{var},i}$は変動費の第$i$項目(12項目)、$C_{\text{fixed},j}$は固定費の第$j$項目(13項目)である。

\subsection{CO$_2$排出量の計算}

\subsubsection{ガソリン車・ディーゼル車}

\begin{equation}
\text{CO}_2\text{排出量}[\text{kg}] = \frac{D}{F} \times \alpha
\end{equation}

ここで、$D$は走行距離(km)、$F$は燃費(km/L)、$\alpha$は排出係数(kg-CO$_2$/L)である。

排出係数は以下の値を使用する(環境省データ):
\begin{itemize}
    \item 軽油:$\alpha = 2.58$ kg-CO$_2$/L
    \item ガソリン:$\alpha = 2.32$ kg-CO$_2$/L
\end{itemize}

\subsection{使用ソフトウェアとライブラリ}

本システムの開発に使用した主要なソフトウェアとライブラリを以下に示す。

\begin{table}[H]
\centering
\caption{使用ソフトウェアとライブラリのバージョン情報}
\label{tab:software}
\begin{tabular}{llp{6cm}}
\toprule
\textbf{ライブラリ} & \textbf{バージョン} & \textbf{用途} \\
\midrule
Python & 3.8+ & システム全体の実装言語 \\
Streamlit & 1.28.0 & Webアプリケーションフレームワーク \\
NetworkX & 3.1 & グラフ理論と最短経路計算 \\
Folium & 0.14.0 & インタラクティブ地図の表示 \\
Pandas & 2.1.0 & データ処理と表形式データ操作 \\
NumPy & 1.24.3 & 数値計算とベクトル演算 \\
SciPy & 1.11.1 & 空間インデックス(KD木) \\
OR-Tools & 9.7 & 制約付き最適化問題の解決 \\
\bottomrule
\end{tabular}
\end{table}

\subsection{システム要件}

\subsubsection{ハードウェア要件}

\begin{table}[H]
\centering
\caption{推奨ハードウェア仕様}
\label{tab:hardware}
\begin{tabular}{lp{8cm}}
\toprule
\textbf{項目} & \textbf{推奨仕様} \\
\midrule
CPU & 2コア以上(4コア推奨) \\
メモリ & 4GB以上(8GB推奨) \\
ストレージ & 500MB以上の空き容量 \\
ディスプレイ & 1920×1080以上の解像度 \\
ネットワーク & インターネット接続(地図表示用) \\
\bottomrule
\end{tabular}
\end{table}

\subsubsection{ソフトウェア要件}

\begin{itemize}
    \item \textbf{OS}:Windows 10/11、macOS 10.14以降、Linux(Ubuntu 20.04以降推奨)
    \item \textbf{Python}:3.8以上(3.10推奨)
    \item \textbf{ブラウザ}:Google Chrome(最新版)、Firefox(最新版)、Edge(最新版)
\end{itemize}

\subsection{データファイル構成}

システムは以下のディレクトリ構造を持つ:

\begin{verbatim}
ResouceCollection_05/
├── run_app.bat                      # システム起動用バッチファイル
├── app.py                           # メインアプリケーション
├── requirements.txt                 # 依存ライブラリリスト
├── data/
│   ├── processed/
│   │   ├── compatibility.json       # 適合性マトリックス
│   │   ├── vehicles.json            # 車両諸元データ
│   │   └── resources.json           # 資源特性データ
│   └── networks/
│       └── default_network.json     # デフォルト道路ネットワーク
├── src/
│   ├── optimization/                # 最適化モジュール
│   ├── visualization/               # 可視化モジュール
│   └── cost/                        # コスト計算モジュール
└── docs/
    ├── report_02/                   # 本報告書
    └── claudedocs/                  # ユーザーガイド
\end{verbatim}

\subsection{計算パフォーマンス}

本システムの計算パフォーマンスを表\ref{tab:performance}に示す。

\begin{table}[H]
\centering
\caption{計算パフォーマンス(標準的なPC環境での測定値)}
\label{tab:performance}
\begin{tabular}{lrrr}
\toprule
\textbf{回収地点数} & \textbf{距離計算} & \textbf{最適化} & \textbf{合計} \\
\midrule
5地点 & 2.1秒 & 1.8秒 & 4.4秒 \\
10地点 & 3.8秒 & 4.2秒 & 8.8秒 \\
20地点 & 8.5秒 & 12.3秒 & 22.3秒 \\
\bottomrule
\end{tabular}
\end{table}

\textbf{注意}:回収地点数が10カ所を超えると計算時間が急激に増大するため、実用上は10カ所程度を上限とすることを推奨する。


\end{document}
