% 第7章 結果
\section{実証実験の結果}

\subsection{実験概要}

\subsubsection{実験目的}

本システムの有効性を検証するため、実際の資源回収業務を想定した実証実験を実施した。
実験の目的は以下の通りである:

\begin{itemize}
    \item システムによる最適化効果の定量評価
    \item 手動計画との比較
    \item 処理時間の測定
    \item ユーザビリティの評価
\end{itemize}

\subsubsection{実験条件}

実験は以下の条件で実施した:

\begin{table}[H]
\centering
\caption{実験条件}
\label{tab:experiment_conditions}
\begin{tabular}{lp{8cm}}
\toprule
\textbf{項目} & \textbf{内容} \\
\midrule
対象エリア & 市内全域(道路ネットワーク:約5,000ノード) \\
回収地点数 & ケース1:5地点、ケース2:10地点、ケース3:20地点 \\
資源種別 & 紙、プラスチック、ガラス \\
車種 & 小型EV(500kg)、大型トラック(1,500kg) \\
実施回数 & 各ケース10回 \\
\bottomrule
\end{tabular}
\end{table}

\subsection{最適化結果}

\subsubsection{ケース1:5地点の回収}

5地点での回収における結果を表\ref{tab:case1_results}に示す。

\begin{table}[H]
\centering
\caption{ケース1の結果(5地点)}
\label{tab:case1_results}
\begin{tabular}{lrr}
\toprule
\textbf{項目} & \textbf{手動計画} & \textbf{システム} \\
\midrule
総走行距離 [km] & 18.5 & 15.2 \\
総コスト [円] & 2,220 & 1,824 \\
計画作成時間 [分] & 15 & 0.5 \\
\midrule
距離削減率 [\%] & - & 17.8 \\
コスト削減率 [\%] & - & 17.8 \\
\bottomrule
\end{tabular}
\end{table}

\subsubsection{ケース2:10地点の回収}

10地点での回収における結果を表\ref{tab:case2_results}に示す。

\begin{table}[H]
\centering
\caption{ケース2の結果(10地点)}
\label{tab:case2_results}
\begin{tabular}{lrr}
\toprule
\textbf{項目} & \textbf{手動計画} & \textbf{システム} \\
\midrule
総走行距離 [km] & 32.8 & 28.1 \\
総コスト [円] & 3,936 & 3,372 \\
計画作成時間 [分] & 30 & 1.0 \\
\midrule
距離削減率 [\%] & - & 14.3 \\
コスト削減率 [\%] & - & 14.3 \\
\bottomrule
\end{tabular}
\end{table}

\subsubsection{ケース3:20地点の回収}

20地点での回収における結果を表\ref{tab:case3_results}に示す。

\begin{table}[H]
\centering
\caption{ケース3の結果(20地点)}
\label{tab:case3_results}
\begin{tabular}{lrr}
\toprule
\textbf{項目} & \textbf{手動計画} & \textbf{システム} \\
\midrule
総走行距離 [km] & 58.3 & 49.7 \\
総コスト [円] & 6,996 & 5,964 \\
計画作成時間 [分] & 60 & 3.0 \\
\midrule
距離削減率 [\%] & - & 14.8 \\
コスト削減率 [\%] & - & 14.8 \\
\bottomrule
\end{tabular}
\end{table}

\subsection{処理時間の測定}

システムの処理時間を測定した結果を表\ref{tab:processing_time}に示す。

\begin{table}[H]
\centering
\caption{処理時間の測定結果}
\label{tab:processing_time}
\begin{tabular}{lrrrr}
\toprule
\textbf{地点数} & \textbf{距離計算} & \textbf{最適化} & \textbf{結果生成} & \textbf{合計} \\
\midrule
5地点 & 2.1秒 & 1.8秒 & 0.5秒 & 4.4秒 \\
10地点 & 3.8秒 & 4.2秒 & 0.8秒 & 8.8秒 \\
20地点 & 8.5秒 & 12.3秒 & 1.5秒 & 22.3秒 \\
\bottomrule
\end{tabular}
\end{table}

\subsection{複数車両での最適化}

異なる資源種別を含むケースでの複数車両最適化の結果を表\ref{tab:multi_vehicle}に示す。

\begin{table}[H]
\centering
\caption{複数車両最適化の結果}
\label{tab:multi_vehicle}
\begin{tabular}{lrr}
\toprule
\textbf{項目} & \textbf{値} \\
\midrule
回収地点数 & 15地点 \\
資源種別 & 紙(8地点)、プラスチック(7地点) \\
使用車両 & 小型EV×1、大型×1 \\
総走行距離 [km] & 42.3 \\
総コスト [円] & 5,076 \\
処理時間 [秒] & 15.2 \\
\bottomrule
\end{tabular}
\end{table}

\subsection{ユーザビリティ評価}

5名のユーザーによる評価を実施した。評価結果を表\ref{tab:usability}に示す。

\begin{table}[H]
\centering
\caption{ユーザビリティ評価(5段階評価)}
\label{tab:usability}
\begin{tabular}{lc}
\toprule
\textbf{評価項目} & \textbf{平均点} \\
\midrule
操作の分かりやすさ & 4.2 \\
地図インターフェースの使いやすさ & 4.6 \\
結果表示の見やすさ & 4.4 \\
処理速度の満足度 & 4.8 \\
総合満足度 & 4.5 \\
\bottomrule
\end{tabular}
\end{table}

\subsection{環境負荷低減効果}

EV車両使用時のCO2削減効果を試算した結果を表\ref{tab:co2_reduction}に示す。

\begin{table}[H]
\centering
\caption{CO2削減効果の試算}
\label{tab:co2_reduction}
\begin{tabular}{lrrr}
\toprule
\textbf{項目} & \textbf{ガソリン車} & \textbf{EV車} & \textbf{削減量} \\
\midrule
走行距離 [km] & 28.1 & 28.1 & - \\
エネルギー消費 & 2.8L & 5.6kWh & - \\
CO2排出量 [kg] & 6.4 & 2.8 & 3.6 \\
削減率 [\%] & - & - & 56.3 \\
\bottomrule
\end{tabular}
\end{table}

※ガソリン車:2.3kg-CO2/L、EV車:0.5kg-CO2/kWhで試算
