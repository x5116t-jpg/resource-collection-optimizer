% 第5章 機能
\section{機能}

\subsection{地点選択機能}

\subsubsection{地図クリックによる選択}

ユーザーは地図上をクリックすることで、以下の地点を直感的に選択できる:

\begin{itemize}
    \item \textbf{車庫}:車両の出発・帰着地点
    \item \textbf{回収地点}:資源を回収する地点(複数選択可)
    \item \textbf{集積場所}:回収した資源を集める終点
\end{itemize}

システムは、クリック位置から最も近いノード(道路上の地点)を自動的に特定し、選択地点として設定する。

\subsubsection{空間インデックスによる高速検索}

地図クリック時の最寄りノード検索には、空間インデックス(KD木)を使用している。
これにより、数千〜数万のノードから最寄りを高速に検索できる。

\textbf{検索時間:} $O(\log N)$ (N: ノード数)

\subsection{車種割当機能}

\subsubsection{自動車種選択}

システムは、以下のロジックで最適な車種を自動選択する:

\begin{enumerate}
    \item 各資源種別に対して対応可能な車種を抽出
    \item 各車種のコスト評価スコアを計算
    \item 最もコスト効率の良い車種を選択
\end{enumerate}

\textbf{コスト評価式:}
\begin{equation}
\text{スコア} = \text{距離単価} + \text{固定費単価}
\end{equation}

\subsubsection{適合性チェック}

マスタデータに基づき、車種と資源の適合性を自動的に判定する。
適合性マトリクスの例を表\ref{tab:compatibility_example}に示す。

\begin{table}[H]
\centering
\caption{車種と資源の適合性マトリクス(例)}
\label{tab:compatibility_example}
\begin{tabular}{lccc}
\toprule
& \textbf{紙} & \textbf{プラスチック} & \textbf{ガラス} \\
\midrule
小型EV & ○ & ○ & × \\
大型トラック & ○ & ○ & ○ \\
専用車 & × & × & ○ \\
\bottomrule
\end{tabular}
\end{table}

\subsection{ルート最適化機能}

\subsubsection{最適化手法}

VRP(Vehicle Routing Problem)の変種を解くため、
貪欲法ベースのヒューリスティックアルゴリズムを採用した。

\textbf{アルゴリズムの特徴:}
\begin{itemize}
    \item 初期解を近傍優先で生成
    \item 2-opt法による局所最適化
    \item 容量制約と資源適合性を考慮
\end{itemize}

\subsubsection{複数車両への対応}

資源種別が複数ある場合、以下の方法で車両を割り当てる:

\begin{enumerate}
    \item 全資源を1台で運べる場合:その車両で最適化
    \item 複数台が必要な場合:資源ごとに車両を割り当て、個別に最適化
\end{enumerate}

\subsection{コスト計算機能}

\subsubsection{詳細内訳の表示}

システムは、以下の項目別にコストを表示する:

\textbf{固定費項目:}
\begin{itemize}
    \item 人件費
    \item 車両償却費
    \item 保険料
    \item 車検・点検費用
\end{itemize}

\textbf{変動費項目:}
\begin{itemize}
    \item 燃料費(または電気代)
    \item タイヤ代
    \item 修繕費
    \item 消耗品費
\end{itemize}

\subsubsection{計算式の可視化}

ユーザーの理解を助けるため、コスト計算式をLaTeX形式で表示する:

\begin{equation}
\text{総コスト} = \text{変動費} + \text{固定費}
\end{equation}

\subsection{エネルギー計算機能}

\subsubsection{消費量の算出}

EV(電気自動車)の場合、以下の式でエネルギー消費量を計算する:

\begin{equation}
\text{消費電力量}[\text{kWh}] = \text{原単位}[\text{kWh/km}] \times \text{走行距離}[\text{km}]
\end{equation}

\subsubsection{CO2削減効果の試算}

ガソリン車からEVへの置き換え時のCO2削減効果を試算する:

\begin{equation}
\text{CO}_2\text{削減量}[\text{kg}] = \text{ガソリン車排出量} - \text{EV車排出量}
\end{equation}

\subsection{可視化機能}

\subsubsection{インタラクティブ地図}

Foliumライブラリを使用し、以下の機能を持つ地図を表示する:

\begin{itemize}
    \item ズーム・パン操作
    \item マーカークリックによる地点情報表示
    \item 訪問順序の番号付きマーカー
    \item 経路の青線表示
\end{itemize}

\subsubsection{マーカーの色分け}

地点の種類に応じてマーカーを色分けする:

\begin{table}[H]
\centering
\caption{マーカーの色分け}
\label{tab:marker_colors}
\begin{tabular}{ll}
\toprule
\textbf{地点種類} & \textbf{色} \\
\midrule
車庫(出発地) & 緑 \\
回収地点 & 青 \\
集積場所(終点) & 赤 \\
最新選択地点 & 黄色枠 \\
\bottomrule
\end{tabular}
\end{table}
