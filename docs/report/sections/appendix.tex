% 付録
\appendix

\section{使用ソフトウェアとライブラリ}
\label{appendix:software}

\subsection{主要ライブラリのバージョン情報}

本システムの開発に使用した主要なPythonライブラリとそのバージョンを表\ref{tab:software_versions}に示す。

\begin{table}[H]
\centering
\caption{使用ライブラリのバージョン情報}
\label{tab:software_versions}
\begin{tabular}{llp{6cm}}
\toprule
\textbf{ライブラリ} & \textbf{バージョン} & \textbf{用途} \\
\midrule
Python & 3.8+ & システム全体の実装言語 \\
Streamlit & 1.28.0 & Webアプリケーションフレームワーク \\
NetworkX & 3.1 & グラフ理論と最短経路計算 \\
Folium & 0.14.0 & インタラクティブ地図の表示 \\
Pandas & 2.0.3 & データ処理と表形式データ操作 \\
NumPy & 1.24.3 & 数値計算とベクトル演算 \\
SciPy & 1.11.1 & 空間インデックス(KD木) \\
\bottomrule
\end{tabular}
\end{table}

\subsection{システム要件}

\subsubsection{ハードウェア要件}

\begin{table}[H]
\centering
\caption{推奨ハードウェア仕様}
\label{tab:hardware_requirements}
\begin{tabular}{lp{8cm}}
\toprule
\textbf{項目} & \textbf{推奨仕様} \\
\midrule
CPU & 2コア以上(4コア推奨) \\
メモリ & 4GB以上(8GB推奨) \\
ストレージ & 500MB以上の空き容量 \\
ディスプレイ & 1920×1080以上の解像度 \\
ネットワーク & インターネット接続(地図表示用) \\
\bottomrule
\end{tabular}
\end{table}

\subsubsection{ソフトウェア要件}

\begin{itemize}
    \item \textbf{OS}:Windows 10/11、macOS 10.14以降、Linux(Ubuntu 20.04以降推奨)
    \item \textbf{Python}:3.8以上(3.10推奨)
    \item \textbf{ブラウザ}:Google Chrome(最新版)、Firefox(最新版)、Edge(最新版)
\end{itemize}

\section{詳細データと補足資料}
\label{appendix:data}

\subsection{実験データの詳細}

\subsubsection{ケース1:5地点の詳細データ}

\begin{table}[H]
\centering
\caption{ケース1の詳細実験データ(10回試行)}
\label{tab:case1_detailed}
\begin{tabular}{crrr}
\toprule
\textbf{試行} & \textbf{手動[km]} & \textbf{システム[km]} & \textbf{削減率[\%]} \\
\midrule
1 & 18.2 & 15.1 & 17.0 \\
2 & 18.7 & 15.4 & 17.6 \\
3 & 18.3 & 15.0 & 18.0 \\
4 & 18.9 & 15.5 & 18.0 \\
5 & 18.1 & 14.9 & 17.7 \\
6 & 18.6 & 15.3 & 17.7 \\
7 & 18.4 & 15.2 & 17.4 \\
8 & 18.8 & 15.4 & 18.1 \\
9 & 18.5 & 15.1 & 18.4 \\
10 & 18.5 & 15.3 & 17.3 \\
\midrule
平均 & 18.5 & 15.2 & 17.8 \\
標準偏差 & 0.26 & 0.20 & 0.35 \\
\bottomrule
\end{tabular}
\end{table}

\subsubsection{処理時間の詳細分析}

\begin{table}[H]
\centering
\caption{処理時間の詳細分析(平均値、標準偏差)}
\label{tab:processing_time_detailed}
\begin{tabular}{lrrrr}
\toprule
\textbf{地点数} & \textbf{距離計算} & \textbf{最適化} & \textbf{結果生成} & \textbf{合計} \\
\midrule
5地点 & 2.1±0.3秒 & 1.8±0.2秒 & 0.5±0.1秒 & 4.4±0.4秒 \\
10地点 & 3.8±0.5秒 & 4.2±0.6秒 & 0.8±0.1秒 & 8.8±0.9秒 \\
20地点 & 8.5±1.2秒 & 12.3±1.8秒 & 1.5±0.2秒 & 22.3±2.5秒 \\
\bottomrule
\end{tabular}
\end{table}

\subsection{マスタデータのサンプル}

\subsubsection{車種マスタの例}

\begin{lstlisting}[caption=vehicles.jsonのサンプル]
{
  "vehicles": [
    {
      "name": "小型EV",
      "capacity_kg": 500,
      "fixed_costs": {
        "labor_per_year": 3000000,
        "depreciation_per_year": 500000,
        "insurance_per_year": 100000,
        "inspection_per_year": 50000
      },
      "variable_costs": {
        "energy_per_km": 20,
        "tire_per_km": 5,
        "repair_per_km": 3,
        "consumables_per_km": 2
      },
      "energy_consumption_kwh_per_km": 0.2,
      "annual_mileage": 25000
    },
    {
      "name": "大型トラック",
      "capacity_kg": 1500,
      "fixed_costs": {
        "labor_per_year": 4000000,
        "depreciation_per_year": 800000,
        "insurance_per_year": 150000,
        "inspection_per_year": 80000
      },
      "variable_costs": {
        "fuel_per_km": 50,
        "tire_per_km": 10,
        "repair_per_km": 8,
        "consumables_per_km": 4
      },
      "fuel_consumption_l_per_km": 0.1,
      "annual_mileage": 35000
    }
  ]
}
\end{lstlisting}

\subsubsection{資源マスタの例}

\begin{lstlisting}[caption=resources.jsonのサンプル]
{
  "resources": [
    {
      "name": "紙",
      "density_kg_per_m3": 100,
      "recyclable": true,
      "hazardous": false
    },
    {
      "name": "プラスチック",
      "density_kg_per_m3": 50,
      "recyclable": true,
      "hazardous": false
    },
    {
      "name": "ガラス",
      "density_kg_per_m3": 400,
      "recyclable": true,
      "hazardous": false
    }
  ]
}
\end{lstlisting}

\section{計算式の詳細}
\label{appendix:formulas}

\subsection{コスト計算の詳細}

\subsubsection{固定費の計算式}

固定費の各項目の計算方法を以下に示す:

\begin{equation}
\text{固定費単価}_i = \frac{\text{年間固定費}_i}{\text{年間走行距離}}
\end{equation}

各項目:
\begin{itemize}
    \item \textbf{人件費}:$\frac{3,000,000 \text{円/年}}{25,000 \text{km/年}} = 120 \text{円/km}$
    \item \textbf{車両償却費}:$\frac{500,000 \text{円/年}}{25,000 \text{km/年}} = 20 \text{円/km}$
    \item \textbf{保険料}:$\frac{100,000 \text{円/年}}{25,000 \text{km/年}} = 4 \text{円/km}$
    \item \textbf{車検・点検費用}:$\frac{50,000 \text{円/年}}{25,000 \text{km/年}} = 2 \text{円/km}$
\end{itemize}

\subsubsection{変動費の計算式}

変動費の各項目:
\begin{itemize}
    \item \textbf{電気代}(小型EV):$20 \text{円/km}$
    \item \textbf{燃料費}(大型トラック):$50 \text{円/km}$
    \item \textbf{タイヤ代}:$5 \text{円/km}$(小型)、$10 \text{円/km}$(大型)
    \item \textbf{修繕費}:$3 \text{円/km}$(小型)、$8 \text{円/km}$(大型)
    \item \textbf{消耗品費}:$2 \text{円/km}$(小型)、$4 \text{円/km}$(大型)
\end{itemize}

\subsection{CO2排出量の計算}

\subsubsection{ガソリン車のCO2排出量}

\begin{equation}
\text{CO}_2\text{排出量}[\text{kg}] = \text{燃料消費量}[\text{L}] \times \text{排出係数}[\text{kg-CO}_2\text{/L}]
\end{equation}

排出係数:$2.3 \text{ kg-CO}_2\text{/L}$(環境省データに基づく)

\subsubsection{EV車のCO2排出量}

\begin{equation}
\text{CO}_2\text{排出量}[\text{kg}] = \text{電力消費量}[\text{kWh}] \times \text{排出係数}[\text{kg-CO}_2\text{/kWh}]
\end{equation}

排出係数:$0.5 \text{ kg-CO}_2\text{/kWh}$(全国平均値)

\section{アルゴリズムの疑似コード}
\label{appendix:algorithms}

\subsection{完全な最適化アルゴリズム}

\begin{lstlisting}[language=Python, caption=完全な最適化アルゴリズム]
def optimize_route(depot, pickup_points, sink, vehicle_types):
    """
    完全なルート最適化アルゴリズム

    Args:
        depot: 車庫地点
        pickup_points: 回収地点のリスト
        sink: 集積場所
        vehicle_types: 利用可能な車種のリスト

    Returns:
        最適化されたルート情報
    """
    # ステップ1: 距離行列の計算
    distance_matrix = calculate_distance_matrix(
        depot, pickup_points, sink
    )

    # ステップ2: 資源種別ごとにグループ化
    resource_groups = group_by_resource_type(pickup_points)

    # ステップ3: 各グループに対して車種を選択
    vehicle_assignments = {}
    for resource_type, points in resource_groups.items():
        compatible_vehicles = find_compatible_vehicles(
            resource_type, vehicle_types
        )
        best_vehicle = select_best_vehicle(
            compatible_vehicles, points, distance_matrix
        )
        vehicle_assignments[resource_type] = best_vehicle

    # ステップ4: 各車種ごとにルート最適化
    routes = {}
    for resource_type, vehicle in vehicle_assignments.items():
        points = resource_groups[resource_type]

        # 初期解の生成(貪欲法)
        initial_route = generate_initial_route(
            depot, points, sink, vehicle, distance_matrix
        )

        # 局所最適化(2-opt法)
        optimized_route = improve_by_2opt(
            initial_route, distance_matrix
        )

        routes[resource_type] = optimized_route

    # ステップ5: コストとエネルギーの計算
    total_cost = 0
    total_energy = 0
    for resource_type, route in routes.items():
        vehicle = vehicle_assignments[resource_type]
        distance = calculate_route_distance(route, distance_matrix)
        cost = calculate_cost(vehicle, distance)
        energy = calculate_energy(vehicle, distance)

        total_cost += cost
        total_energy += energy

    return {
        'routes': routes,
        'vehicle_assignments': vehicle_assignments,
        'total_cost': total_cost,
        'total_energy': total_energy
    }


def generate_initial_route(depot, points, sink, vehicle,
                          distance_matrix):
    """貪欲法による初期解生成"""
    route = [depot]
    remaining = set(points)
    current_load = 0

    while remaining:
        current = route[-1]

        # 最も近い未訪問地点を選択
        nearest = None
        min_distance = float('inf')

        for point in remaining:
            dist = distance_matrix[current.id][point.id]
            if dist < min_distance:
                # 容量制約チェック
                if current_load + point.demand <= vehicle.capacity:
                    min_distance = dist
                    nearest = point

        if nearest is None:
            break  # これ以上積載できない

        route.append(nearest)
        current_load += nearest.demand
        remaining.remove(nearest)

    route.append(sink)
    return route


def improve_by_2opt(route, distance_matrix):
    """2-opt法による局所最適化"""
    improved = True
    best_route = route[:]

    while improved:
        improved = False
        best_distance = calculate_route_distance(
            best_route, distance_matrix
        )

        for i in range(1, len(best_route) - 2):
            for j in range(i + 1, len(best_route)):
                # エッジ (i-1, i) と (j-1, j) を交換
                new_route = two_opt_swap(best_route, i, j)
                new_distance = calculate_route_distance(
                    new_route, distance_matrix
                )

                if new_distance < best_distance:
                    best_route = new_route
                    best_distance = new_distance
                    improved = True
                    break

            if improved:
                break

    return best_route


def two_opt_swap(route, i, j):
    """2-opt交換の実行"""
    new_route = route[:i]
    new_route.extend(reversed(route[i:j]))
    new_route.extend(route[j:])
    return new_route
\end{lstlisting}

\section{用語集}
\label{appendix:glossary}

\begin{table}[H]
\centering
\caption{本報告書で使用する用語の定義}
\label{tab:glossary}
\begin{tabular}{p{4cm}p{10cm}}
\toprule
\textbf{用語} & \textbf{定義} \\
\midrule
VRP & Vehicle Routing Problem(配送計画問題)。複数の訪問地点を効率的に巡回するルートを求める最適化問題。 \\
\midrule
ヒューリスティック & 厳密解を保証しないが、現実的な時間で良質な近似解を得る方法。 \\
\midrule
貪欲法 & 各ステップで局所的に最良の選択を行うアルゴリズム。 \\
\midrule
2-opt法 & ルート上の2つのエッジを交換して改善を試みる局所探索法。 \\
\midrule
Dijkstra法 & グラフ上の最短経路を求めるアルゴリズム。 \\
\midrule
KD木 & k次元空間における近傍検索を高速化するデータ構造。 \\
\midrule
NetworkX & Pythonのグラフ理論ライブラリ。 \\
\midrule
Streamlit & Pythonでデータアプリを構築するWebフレームワーク。 \\
\midrule
Folium & Leaflet.jsを用いたPython地図表示ライブラリ。 \\
\midrule
OSM & OpenStreetMap。オープンソースの地図データプロジェクト。 \\
\bottomrule
\end{tabular}
\end{table}

\section{システム設定ファイルの例}
\label{appendix:config}

\begin{lstlisting}[caption=config.jsonの例]
{
  "system": {
    "name": "資源回収ルート最適化システム",
    "version": "1.0.0",
    "default_map_center": [35.6812, 139.7671],
    "default_zoom": 12
  },
  "optimization": {
    "max_iterations": 1000,
    "time_limit_seconds": 300,
    "improvement_threshold": 0.001
  },
  "ui": {
    "language": "ja",
    "theme": "light",
    "max_points_display": 100
  },
  "performance": {
    "cache_enabled": true,
    "parallel_processing": true,
    "max_workers": 4
  }
}
\end{lstlisting}

\end{document}
