% 第8章 考察
\section{考察}

\subsection{最適化効果の分析}

\subsubsection{距離削減効果}

実証実験の結果、システムによる最適化で平均15\%の距離削減を達成した。
地点数別の削減率を分析すると、以下の傾向が見られた:

\begin{itemize}
    \item 5地点:17.8\%削減(最も高い削減率)
    \item 10地点:14.3\%削減
    \item 20地点:14.8\%削減
\end{itemize}

5地点での削減率が最も高い理由は、問題の複雑度が低く、最適解に近い解を見つけやすいためと考えられる。
一方、10地点以上では複雑度が増すため、ヒューリスティックアルゴリズムの限界が表れていると推測される。

\subsubsection{計画作成時間の短縮}

手動計画と比較して、計画作成時間は大幅に短縮された:

\begin{itemize}
    \item 5地点:15分 → 0.5分(30分の1)
    \item 10地点:30分 → 1.0分(30分の1)
    \item 20地点:60分 → 3.0分(20分の1)
\end{itemize}

この結果は、日常業務における計画作成の負荷を大幅に軽減できることを示している。
特に、計画の見直しや複数パターンの比較検討が容易になることは、業務改善に大きく貢献すると考えられる。

\subsection{アルゴリズムの評価}

\subsubsection{貪欲法の適用妥当性}

本システムで採用した貪欲法ベースのヒューリスティックは、以下の点で妥当であったと評価できる:

\begin{table}[H]
\centering
\caption{貪欲法の評価}
\label{tab:greedy_evaluation}
\begin{tabular}{lp{10cm}}
\toprule
\textbf{評価項目} & \textbf{結果} \\
\midrule
計算時間 & 20地点でも22.3秒と実用的な速度 \\
解の品質 & 手動計画より平均15\%改善 \\
安定性 & 10回の試行で安定した結果 \\
拡張性 & 50地点程度まで対応可能 \\
\bottomrule
\end{tabular}
\end{table}

\subsubsection{2-opt法による改善効果}

局所最適化として採用した2-opt法は、初期解から平均5-8\%の追加改善をもたらした。
この改善幅は、追加の計算コスト(約2-3秒)に対して十分な効果があると評価できる。

\subsection{システムの実用性}

\subsubsection{ユーザビリティ}

ユーザー評価(表\ref{tab:usability})において、全項目で4.0以上の高評価を得た。
特に以下の点が評価された:

\begin{itemize}
    \item \textbf{処理速度}(4.8点):計画作成が迅速で待ち時間が少ない
    \item \textbf{地図インターフェース}(4.6点):直感的な地点選択が可能
    \item \textbf{結果表示}(4.4点):コスト内訳が詳細でわかりやすい
\end{itemize}

一方、操作の分かりやすさが4.2点と相対的に低く、初回利用時のガイダンス強化が今後の課題として挙げられる。

\subsubsection{業務適用可能性}

実証実験の条件は実際の資源回収業務を想定したものであり、結果は実務適用可能性を示している。
ただし、以下の点に留意する必要がある:

\begin{itemize}
    \item 道路状況の変化(工事、渋滞)は考慮されていない
    \item 回収作業時間は距離計算に含まれていない
    \item 複数日のスケジューリングには非対応
\end{itemize}

\subsection{環境負荷低減効果}

\subsubsection{CO2削減効果の評価}

ガソリン車からEV車への置き換えにより、56.3\%のCO2削減効果が試算された。
この効果は、以下の2つの要素から構成される:

\begin{enumerate}
    \item \textbf{走行距離削減}(システム最適化):平均15\%の距離削減
    \item \textbf{燃料転換}(EV採用):ガソリン→電力で約50\%の排出削減
\end{enumerate}

年間1,000回の回収業務を想定すると、年間約3.6トンのCO2削減が見込まれる。

\subsubsection{長期的効果}

本システムの継続的な利用により、以下の長期的効果が期待できる:

\begin{itemize}
    \item 累積的なCO2削減効果(10年で約36トン)
    \item 燃料コスト削減による経済効果
    \item 環境配慮企業としてのブランド価値向上
\end{itemize}

\subsection{制限事項と課題}

\subsubsection{現状の制限事項}

本システムには以下の制限事項がある:

\begin{table}[H]
\centering
\caption{システムの制限事項}
\label{tab:limitations}
\begin{tabular}{lp{8cm}}
\toprule
\textbf{項目} & \textbf{制限内容} \\
\midrule
地点数 & 50地点以上では処理時間が長くなる \\
スケジュール & 1日のみ、複数日非対応 \\
時刻指定 & 到着時刻の指定不可 \\
動的要素 & 渋滞、工事等のリアルタイム情報非対応 \\
解の品質 & 厳密解でなく近似解 \\
\bottomrule
\end{tabular}
\end{table}

\subsubsection{今後の課題}

システムの更なる改善のため、以下の課題に取り組む必要がある:

\begin{enumerate}
    \item \textbf{アルゴリズムの高度化}
    \begin{itemize}
        \item メタヒューリスティック(遺伝的アルゴリズム、シミュレーテッドアニーリング)の導入
        \item 厳密解法との性能比較
    \end{itemize}

    \item \textbf{実務機能の拡張}
    \begin{itemize}
        \item 時刻指定への対応
        \item 複数日スケジューリング機能
        \item リアルタイム交通情報の統合
    \end{itemize}

    \item \textbf{ユーザビリティ向上}
    \begin{itemize}
        \item 初回利用者向けガイダンス
        \item モバイル対応
        \item 音声入力サポート
    \end{itemize}

    \item \textbf{データ分析機能}
    \begin{itemize}
        \item 過去の回収データの蓄積と分析
        \item 需要予測機能
        \item 最適な車両配置提案
    \end{itemize}
\end{enumerate}

\subsection{他分野への応用可能性}

本システムで採用したVRP最適化技術は、資源回収以外の分野にも応用可能である:

\begin{itemize}
    \item \textbf{配送業務}:宅配便、食品配送等の配送ルート最適化
    \item \textbf{訪問サービス}:訪問介護、営業訪問等のスケジュール最適化
    \item \textbf{公共サービス}:除雪作業、道路清掃等の効率化
    \item \textbf{点検業務}:設備点検、メーター検針等の巡回計画
\end{itemize}

これらの分野への展開により、社会全体の効率化と環境負荷低減に貢献できる可能性がある。
