% 第2章 システム概要
\section{システム概要}

\subsection{システムの目的}

資源回収ルート最適化システムは、道路ネットワーク上での資源回収業務において、
以下の最適化を実現することを目的とする:

\begin{itemize}
    \item \textbf{コスト最小化}:固定費・変動費を含む総コストの最小化
    \item \textbf{距離最小化}:総走行距離の最小化による効率化
    \item \textbf{環境負荷低減}:エネルギー消費量(CO2排出量)の削減
    \item \textbf{計画作成の効率化}:視覚的なインターフェースによる迅速な計画立案
\end{itemize}

\subsection{システムの特徴}

本システムは、以下の特徴を有する:

\subsubsection{直感的な操作性}
\begin{itemize}
    \item 地図上でのクリック操作による地点選択
    \item リアルタイムでの選択状況の確認
    \item 実行前チェックリストによる入力ミスの防止
\end{itemize}

\subsubsection{高度な最適化機能}
\begin{itemize}
    \item VRP(Vehicle Routing Problem)の変種を解くアルゴリズム
    \item 車両容量制約と資源適合性を考慮した最適化
    \item 複数車両の同時最適化
\end{itemize}

\subsubsection{詳細な分析機能}
\begin{itemize}
    \item 固定費・変動費の項目別詳細表示
    \item エネルギー消費量の計算
    \item CO2削減効果の可視化
\end{itemize}

\subsubsection{視覚的な結果表示}
\begin{itemize}
    \item インタラクティブ地図上でのルート表示
    \item 訪問順序の番号付きマーカー
    \item コスト計算式の数式表示
\end{itemize}

\subsection{システムの適用範囲}

本システムは、以下の業務に適用可能である:

\begin{table}[H]
\centering
\caption{システムの適用可能な業務}
\label{tab:applicable_tasks}
\begin{tabular}{ll}
\toprule
\textbf{業務分類} & \textbf{具体例} \\
\midrule
資源回収 & 紙、プラスチック、ガラス等の分別回収 \\
廃棄物収集 & 一般廃棄物、産業廃棄物の収集運搬 \\
配送計画 & リサイクル品の配送、資材の輸送 \\
巡回サービス & 点検業務、メンテナンス業務 \\
\bottomrule
\end{tabular}
\end{table}

\subsection{システムの制約事項}

本システムには、以下の制約事項が存在する:

\begin{itemize}
    \item \textbf{地点数の上限}:推奨20地点以内、最大50地点程度
    \item \textbf{単一日のルート}:複数日のスケジューリングには非対応
    \item \textbf{時間指定}:到着時刻の指定には非対応
    \item \textbf{最適性}:ヒューリスティックアルゴリズムのため厳密解ではない
    \item \textbf{道路制約}:一方通行や車両制限は道路ネットワークデータに依存
\end{itemize}

\subsection{期待される効果}

本システムの導入により、以下の効果が期待される:

\begin{table}[H]
\centering
\caption{期待される効果}
\label{tab:expected_effects}
\begin{tabular}{lp{8cm}}
\toprule
\textbf{項目} & \textbf{効果} \\
\midrule
走行距離 & 最適化により平均10〜20\%削減 \\
燃料費 & 走行距離削減に比例した燃料費の削減 \\
CO2排出量 & EV導入時の削減効果の定量化 \\
計画作成時間 & 手動計画と比較して約90\%短縮 \\
作業負荷 & 担当者の負担軽減と属人化の解消 \\
意思決定 & データに基づく客観的な判断が可能 \\
\bottomrule
\end{tabular}
\end{table}
