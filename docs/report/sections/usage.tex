% 第6章 使用方法
\section{使用方法}

\subsection{システムの起動}

\subsubsection{起動手順}

システムを起動するには、以下の手順を実行する:

\begin{enumerate}
    \item プロジェクトフォルダに移動
    \item 仮想環境をアクティブ化(必要な場合)
    \item Streamlitアプリを起動
\end{enumerate}

\textbf{コマンド例:}
\begin{lstlisting}[language=bash, caption=起動コマンド]
cd /path/to/project
.venv/Scripts/activate
streamlit run src/app.py
\end{lstlisting}

起動後、ブラウザが自動的に開き、システムのUIが表示される(通常は http://localhost:8501)。

\subsection{基本的な操作フロー}

システムの基本的な使用フローを図\ref{fig:usage_flow}に示す。

\begin{figure}[H]
\centering
\fbox{
\begin{minipage}{0.8\textwidth}
\begin{enumerate}
    \item 道路ネットワークファイルを選択
    \item 車庫を地図上で設定
    \item 回収地点を地図上で設定(複数)
    \begin{itemize}
        \item 資源種別を選択
        \item 回収量を入力
    \end{itemize}
    \item 集積場所を地図上で設定
    \item 実行前チェックを確認
    \item 最適化を実行
    \item 結果を確認
\end{enumerate}
\end{minipage}
}
\caption{基本的な操作フロー}
\label{fig:usage_flow}
\end{figure}

\subsection{詳細な操作手順}

\subsubsection{ステップ1:道路ネットワークの選択}

サイドバーのドロップダウンメニューから、使用する道路ネットワークファイルを選択する。
選択後、ノード数とエッジ数が表示される。

\subsubsection{ステップ2:車庫の設定}

\begin{enumerate}
    \item 「地図クリックモード」で「車庫」を選択
    \item 地図上の車庫にしたい地点をクリック
    \item 緑色のマーカーが表示されることを確認
\end{enumerate}

\subsubsection{ステップ3:回収地点の設定}

\begin{enumerate}
    \item 「地図クリックモード」で「回収地点」を選択
    \item 地図上の回収地点をクリック
    \item ダイアログが表示される
    \item 以下の情報を入力:
    \begin{itemize}
        \item 回収量(kg):0〜100,000の範囲で入力
        \item 資源種別:プルダウンから選択
    \end{itemize}
    \item 「追加」ボタンをクリック
    \item 必要な回収地点すべてについて繰り返す
\end{enumerate}

\subsubsection{ステップ4:集積場所の設定}

\begin{enumerate}
    \item 「地図クリックモード」で「集積場所」を選択
    \item 地図上の集積場所をクリック
    \item 赤色のマーカーが表示されることを確認
\end{enumerate}

\textbf{注意:} 車庫と集積場所は異なる地点を選択する必要がある。

\subsubsection{ステップ5:実行前チェック}

最適化実行前に、以下の項目がすべて満たされていることを確認する:

\begin{itemize}
    \item 車庫が設定されている
    \item 集積場所が設定されている
    \item 車庫と集積場所が異なる
    \item 回収地点が1箇所以上ある
    \item 全回収地点に資源種別が設定されている
    \item 車種が1種類以上設定されている
    \item 車種割当プランが作成されている
    \item 車種割当に警告がない
\end{itemize}

\subsubsection{ステップ6:最適化の実行}

\begin{enumerate}
    \item 「最適化を実行」ボタンをクリック
    \item プログレスバーで進捗を確認
    \item 完了まで待機(数秒〜数十秒)
\end{enumerate}

\subsection{結果の確認方法}

最適化完了後、以下の情報が表示される:

\subsubsection{サマリー情報}
\begin{itemize}
    \item 総距離 [km]
    \item 総コスト [円]
    \item エネルギー消費量 [kWh](EV車の場合)
    \item 採用車種
    \item ルート順(訪問順序)
\end{itemize}

\subsubsection{コスト内訳}
\begin{itemize}
    \item 固定費の合計
    \item 変動費の合計
    \item 総コスト
    \item 項目別の詳細内訳
\end{itemize}

\subsubsection{地図表示}
\begin{itemize}
    \item 訪問順序を示す番号付きマーカー
    \item 実際の道路に沿った経路(青線)
    \item マーカーの色分け(緑:出発、青:経由、赤:終点)
\end{itemize}

\subsection{注意事項}

\subsubsection{推奨される使用条件}

本システムを効果的に使用するため、以下の条件を推奨する:

\begin{table}[H]
\centering
\caption{推奨使用条件}
\label{tab:recommended_conditions}
\begin{tabular}{lp{8cm}}
\toprule
\textbf{項目} & \textbf{推奨値} \\
\midrule
回収地点数 & 10〜20地点以内 \\
処理時間 & 10地点:約10秒、20地点:約30秒 \\
ブラウザ & Google Chrome(最新版) \\
画面解像度 & 1920×1080以上 \\
\bottomrule
\end{tabular}
\end{table}

\subsubsection{制限事項}

以下の制限事項に注意すること:

\begin{itemize}
    \item 地点数が多い(50地点以上)と処理時間が長くなる
    \item 複数日のスケジュールには非対応
    \item 到着時刻の指定には非対応
    \item 厳密解ではなく、ヒューリスティックによる近似解
\end{itemize}
