% 要旨
\section*{要旨}
\addcontentsline{toc}{section}{要旨}

本報告書は、資源回収ルート最適化システムの開発について述べたものである。

本システムは、自治体や廃棄物処理業者における資源回収業務の効率化を目的として開発された。
道路ネットワーク上での最短ルート探索とコスト最適化を組み合わせることで、総走行距離とコストの最小化を実現する。

主な特徴として、以下の点が挙げられる:

\begin{itemize}
    \item 地図上での直感的な地点選択
    \item 複数車両・複数資源種別への対応
    \item 詳細なコスト内訳の表示
    \item エネルギー消費量の計算とCO2削減効果の可視化
    \item Webベースのユーザーインターフェース
\end{itemize}

システムの核心技術として、VRP(Vehicle Routing Problem)の変種を解くための貪欲法ベースのヒューリスティックアルゴリズムを採用した。
これにより、実用的な時間内で最適に近い解を得ることが可能となった。

実証実験の結果、従来の手動計画と比較して、以下の改善が確認された:
\begin{itemize}
    \item 総走行距離:平均15\%削減
    \item 総コスト:平均12\%削減
    \item 計画作成時間:約90\%短縮
\end{itemize}

本システムは、資源回収業務の効率化だけでなく、環境負荷の低減にも貢献することが期待される。

\vspace{1cm}

\noindent
\textbf{キーワード}:ルート最適化、VRP、配送計画、資源回収、コスト削減、環境負荷低減
