% 第1章 序論
\section{序論}

\subsection{研究背景}

近年、持続可能な社会の実現に向けて、資源の有効活用とリサイクルの推進が重要な課題となっている。
自治体や廃棄物処理業者は、限られた予算と人的資源の中で、効率的な資源回収システムの構築を求められている。

従来の資源回収業務では、担当者の経験と勘に基づいた手動での計画立案が主流であった。
しかし、この方法には以下のような課題が存在する:

\begin{itemize}
    \item \textbf{非効率な経路}:最適でないルートによる走行距離の増加
    \item \textbf{高コスト}:燃料費や人件費の増大
    \item \textbf{計画作成の負担}:熟練者の経験に依存し、時間がかかる
    \item \textbf{環境負荷}:不必要なCO2排出量の増加
    \item \textbf{柔軟性の欠如}:条件変更時の再計画が困難
\end{itemize}

これらの課題を解決するため、情報技術を活用した最適化手法の導入が期待されている。

\subsection{研究目的}

本研究の目的は、資源回収業務の効率化とコスト削減を実現する最適化システムを開発することである。
具体的には、以下の目標を設定した:

\begin{enumerate}
    \item \textbf{総コストの最小化}:固定費と変動費を考慮した総コストの削減
    \item \textbf{使いやすいインターフェース}:専門知識がなくても操作可能なシステム
    \item \textbf{柔軟な対応}:複数車両・複数資源種別への対応
    \item \textbf{環境配慮}:エネルギー消費量とCO2排出量の可視化
    \item \textbf{実用的な処理時間}:実務で使用可能な計算速度
\end{enumerate}

\subsection{本報告書の構成}

本報告書は以下の構成となっている:

第2章では、システムの概要について述べる。
第3章では、システムアーキテクチャと設計思想を説明する。
第4章では、実装の詳細と採用した技術について述べる。
第5章では、システムの主要機能を詳述する。
第6章では、システムの使用方法を説明する。
第7章では、実証実験の結果を示す。
第8章では、結果の考察と今後の課題について述べる。
第9章では、本研究のまとめを行う。

\subsection{想定ユーザー}

本システムは、以下のユーザーを想定して開発された:

\begin{itemize}
    \item \textbf{自治体の資源回収担当者}:回収ルートの計画立案とコスト削減策の検討
    \item \textbf{廃棄物処理業者の配送計画担当者}:効率的な配送計画の作成と車両運用の最適化
    \item \textbf{環境コンサルタント}:CO2削減効果の試算とEV導入効果の分析
    \item \textbf{ロジスティクス最適化の研究者}:VRPアルゴリズムの評価と実データでの検証
\end{itemize}
