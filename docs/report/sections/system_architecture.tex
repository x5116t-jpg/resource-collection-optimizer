% 第3章 システムアーキテクチャ
\section{システムアーキテクチャ}

\subsection{全体構成}

本システムは、入力層、処理層、データ格納層、出力層の4層で構成される。
図\ref{fig:system_architecture}にシステムの全体構成を示す。

\begin{figure}[H]
\centering
\fbox{
\begin{minipage}{0.9\textwidth}
\centering
\vspace{0.5cm}
\textbf{【入力層】} \\
道路ネットワークデータ、マスタデータ、ユーザー選択 \\
\vspace{0.3cm}
$\downarrow$ \\
\vspace{0.3cm}
\textbf{【データ格納層】} \\
NetworkXグラフ、地点レジストリ、車種カタログ、\\
距離行列、空間インデックス \\
\vspace{0.3cm}
$\downarrow$ \\
\vspace{0.3cm}
\textbf{【処理層】} \\
データ読込 → 地点選択 → 車種割当 → 距離計算 →\\
ルート最適化 → 結果生成 \\
\vspace{0.3cm}
$\downarrow$ \\
\vspace{0.3cm}
\textbf{【出力層】} \\
最適ルート情報、コスト詳細、エネルギー消費量、地図表示 \\
\vspace{0.5cm}
\end{minipage}
}
\caption{システムの全体構成}
\label{fig:system_architecture}
\end{figure}

\subsection{入力層}

入力層では、以下の3種類のデータを受け取る:

\subsubsection{道路ネットワークデータ}
\begin{itemize}
    \item \textbf{形式}:JSON
    \item \textbf{内容}:ノード情報(地点ID、緯度経度)、エッジ情報(道路接続、距離)
    \item \textbf{データソース}:OSM(OpenStreetMap)
\end{itemize}

\subsubsection{マスタデータ}
\begin{itemize}
    \item \textbf{resources.json}:資源種別(紙、プラスチック等)の特性データ
    \item \textbf{vehicles.json}:車種情報(容量、コスト、エネルギー消費原単位)
    \item \textbf{compatibility.json}:資源と車種の適合性マトリクス
\end{itemize}

\subsubsection{ユーザー選択情報}
\begin{itemize}
    \item \textbf{車庫地点}:出発・帰着地点の座標
    \item \textbf{回収地点}:回収する地点の座標、資源種別、回収量
    \item \textbf{集積場所}:終点の座標
\end{itemize}

\subsection{データ格納層}

データ格納層では、最適化処理に必要なデータ構造を保持する:

\begin{table}[H]
\centering
\caption{データ格納層の構成要素}
\label{tab:data_layer}
\begin{tabular}{lp{9cm}}
\toprule
\textbf{データ構造} & \textbf{役割} \\
\midrule
NetworkXグラフ & ノードとエッジの関係性、最短経路探索の基盤 \\
地点レジストリ & 車庫・回収地点・集積場所の統合管理 \\
車種カタログ & 利用可能な車種のリストとコスト情報 \\
距離行列 & 全地点間の最短距離(N×N行列) \\
空間インデックス & 地図クリック時の最寄りノード高速検索 \\
\bottomrule
\end{tabular}
\end{table}

\subsection{処理層}

処理層では、以下の6ステップで最適化を実行する:

\subsubsection{ステップ1:データ読込・初期化}
\begin{itemize}
    \item JSON形式の道路ネットワークを読み込み
    \item NetworkXグラフに変換
    \item マスタデータのパースとキャッシュ構築
\end{itemize}

\subsubsection{ステップ2:地点選択}
\begin{itemize}
    \item ユーザーの地図クリック位置を取得
    \item 空間インデックスで最寄りノードを検索
    \item 選択地点の管理と検証
\end{itemize}

\subsubsection{ステップ3:車種割当プラン生成}
\begin{itemize}
    \item 資源種別ごとに対応可能な車種を抽出
    \item コスト評価関数による最適車種の選択
    \item 複数資源の場合の車種割当最適化
\end{itemize}

\subsubsection{ステップ4:距離行列計算}
\begin{itemize}
    \item 選択された全地点間の最短経路を計算
    \item NetworkXのDijkstra法による実装
    \item 計算結果のキャッシュ保存
\end{itemize}

\subsubsection{ステップ5:ルート最適化}
\begin{itemize}
    \item VRPを解くためのヒューリスティックアルゴリズム
    \item 容量制約と資源適合性の考慮
    \item 局所最適化による改善
\end{itemize}

\subsubsection{ステップ6:結果生成・可視化}
\begin{itemize}
    \item コスト詳細の計算(固定費・変動費)
    \item エネルギー消費量の算出
    \item 経路の再構成と地図表示用データの生成
\end{itemize}

\subsection{出力層}

出力層では、最適化結果を以下の形式で提供する:

\begin{itemize}
    \item \textbf{最適ルート情報}:訪問順序、総距離、使用車種
    \item \textbf{コスト詳細}:固定費・変動費の項目別内訳、総コスト
    \item \textbf{エネルギー情報}:総消費電力量、CO2削減効果
    \item \textbf{地図表示}:インタラクティブな経路可視化
\end{itemize}

\subsection{技術スタック}

本システムの実装に使用した主要技術を表\ref{tab:tech_stack}に示す。

\begin{table}[H]
\centering
\caption{主要技術スタック}
\label{tab:tech_stack}
\begin{tabular}{llp{6cm}}
\toprule
\textbf{カテゴリ} & \textbf{技術} & \textbf{用途} \\
\midrule
言語 & Python 3.8+ & システム全体の実装 \\
Webフレームワーク & Streamlit & UIの構築、インタラクティブ操作 \\
グラフ処理 & NetworkX & 道路ネットワーク解析、最短経路計算 \\
地図表示 & Folium & インタラクティブ地図の表示 \\
データ処理 & Pandas & テーブル編集、データ整形 \\
\bottomrule
\end{tabular}
\end{table}
