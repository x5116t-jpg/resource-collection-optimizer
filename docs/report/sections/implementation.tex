% 第4章 実装
\section{実装}

\subsection{最適化アルゴリズム}

\subsubsection{問題定義}

本システムで扱う最適化問題は、VRP(Vehicle Routing Problem)の変種である。
問題の定式化を以下に示す。

\textbf{目的関数:}
\begin{equation}
\min \quad \text{総コスト} = \sum_{v \in V} (\text{固定費}_v + \text{変動費}_v)
\end{equation}

\textbf{制約条件:}
\begin{itemize}
    \item 各回収地点を必ず1回訪問
    \item 車両容量を超えない
    \item 車庫から出発し、集積場所で終了
    \item 資源と車種の適合性を満たす
\end{itemize}

\subsubsection{アルゴリズムの概要}

本システムでは、以下のアプローチを採用した:

\begin{enumerate}
    \item \textbf{初期解生成}:近傍優先の貪欲法
    \item \textbf{局所最適化}:2-opt法による改善
    \item \textbf{車種選択}:コスト評価関数による最適化
\end{enumerate}

\textbf{計算量:} $O(N^2 \times M)$ (N: 地点数、M: 車種数)

\subsubsection{貪欲法による初期解生成}

\begin{lstlisting}[language=Python, caption=初期解生成アルゴリズム(疑似コード)]
def generate_initial_solution(depot, pickup_points, sink):
    route = [depot]
    remaining = set(pickup_points)
    current_load = 0

    while remaining:
        # 現在地から最も近い未訪問地点を選択
        nearest = find_nearest(route[-1], remaining)

        # 容量制約チェック
        if current_load + nearest.demand <= capacity:
            route.append(nearest)
            current_load += nearest.demand
            remaining.remove(nearest)
        else:
            break  # 容量オーバー

    route.append(sink)
    return route
\end{lstlisting}

\subsubsection{2-opt法による局所最適化}

\begin{lstlisting}[language=Python, caption=2-opt法による改善(疑似コード)]
def improve_by_2opt(route):
    improved = True
    while improved:
        improved = False
        for i in range(1, len(route) - 2):
            for j in range(i + 1, len(route)):
                # ルートの一部を反転
                new_route = two_opt_swap(route, i, j)

                # 改善されたか確認
                if calculate_cost(new_route) < calculate_cost(route):
                    route = new_route
                    improved = True
    return route
\end{lstlisting}

\subsection{コスト計算}

\subsubsection{固定費の計算}

固定費は、年間固定費をkm単価に換算して計算する:

\begin{equation}
\text{固定費} = \sum_{i=1}^{n} \frac{\text{年間固定費}_i}{\text{年間走行距離}} \times \text{実走行距離}
\end{equation}

固定費項目には、人件費、車両償却費、保険料、車検・点検費用等が含まれる。

\subsubsection{変動費の計算}

変動費は、距離単価を元に計算する:

\begin{equation}
\text{変動費} = \sum_{j=1}^{m} \text{単価}_j \times \text{実走行距離}
\end{equation}

変動費項目には、燃料費(または電気代)、タイヤ代、修繕費、消耗品費等が含まれる。

\subsubsection{総コストの計算}

\begin{equation}
\text{総コスト} = \text{固定費} + \text{変動費}
\end{equation}

\subsubsection{エネルギー消費量の計算}

EV(電気自動車)の場合のエネルギー消費量:

\begin{equation}
\text{消費電力量}[\text{kWh}] = \text{消費原単位}[\text{kWh/km}] \times \text{実走行距離}[\text{km}]
\end{equation}

\subsection{データ構造}

\subsubsection{地点レジストリ}

地点情報を管理するためのクラス構造:

\begin{lstlisting}[language=Python, caption=地点レジストリの実装例]
@dataclass
class Point:
    point_id: str
    lat: float
    lon: float
    point_type: PointType  # DEPOT, PICKUP, SINK
    demand: int = 0
    resource_type: str = ""

class PointRegistry:
    def __init__(self):
        self.points: Dict[str, Point] = {}

    def add_point(self, point: Point):
        self.points[point.point_id] = point

    def get_pickups(self) -> List[Point]:
        return [p for p in self.points.values()
                if p.point_type == PointType.PICKUP]
\end{lstlisting}

\subsubsection{車種カタログ}

車種情報を管理するためのクラス構造:

\begin{lstlisting}[language=Python, caption=車種カタログの実装例]
@dataclass
class VehicleType:
    name: str
    capacity_kg: int
    fixed_cost: float
    per_km_cost: float
    fixed_cost_per_km: float
    energy_consumption_kwh_per_km: float

class VehicleCatalog:
    def __init__(self):
        self.vehicles: List[VehicleType] = []

    def add_vehicle(self, vehicle: VehicleType):
        self.vehicles.append(vehicle)

    def find_compatible(self, resource: str) -> List[VehicleType]:
        # 適合性チェック
        return [v for v in self.vehicles
                if self.is_compatible(v, resource)]
\end{lstlisting}

\subsection{キャッシュ戦略}

計算の高速化のため、以下のキャッシュ戦略を採用した:

\begin{table}[H]
\centering
\caption{キャッシュ戦略}
\label{tab:cache_strategy}
\begin{tabular}{lp{6cm}p{4cm}}
\toprule
\textbf{対象} & \textbf{キャッシュ内容} & \textbf{効果} \\
\midrule
NetworkXグラフ & 道路ネットワーク全体 & 読み込み時間の削減 \\
距離行列 & 地点間の最短距離 & 再計算の回避 \\
空間インデックス & 座標の空間分割 & 最寄り検索の高速化 \\
最短経路 & 計算済みの経路 & 経路再計算の回避 \\
\bottomrule
\end{tabular}
\end{table}
