\subsection{システムの起動方法}

本システムは、配布されたフォルダ「ResouceCollection\_05」に含まれる\texttt{run\_app.bat}をダブルクリックするだけで起動できる。以下に、起動から運用までの手順を説明する。

\subsubsection{初回起動}

\paragraph{ステップ1:必要な環境の確認}

本システムの実行には、以下の環境が必要である:

\begin{itemize}
    \item \textbf{OS}:Windows 10以降(推奨:Windows 11)
    \item \textbf{Python}:3.8以降(インストールされていない場合は、\url{https://www.python.org/}からダウンロード)
    \item \textbf{インターネット接続}:初回起動時に必要なライブラリのダウンロードと地図表示のため
    \item \textbf{Webブラウザ}:Google Chrome、Firefox、Edge(最新版)
\end{itemize}

\paragraph{ステップ2:システムの起動}

\begin{enumerate}
    \item 配布フォルダ「\texttt{D:\textbackslash py\textbackslash Resource Collection\textbackslash ResouceCollection\_05}」を開く
    \item \texttt{run\_app.bat}ファイルをダブルクリックする
    \item コマンドプロンプトウィンドウが開き、システムの初期化が始まる
    \item 初回起動時は必要なPythonライブラリのインストールが自動的に行われる(数分かかる場合がある)
    \item インストールが完了すると、Webブラウザが自動的に起動し、システムのUIが表示される
    \item ブラウザのアドレスバーには「\texttt{http://localhost:8501}」と表示される
\end{enumerate}

\subsubsection{システムの操作手順}

システムのUIは、左側のサイドバーと右側のメイン画面で構成される。以下に、基本的な操作手順を示す。

\paragraph{1. 道路ネットワークの選択}

\begin{enumerate}
    \item サイドバーの「道路ネットワークの選択」セクションで、使用する道路データを選択する
    \item デフォルトでは「群馬県道路ネットワーク」が選択されている
    \item カスタムネットワークを使用する場合は、「カスタムファイルをアップロード」を選択し、JSON形式の道路データファイルを選択する
\end{enumerate}

\paragraph{2. 拠点の設定}

\begin{enumerate}
    \item サイドバーの「拠点設定」セクションで、「デポ(車庫)を設定」ボタンをクリックする
    \item メイン画面の地図上で、車両の出発・帰着地点をクリックする
    \item 設定された地点に赤色のマーカーが表示される
\end{enumerate}

\paragraph{3. 回収地点の設定}

\begin{enumerate}
    \item サイドバーの「回収地点設定」セクションで、「回収地点を追加」ボタンをクリックする
    \item 地図上で回収地点をクリックする
    \item ポップアップウィンドウが表示されるので、以下の情報を入力する:
    \begin{itemize}
        \item 資源種別(11種類から選択)
        \item 回収量(kg単位)
        \item 地点名(任意)
    \end{itemize}
    \item 「追加」ボタンをクリックして確定する
    \item 設定された地点に青色のマーカーが表示される
    \item 複数の回収地点を追加する場合は、この操作を繰り返す
\end{enumerate}

\paragraph{4. 集積場所の設定}

\begin{enumerate}
    \item サイドバーの「集積場所設定」セクションで、「集積場所を設定」ボタンをクリックする
    \item 地図上で資源の最終目的地をクリックする
    \item 設定された地点に緑色のマーカーが表示される
\end{enumerate}

\paragraph{5. 車両の選択}

\begin{enumerate}
    \item サイドバーの「車両選択」セクションで、使用する車両タイプを選択する
    \item システムが自動的に、選択された資源に適合する車両のみを表示する
    \item 適合する車両が複数ある場合、コスト試算を確認して選択できる
\end{enumerate}

\paragraph{6. 最適化の実行}

\begin{enumerate}
    \item すべての設定が完了したら、サイドバー下部の「最適化を実行」ボタンをクリックする
    \item システムが最適経路を計算する(通常数秒~数十秒)
    \item 計算が完了すると、最適化されたルートが地図上に表示される
\end{enumerate}

\subsubsection{結果の確認}

最適化計算が完了すると、以下の情報がメイン画面に表示される。

\paragraph{地図表示}

\begin{itemize}
    \item 最適化されたルートが青色の線で表示される
    \item 各区間をクリックすると、区間距離とコストがポップアップ表示される
    \item ズームイン・ズームアウト、地図の移動が可能
\end{itemize}

\paragraph{ルート情報}

\begin{itemize}
    \item 訪問順序と各地点間の距離
    \item 総走行距離(km)
    \item 推定所要時間
\end{itemize}

\paragraph{コスト詳細}

\begin{itemize}
    \item 変動費12項目の内訳と金額
    \item 固定費13項目の内訳と金額
    \item 総コスト(円)
    \item コスト単価(円/km)
\end{itemize}

\paragraph{エネルギー消費}

\begin{itemize}
    \item 燃料消費量(L)
    \item CO$_2$排出量(kg)
\end{itemize}

\subsubsection{データのエクスポート}

計算結果は、以下の形式でエクスポート可能である:

\begin{enumerate}
    \item サイドバーの「結果をエクスポート」セクションで、出力形式を選択する
    \item \textbf{CSV形式}:Excelで開けるデータファイル(ルート情報、コスト詳細)
    \item \textbf{JSON形式}:プログラムで読み込み可能な構造化データ
    \item \textbf{HTML形式}:地図を含む完全なレポート(印刷可能)
    \item \textbf{PDF形式}:印刷用の正式レポート(要追加ライブラリ)
    \item 「ダウンロード」ボタンをクリックして、ファイルを保存する
\end{enumerate}

\subsubsection{システムの終了}

\begin{enumerate}
    \item Webブラウザを閉じる
    \item コマンドプロンプトウィンドウで「Ctrl + C」キーを押す
    \item 確認メッセージが表示されたら「Y」を入力してEnterキーを押す
    \item コマンドプロンプトウィンドウが閉じ、システムが終了する
\end{enumerate}

\subsubsection{トラブルシューティング}

\paragraph{起動しない場合}

\begin{itemize}
    \item Pythonが正しくインストールされているか確認する
    \item コマンドプロンプトで「\texttt{python --version}」を実行し、バージョン3.8以降であることを確認する
    \item インターネット接続を確認する(初回起動時のライブラリインストールに必要)
    \item ウイルス対策ソフトによってブロックされていないか確認する
\end{itemize}

\paragraph{地図が表示されない場合}

\begin{itemize}
    \item インターネット接続を確認する
    \item ブラウザのキャッシュをクリアする
    \item ブラウザを変更してみる(Chrome推奨)
\end{itemize}

\paragraph{最適化が終わらない場合}

\begin{itemize}
    \item 回収地点が多すぎる可能性がある(10地点以下を推奨)
    \item 一度ブラウザをリロードして、設定をやり直す
    \item 道路ネットワークデータが正しく読み込まれているか確認する
\end{itemize}

\paragraph{エラーメッセージが表示される場合}

\begin{itemize}
    \item エラーメッセージの内容を記録する
    \item 入力データが正しいか確認する(座標、資源量など)
    \item システムを再起動する
    \item 問題が解決しない場合は、システム管理者に連絡する
\end{itemize}
