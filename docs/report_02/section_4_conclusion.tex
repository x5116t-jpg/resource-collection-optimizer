\subsection{成果のまとめ}

本研究では、群馬県内における未利用資源の効率的な収集運搬を支援するシステムを開発した。主な成果を以下にまとめる。

\subsubsection{システムの実現}

交通安全環境研究所で開発中の「地域交通計画立案ツール」を基盤として、未利用資源運搬に特化した最適化システムを実現した。Streamlitフレームワークを用いたWebアプリケーションとして実装し、専門知識を持たないユーザーでも直感的に操作できるUIを提供している。

\subsubsection{体系的なデータ整備}

\begin{itemize}
    \item \textbf{未利用資源}:11種類の資源(建設廃材、農業残渣、林業残材、食品廃棄物、廃プラスチック、金属スクラップ、古紙・段ボール、剪定枝・草、家畜糞尿、下水汚泥、廃食用油)について、物理的・化学的特性と運搬上の注意点を整理した
    \item \textbf{運搬車両}:14種類の車両(軽トラックから10tクラスまで、および特殊車両)について、積載容積、燃費、コスト構造を詳細に調査・設定した
    \item \textbf{適合性マトリックス}:154組み合わせ(11資源×14車両)について、法令遵守、安全性、経済性の観点から適合性を判定し、体系的に整理した
\end{itemize}

\subsubsection{最適化アルゴリズムの実装}

\begin{itemize}
    \item グラフ理論に基づくDijkstra法とA*アルゴリズムによる最短経路探索を実装した
    \item 車両経路問題(VRP)の解法として、貪欲法による初期解生成と2-opt法による局所改善を組み合わせた
    \item 容量制約と資源適合性を考慮した現実的な最適化を実現した
\end{itemize}

\textbf{注意点}:本システムでは訪問順序の最適化に厳密解を求めるアプローチを採用しているため、回収地点数が増加すると計算時間が指数的に増大する。実用上は\textbf{10カ所程度がリミット}である。より多くの地点を扱う場合は、ヒューリスティック手法への切り替えが必要となる。

\subsubsection{詳細なコスト計算機能}

運搬費用を変動費12項目、固定費13項目に分解し、合計25項目の詳細なコスト計算を実現した。これにより、運搬計画の経済性を事前に評価し、コスト削減の余地を特定することが可能となった。

\subsubsection{実用的な提供形態}

配布フォルダに\texttt{run\_app.bat}を含めることで、複雑なセットアップ作業なしに、ダブルクリック一つでシステムを起動できる形態を実現した。計算結果はCSV、JSON、HTML形式でエクスポート可能であり、既存の業務フローへの統合が容易である。

\subsection{今後の展望}

本研究で開発したシステムを基盤として、以下の発展的研究の可能性が考えられる。

\subsubsection{貨客混載による地域内最適化への拡張}

現在のシステムは未利用資源の運搬のみを対象としているが、これを人流(旅客)と物流(貨物)を統合した貨客混載システムへ拡張することで、地域交通全体の最適化が可能となる。図\ref{fig:future_system}に、拡張システムの概念を示す。

\begin{figure}[H]
\centering
\fbox{
\begin{minipage}{0.9\textwidth}
\centering
\vspace{0.5cm}
\textbf{【現行システム】} \\
未利用資源運搬の最適化 \\
\vspace{0.3cm}
$\downarrow$ 拡張 \\
\vspace{0.3cm}
\textbf{【将来システム:貨客混載最適化】} \\
\vspace{0.3cm}
\begin{tabular}{cc}
\fbox{\begin{minipage}{0.4\textwidth}
\centering
\textbf{人流} \\
通勤・通学 \\
通院・買い物 \\
観光
\end{minipage}} &
\fbox{\begin{minipage}{0.4\textwidth}
\centering
\textbf{物流} \\
未利用資源 \\
農産物・加工品 \\
宅配便
\end{minipage}}
\end{tabular} \\
\vspace{0.3cm}
$\downarrow$ 統合最適化 \\
\vspace{0.3cm}
\textbf{総コスト最小化} \\
人流コスト + 物流コスト $<$ 独立運行コスト \\
\vspace{0.5cm}
\end{minipage}
}
\caption{貨客混載システムへの拡張概念}
\label{fig:future_system}
\end{figure}

\subsubsection{研究課題1:混載可能性の分類}

未利用資源を、旅客と同時運搬可能なものと不可能なものに分類する必要がある。

\paragraph{混載可能な資源}
\begin{itemize}
    \item 古紙・段ボール(梱包済み、無臭)
    \item 廃食用油(密閉容器入り)
    \item 廃プラスチック(洗浄済み、袋詰め)
    \item 農産物(出荷品質)
\end{itemize}

\paragraph{混載不可能な資源}
\begin{itemize}
    \item 家畜糞尿(臭気が強い)
    \item 下水汚泥(衛生上の問題)
    \item 食品廃棄物(腐敗リスク)
    \item 建設廃材(汚損・安全性の問題)
\end{itemize}

\subsubsection{研究課題2:混載不可資源の同時運搬}

混載不可能とされる資源同士でも、適切な工夫により同一車両で効率的に運搬できる可能性がある。

\paragraph{区画分離方式}
車両内部を物理的に区画分離することで、臭気や汚染の相互影響を防止する。例えば、家畜糞尿と建設廃材を同一車両で運搬する場合、密閉コンテナと開放エリアに分離する。

\paragraph{時間帯分離方式}
同一車両を時間帯によって用途分けし、洗浄・消毒を徹底することで、午前は旅客、午後は貨物といった運用が可能となる。

\paragraph{車両改造方式}
着脱式の内装(シートカバー、床マット等)や可動式隔壁を導入し、柔軟な用途変更を実現する。

\subsubsection{研究課題3:貨客混載の優位性検証}

貨客独立運搬と貨客混載のコストを詳細に比較し、混載の優位性を定量的に検証する必要がある。

\paragraph{比較対象}
\begin{equation}
\text{独立運搬コスト} = \text{人流運搬コスト} + \text{資源運搬コスト}
\end{equation}
\begin{equation}
\text{混載運搬コスト} = \text{統合運行コスト} + \text{追加設備費用}
\end{equation}

\paragraph{シミュレーション項目}
\begin{itemize}
    \item 運行パターン:デマンド型、定時定路線型、ハイブリッド型
    \item 需要変動:平日/休日、時間帯別の人流・物流需要
    \item 地域特性:市街地、郊外、中山間地域での比較
    \item 車両タイプ:小型・中型・大型、専用車・兼用車
    \item 環境負荷:CO$_2$排出量、エネルギー消費量
\end{itemize}

\paragraph{期待される効果}
\begin{itemize}
    \item 車両稼働率の向上(空車走行の削減)
    \item 人件費の削減(ドライバー数の削減)
    \item 環境負荷の低減(総走行距離の削減)
    \item 地域サービスの維持(採算性の改善)
\end{itemize}

\subsubsection{研究課題4:統合最適化システムの開発}

人流と物流を統合的に扱う最適化システムの開発が必要である。

\paragraph{技術的課題}
\begin{itemize}
    \item マルチモーダル需要予測:人流と物流の時空間需要の統合的予測
    \item 動的配車アルゴリズム:リアルタイムの需要変動に対応した配車最適化
    \item 制約条件の複雑化:旅客の乗車時間制約、貨物の時間指定配送
    \item 公平性の確保:旅客サービス品質と物流効率のトレードオフ
\end{itemize}

\paragraph{システム要件}
\begin{itemize}
    \item リアルタイム性:需要発生から配車決定まで数分以内
    \item スケーラビリティ:広域(市町村レベル)での運用に対応
    \item 利用者インターフェース:住民・事業者向けの予約・追跡機能
    \item データ連携:既存の公共交通・物流システムとの情報連携
\end{itemize}

\subsubsection{期待される社会的効果}

貨客混載システムの実現により、以下の社会的効果が期待される:

\begin{itemize}
    \item \textbf{持続可能な地域交通の実現}:過疎地域での交通サービス維持
    \item \textbf{未利用資源の利活用促進}:収集運搬コスト低減によるリサイクル推進
    \item \textbf{地域経済の活性化}:農産物等の輸送効率化による競争力向上
    \item \textbf{環境負荷の低減}:総走行距離削減によるCO$_2$排出量削減
    \item \textbf{ドライバー不足への対応}:車両・人員の効率的活用
\end{itemize}

\subsubsection{実証実験の提案}

理論的検討を経て、以下の段階的な実証実験を提案する:

\begin{enumerate}
    \item \textbf{フェーズ1(6ヶ月)}:混載可能資源の選定と小規模実験(1-2路線)
    \item \textbf{フェーズ2(12ヶ月)}:車両改造・運用ルールの確立と中規模実験(地区レベル)
    \item \textbf{フェーズ3(24ヶ月)}:統合システムの開発と広域実証(市町村レベル)
\end{enumerate}

本研究で開発したシステムは、これらの発展的研究の基盤として活用され、持続可能な地域交通システムの実現に貢献することが期待される。
