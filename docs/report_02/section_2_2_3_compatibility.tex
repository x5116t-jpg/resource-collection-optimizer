\subsubsection{資源・車両対応表}

本システムでは、11種類の未利用資源と14種類の運搬車両の適合性を体系的に整理している。表\ref{tab:compatibility}に、各資源と車両の組み合わせにおける適合性を示す。

\begin{table}[H]
\centering
\caption{資源と運搬車両の適合性マトリックス}
\label{tab:compatibility}
\begin{small}
\begin{tabular}{lccccccccccc}
\toprule
\textbf{車両} & \rotatebox{90}{\textbf{建設廃材}} & \rotatebox{90}{\textbf{農業残渣}} & \rotatebox{90}{\textbf{林業残材}} & \rotatebox{90}{\textbf{食品廃棄物}} & \rotatebox{90}{\textbf{廃プラスチック}} & \rotatebox{90}{\textbf{金属スクラップ}} & \rotatebox{90}{\textbf{古紙・段ボール}} & \rotatebox{90}{\textbf{剪定枝・草}} & \rotatebox{90}{\textbf{家畜糞尿}} & \rotatebox{90}{\textbf{下水汚泥}} & \rotatebox{90}{\textbf{廃食用油}} \\
\midrule
軽トラック & ○ & ○ & ○ & × & ○ & ○ & ○ & ○ & × & × & ○ \\
2t平ボディ & ○ & ○ & ○ & × & ○ & ○ & ○ & ○ & × & × & ○ \\
2tダンプ & ○ & ○ & ○ & × & × & ○ & × & ○ & ○ & × & × \\
4t平ボディ & ○ & ○ & ○ & × & ○ & ○ & ○ & ○ & × & × & ○ \\
4tダンプ & ○ & ○ & ○ & × & × & ○ & × & ○ & ○ & × & × \\
4tユニック & ○ & × & ○ & × & × & ○ & × & × & × & × & × \\
4tウイング & ○ & ○ & ○ & ○ & ○ & × & ○ & ○ & ○ & × & ○ \\
4tパッカー & × & ○ & ○ & ○ & ○ & × & ○ & ○ & × & × & × \\
4tアームロール & ○ & ○ & ○ & ○ & ○ & ○ & ○ & ○ & ○ & × & ○ \\
10t平ボディ & ○ & ○ & ○ & × & ○ & ○ & ○ & ○ & × & × & ○ \\
10tダンプ & ○ & ○ & ○ & × & × & ○ & × & ○ & ○ & × & × \\
10tウイング & ○ & ○ & ○ & ○ & ○ & × & ○ & ○ & ○ & × & ○ \\
大型パッカー & × & ○ & ○ & ○ & ○ & × & ○ & ○ & × & × & × \\
バキューム車 & × & × & × & × & × & × & × & × & ○ & ○ & × \\
\bottomrule
\end{tabular}
\end{small}
\end{table}

\vspace{0.5em}
\noindent
\textbf{記号の意味}:○=適合、×=不適合

\paragraph{適合性判定の基準}

車両と資源の適合性は、以下の観点から総合的に評価されている:

\begin{itemize}
    \item \textbf{車両構造上の適合性}:開放型(平ボディ、ダンプ)、密閉型(ウイング、パッカー)、専用型(バキューム)などの車両構造が資源の性状に適しているか
    \item \textbf{法令遵守}:廃棄物処理法等の関連法規における運搬基準を満たしているか
    \item \textbf{衛生上の安全性}:汁漏れ、臭気、飛散等のリスクが適切に管理できるか
    \item \textbf{経済性}:当該車両で運搬することが経済的に合理的か
\end{itemize}

\paragraph{主な適合・不適合の理由}

\subparagraph{ダンプ車×食品廃棄物(不適合)}
食品廃棄物は水分含有率が高く、開放型のダンプ車では汁漏れリスクがあり、衛生上不適切である。

\subparagraph{パッカー車×金属スクラップ(不適合)}
金属スクラップは硬質で重量物が多く、パッカー車の圧縮機構を破損させるリスクが高いため不適合である。

\subparagraph{ユニック車×農業残渣(不適合)}
農業残渣は軽量でクレーンによる積み下ろしの必要がなく、クレーン付きユニック車を使用することは経済性を欠くため不適合とした。

\subparagraph{バキューム車×固体資源(不適合)}
バキューム車は液状・泥状物質専用の構造であり、固体資源の運搬には使用できない。下水汚泥と家畜糞尿(液状・泥状のもの)のみに適合する。

\subparagraph{ウイング車・アームロール車(高汎用性)}
4tウイング車と4tアームロール車は、密閉構造と高い作業性を兼ね備えており、多様な資源に対応可能である。特にアームロール車は10種類の資源に適合し、最も汎用性が高い。

\paragraph{条件付き適合}

一部の組み合わせでは、追加対策により適合化が可能である:

\begin{itemize}
    \item \textbf{平ボディ×古紙・段ボール}:雨天時はシート養生により適合化(追加コスト2-5円/km)
    \item \textbf{平ボディ×廃プラスチック}:防風シートにより飛散防止(追加コスト2-5円/km)
    \item \textbf{平ボディ×廃食用油}:密閉容器(缶・ペットボトル)使用により適合化(追加コスト5-15円/km)
\end{itemize}

本システムでは、これらの適合性データを基に、選択された資源に対して最適な車両を自動的に推奨する機能を実装している。
