本付録では、システムで使用される計算式の詳細と、使用したソフトウェア・ライブラリについて説明する。

\subsection{最短経路アルゴリズムの詳細}

\subsubsection{Dijkstra法}

Dijkstra法は、単一始点からすべてのノードへの最短経路を求めるアルゴリズムである。計算量は$O((V + E) \log V)$($V$はノード数、$E$はエッジ数)である。

\paragraph{アルゴリズムの概要}
\begin{enumerate}
    \item すべてのノードの距離を無限大に初期化し、始点のみを0とする
    \item 未訪問ノードの中から、最小距離のノードを選択する
    \item 選択したノードの隣接ノードについて、距離を更新する(緩和処理)
    \item すべてのノードを訪問するまで2-3を繰り返す
\end{enumerate}

\subsubsection{A*アルゴリズム}

A*アルゴリズムは、ヒューリスティック関数を用いてDijkstra法を拡張したアルゴリズムである。

評価関数:
\begin{equation}
f(n) = g(n) + h(n)
\end{equation}

ここで、$g(n)$は始点からノード$n$までの実際の距離、$h(n)$はノード$n$から目標ノードまでの推定距離(ヒューリスティック値)である。

本システムでは、ヒューリスティック関数としてユークリッド距離を使用する:
\begin{equation}
h(n) = \sqrt{(x_n - x_{\text{goal}})^2 + (y_n - y_{\text{goal}})^2}
\end{equation}

\subsection{コスト計算の詳細式}

\subsubsection{燃料費}

\begin{equation}
C_{\text{fuel}} = \frac{P_{\text{fuel}}}{F} \times D
\end{equation}

ここで、$P_{\text{fuel}}$は燃料単価(円/L)、$F$は燃費(km/L)、$D$は走行距離(km)である。

\subsubsection{運転手人件費}

\begin{equation}
C_{\text{labor}} = \frac{W_{\text{hour}}}{V_{\text{avg}}} \times D
\end{equation}

ここで、$W_{\text{hour}}$は時給(円/時)、$V_{\text{avg}}$は平均速度(km/時)である。

\subsubsection{作業時間人件費}

\begin{equation}
C_{\text{work}} = \frac{T_{\text{work}} \times W_{\text{hour}}}{D_{\text{avg}}} \times D
\end{equation}

ここで、$T_{\text{work}}$は1回あたりの作業時間(時)、$D_{\text{avg}}$は平均運搬距離(km)である。

\subsubsection{固定費の距離単価換算}

年間固定費を距離単価に換算する:
\begin{equation}
C_{\text{fixed/km}} = \frac{C_{\text{fixed/year}}}{D_{\text{annual}}}
\end{equation}

ここで、$C_{\text{fixed/year}}$は年間固定費(円/年)、$D_{\text{annual}}$は年間走行距離(km/年)である。

\subsubsection{総コストの計算}

総コストは以下の式で計算される:
\begin{equation}
C_{\text{total}} = D \times \left( \sum_{i=1}^{12} C_{\text{var},i} + \frac{\sum_{j=1}^{13} C_{\text{fixed},j}}{D_{\text{annual}}} \right)
\end{equation}

ここで、$C_{\text{var},i}$は変動費の第$i$項目(12項目)、$C_{\text{fixed},j}$は固定費の第$j$項目(13項目)である。

\subsection{CO$_2$排出量の計算}

\subsubsection{ガソリン車・ディーゼル車}

\begin{equation}
\text{CO}_2\text{排出量}[\text{kg}] = \frac{D}{F} \times \alpha
\end{equation}

ここで、$D$は走行距離(km)、$F$は燃費(km/L)、$\alpha$は排出係数(kg-CO$_2$/L)である。

排出係数は以下の値を使用する(環境省データ):
\begin{itemize}
    \item 軽油:$\alpha = 2.58$ kg-CO$_2$/L
    \item ガソリン:$\alpha = 2.32$ kg-CO$_2$/L
\end{itemize}

\subsection{使用ソフトウェアとライブラリ}

本システムの開発に使用した主要なソフトウェアとライブラリを以下に示す。

\begin{table}[H]
\centering
\caption{使用ソフトウェアとライブラリのバージョン情報}
\label{tab:software}
\begin{tabular}{llp{6cm}}
\toprule
\textbf{ライブラリ} & \textbf{バージョン} & \textbf{用途} \\
\midrule
Python & 3.8+ & システム全体の実装言語 \\
Streamlit & 1.28.0 & Webアプリケーションフレームワーク \\
NetworkX & 3.1 & グラフ理論と最短経路計算 \\
Folium & 0.14.0 & インタラクティブ地図の表示 \\
Pandas & 2.1.0 & データ処理と表形式データ操作 \\
NumPy & 1.24.3 & 数値計算とベクトル演算 \\
SciPy & 1.11.1 & 空間インデックス(KD木) \\
OR-Tools & 9.7 & 制約付き最適化問題の解決 \\
\bottomrule
\end{tabular}
\end{table}

\subsection{システム要件}

\subsubsection{ハードウェア要件}

\begin{table}[H]
\centering
\caption{推奨ハードウェア仕様}
\label{tab:hardware}
\begin{tabular}{lp{8cm}}
\toprule
\textbf{項目} & \textbf{推奨仕様} \\
\midrule
CPU & 2コア以上(4コア推奨) \\
メモリ & 4GB以上(8GB推奨) \\
ストレージ & 500MB以上の空き容量 \\
ディスプレイ & 1920×1080以上の解像度 \\
ネットワーク & インターネット接続(地図表示用) \\
\bottomrule
\end{tabular}
\end{table}

\subsubsection{ソフトウェア要件}

\begin{itemize}
    \item \textbf{OS}:Windows 10/11、macOS 10.14以降、Linux(Ubuntu 20.04以降推奨)
    \item \textbf{Python}:3.8以上(3.10推奨)
    \item \textbf{ブラウザ}:Google Chrome(最新版)、Firefox(最新版)、Edge(最新版)
\end{itemize}

\subsection{データファイル構成}

システムは以下のディレクトリ構造を持つ:

\begin{verbatim}
ResouceCollection_05/
├── run_app.bat                      # システム起動用バッチファイル
├── app.py                           # メインアプリケーション
├── requirements.txt                 # 依存ライブラリリスト
├── data/
│   ├── processed/
│   │   ├── compatibility.json       # 適合性マトリックス
│   │   ├── vehicles.json            # 車両諸元データ
│   │   └── resources.json           # 資源特性データ
│   └── networks/
│       └── default_network.json     # デフォルト道路ネットワーク
├── src/
│   ├── optimization/                # 最適化モジュール
│   ├── visualization/               # 可視化モジュール
│   └── cost/                        # コスト計算モジュール
└── docs/
    ├── report_02/                   # 本報告書
    └── claudedocs/                  # ユーザーガイド
\end{verbatim}

\subsection{計算パフォーマンス}

本システムの計算パフォーマンスを表\ref{tab:performance}に示す。

\begin{table}[H]
\centering
\caption{計算パフォーマンス(標準的なPC環境での測定値)}
\label{tab:performance}
\begin{tabular}{lrrr}
\toprule
\textbf{回収地点数} & \textbf{距離計算} & \textbf{最適化} & \textbf{合計} \\
\midrule
5地点 & 2.1秒 & 1.8秒 & 4.4秒 \\
10地点 & 3.8秒 & 4.2秒 & 8.8秒 \\
20地点 & 8.5秒 & 12.3秒 & 22.3秒 \\
\bottomrule
\end{tabular}
\end{table}

\textbf{注意}:回収地点数が10カ所を超えると計算時間が急激に増大するため、実用上は10カ所程度を上限とすることを推奨する。
