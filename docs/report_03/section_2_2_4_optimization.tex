\subsubsection{最適化アルゴリズム}

本システムでは、未利用資源の収集運搬経路を最適化するために、車両経路問題(Vehicle Routing Problem, VRP)の解法を実装している。VRPは、複数の地点を訪問する最適な経路を決定する組合せ最適化問題であり、物流分野において広く研究されている。

\paragraph{問題の定式化}

本システムが扱う問題は、以下のように定式化される:

\begin{itemize}
    \item \textbf{入力}:
    \begin{itemize}
        \item デポ(車両の出発地点・帰着地点)
        \item 複数の回収地点(未利用資源の発生場所)
        \item 集積場所(資源の最終目的地)
        \item 道路ネットワーク(ノードとエッジのグラフ構造)
        \item 車両の容量制約
        \item 各回収地点での資源量
    \end{itemize}
    \item \textbf{目的}:
    \begin{itemize}
        \item 総走行距離の最小化
        \item 運搬費用の最小化
        \item 車両台数の最小化
    \end{itemize}
    \item \textbf{制約}:
    \begin{itemize}
        \item すべての回収地点を訪問すること
        \item 車両の積載容量を超えないこと
        \item すべての車両がデポから出発し、デポに帰着すること
        \item 資源は最終的に集積場所に運搬されること
    \end{itemize}
\end{itemize}

\paragraph{最短経路探索}

本システムでは、2つの地点間の最も短い道のりを見つける計算方法を採用しています。

道路ネットワークを地図上の点(交差点など)と線(道路)でとらえ、出発地点から目的地点までのすべての可能な経路の中から、最も距離が短い経路を自動的に見つけ出します。この計算を、車庫から各回収地点、回収地点の間、そして回収地点から集積場所まで、すべての地点の組み合わせについて事前に行い、その結果を記録しておきます。こうすることで、実際に経路を決める際には記録した情報を参照するだけで済み、すばやく最適な経路を組み立てることができます。

\paragraph{車両経路問題の解法}

本システムでは、複数の回収地点を効率的に巡回するため、以下の方法で訪問順序を決定しています。

まず、車庫から最も近い回収地点を訪れ、そこから次に近い地点へと順番に移動していきます。車の積載量がいっぱいになったら、集積場所へ運搬してから次の回収地点へ向かいます。この作業を、すべての回収地点を回り終えるまで繰り返します。

最初に決めた訪問順序は、必ずしも最も効率的とは限りません。そこで、経路を見直して改善を行います。具体的には、訪問順序を部分的に入れ替えてみて、移動距離が短くなるかどうかを確認します。改善の余地がなくなるまで、この見直し作業を繰り返すことで、より効率的な経路を見つけ出します。

また、車の積載容量を超えないように、各回収地点で積み込む資源の量を常に確認しながら、訪問できる地点を選んでいきます。

