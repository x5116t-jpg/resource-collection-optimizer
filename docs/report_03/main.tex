\documentclass[12pt,a4paper]{jsarticle}

% ntsel-reportスタイルファイルの読み込み
\usepackage{ntsel-report}

% rotating パッケージ(表の回転用)
\usepackage{rotating}

% タイトル情報
\title{地域交通計画立案ツールを改良した\\群馬県内の未利用資源運搬等に関する\\基礎的研究}
\author{独立行政法人自動車技術総合機構\\交通安全環境研究所}
\date{令和8年2月}

\begin{document}

% タイトルページ
\maketitlepage

% 目次
\newpage
\tableofcontents
\newpage

% ========================================
% 1. 序論
% ========================================
\section{序論}

\subsection{仕様の確認}

本研究は、以下の仕様に基づいて実施された。

\begin{itemize}
    \item \textbf{名称}:地域交通計画立案ツールを改良した群馬県内の未利用資源運搬等に関する基礎的研究
    \item \textbf{目的}:交通安全環境研究所で開発中の「地域交通計画立案ツール」を改良し、群馬県内の未利用資源運搬時の最短経路と運搬費用を推計する
    \item \textbf{期間}:契約締結日から令和8年2月27日まで
    \item \textbf{内容}:
    \begin{enumerate}
        \item 交通安全環境研究所で開発中の「地域交通計画立案ツール」(以下、ツールと呼ぶ)を改良する
        \item ツールの変数として、数種類の未利用資源を使用する
        \item ツールの変数として、数種類のモビリティ(軽トラ等の貨物車等)を使用する
        \item 単一の収集場所から単一の集積場所まで運搬する際の、最短経路と運搬費用を推計する
    \end{enumerate}
    \item \textbf{提出形式}:印刷物3部、デジタルデータ(数値データ、編集可能書類データ、PDFデータ)
    \item \textbf{提出期限}:2026年2月27日(金)
\end{itemize}

\subsection{研究の目的}

未利用資源の効果的な収集運搬は、地域の循環型社会形成において重要な課題である。本研究では、群馬県内における未利用資源の運搬最適化を目的として、以下の目標を設定した。

\begin{enumerate}
    \item 未利用資源と運搬車両の適合性を体系的に整理する
    \item 最短経路探索アルゴリズムを実装し、運搬距離を最小化する
    \item 運搬費用を詳細に推計し、経済性を評価する
    \item 実用的なWebアプリケーションとして実装し、利用可能な形態で提供する
\end{enumerate}

これらの目標達成により、未利用資源の効率的な収集運搬計画の策定を支援し、地域におけるバイオマス利活用の促進に貢献することを目指す。

\newpage

% ========================================
% 2. システムの説明
% ========================================
\section{システムの説明}

\subsection{システムの全体構成}

本システムは、未利用資源の収集運搬計画を最適化するため、以下の主要コンポーネントから構成される(図\ref{fig:system_architecture})。

\begin{figure}[H]
\centering
\begin{minipage}{0.9\textwidth}
\centering

% ユーザー入力ボックス(システムの外側)
\fbox{
\begin{minipage}{0.85\textwidth}
\centering
\vspace{0.3cm}
\textbf{【ユーザー入力】} \\
\vspace{0.2cm}
資源回収地点・資源種別・回収量、\\
集積地点、車庫地点
\vspace{0.3cm}
\end{minipage}
}

\vspace{0.3cm}
$\downarrow$ \textit{入力}
\vspace{0.3cm}

% システム全体ボックス
\fbox{
\begin{minipage}{0.85\textwidth}
\centering
\vspace{0.3cm}
\textbf{【システム】}
\vspace{0.3cm}

\hrule
\vspace{0.3cm}

\textbf{【入力層】} \\
\textit{システムデフォルト}:\\
道路ネットワークデータ、マスタデータ\\
(運搬車両の諸元、未利用資源の特性、車両と資源の適合マトリクス)
\vspace{0.3cm}

$\downarrow$ \\
\vspace{0.3cm}

\textbf{【処理層】} \\
地点間の最短ルート探索 → 回収順序の最適化
\vspace{0.3cm}

$\downarrow$ \\
\vspace{0.3cm}

\textbf{【出力層】} \\
最適ルート情報、コスト詳細、\\
エネルギー消費量、地図表示 \\
\vspace{0.3cm}
\end{minipage}
}

\end{minipage}
\caption{システムの全体構成}
\label{fig:system_architecture}
\end{figure}

\subsubsection{システム構成要素}

システムは3層構造で構成されており、図\ref{fig:system_architecture}に示す各層の要素を以下に説明する。

\paragraph{ユーザー入力}
ユーザーが入力・選択する情報である:
\begin{itemize}
    \item 車庫地点:車両の出発地点・帰着地点の座標
    \item 資源回収地点:資源を回収する地点の座標
    \item 資源種別:回収する資源の種類(紙、プラスチック、ガラス等)
    \item 回収量:各地点での資源回収量
    \item 集積地点:資源の最終目的地の座標
\end{itemize}

\paragraph{入力層}
入力層では、システムが最適化計算を実行するために必要なデータを管理する。

\begin{detaillist}
    \detailitem{システムデフォルトのデータ} システムに予め組み込まれており、ユーザーの入力作業を軽減する基本データである。

    \textbf{道路ネットワークデータ}:OpenStreetMap(OSM)またはカスタムJSON形式で提供される道路ネットワーク情報である。ノード(交差点・地点)とエッジ(道路区間)の接続関係、各道路区間の距離情報を含む。

    \textbf{マスタデータ}:システムの基礎となる静的データであり、以下の3つから構成される。
    \begin{itemize}
        \item 14種類の運搬車両の諸元(積載容積、燃費、購入費用、固定費、変動費)
        \item 11種類の未利用資源の特性(比重、取扱い注意事項)
        \item 車両と資源の適合性マトリックス(154組み合わせ)
    \end{itemize}

    これらのデータはシステム内に標準装備されており、ユーザーは必要に応じて選択のみを行えばよい。
\end{detaillist}

\paragraph{処理層}
処理層では、入力されたデータを用いて最適化計算を実行する。

\begin{detaillist}
    \detailitem{地点間の最短ルート探索} 道路ネットワーク上で、すべての地点間の最短経路をDijkstra法で計算する。交差点や道路区間の接続関係を考慮し、実際の道路距離に基づいた最短ルートを求める。

    \detailitem{回収順序の最適化} 車両経路問題を解決し、最適な訪問順序を決定する。容量制約と資源適合性を考慮し、総走行距離やコストが最小となるルートを計算する。
\end{detaillist}

\paragraph{出力層}
出力層では、最適化結果をユーザーに分かりやすい形式で提供する。

\begin{detaillist}
    \detailitem{最適ルート情報} 車両ごとの訪問順序、各区間の距離、総走行距離、所要時間を表示する。

    \detailitem{コスト詳細} 変動費12項目と固定費13項目の内訳、項目別金額、総コストを詳細に表示する。

    \detailitem{エネルギー消費量} 燃料消費量、CO$_2$排出量を車両の燃費データから算出し表示する。

    \detailitem{地図表示} 最適化されたルートをインタラクティブな地図上に可視化する。経路は色分けされ、クリックすると詳細情報がポップアップ表示される。
\end{detaillist}

\subsubsection{データフロー}

システムのデータフローは以下の通りである:

\begin{enumerate}
    \item ユーザーが道路ネットワークを選択し、システムに読み込む
    \item 地図上で車庫地点、資源回収地点、集積地点を設定する
    \item 使用する車両タイプと運搬する資源タイプを選択する
    \item システムが車両・資源適合性を自動的に検証する
    \item 最適化エンジンが地点間の最短ルートを探索し、回収順序を最適化する
    \item 結果が地図上に可視化され、詳細レポートが生成される
    \item ユーザーは結果をCSV、JSON、HTML形式でエクスポート可能
\end{enumerate}

\subsubsection{技術スタック}

本システムは以下の技術により構築されている:

\begin{itemize}
    \item \textbf{グラフ処理}:NetworkX(道路ネットワークのグラフ表現と経路探索)
    \item \textbf{地図可視化}:Folium(インタラクティブマップ生成)
    \item \textbf{データ処理}:Pandas、NumPy(データ管理と数値計算)
    \item \textbf{最適化}:OR-Tools(制約付き最適化問題の解決)
    \item \textbf{データ形式}:JSON、CSV(データの入出力)
\end{itemize}


\subsection{システムの詳細}

本節では、システムを構成する主要要素について詳細に説明する。

\subsubsection{資源}

本システムでは、群馬県内で発生する11種類の未利用資源を対象としている。各資源の特性と収集における注意点を以下に示す。

\begin{detaillist}
    \detailitem{建設廃材} 建設・解体工事から発生する木材、金属、コンクリートがらなどを含む。比重が大きく重量物が多いため、積載重量に注意が必要である。混合廃棄物の場合は分別処理が必要となる。

    \detailitem{農業残渣} 稲わら、もみ殻、麦わらなど農作物収穫後の残留物である。軽量でかさばる特性を持ち、乾燥状態と湿潤状態で性状が大きく異なる。運搬時は飛散防止対策が重要である。

    \detailitem{林業残材} 間伐材、枝葉、樹皮、製材端材などを指す。長尺物が多く、水分含有率により重量が大きく変動する。積載時は荷崩れ防止のための固定が必要である。

    \detailitem{食品廃棄物} 生ごみ、食品残渣、売れ残り食品などを含む。水分含有率が高く腐敗しやすいため、迅速な運搬が求められる。汁漏れ防止のため、密閉容器の使用または専用車両が必要である。

    \detailitem{廃プラスチック} 容器包装、フィルム、硬質プラスチックなどを含む。軽量でかさばり、風で飛散しやすいため、シート養生または密閉車両での運搬が推奨される。

    \detailitem{金属スクラップ} 鉄くず、非鉄金属、廃家電から回収した金属などを含む。比重が極めて大きいため、積載重量に特に注意が必要である。クレーン付き車両の使用が効率的な場合が多い。

    \detailitem{古紙・段ボール} 新聞紙、雑誌、段ボール箱などを含む。水濡れ厳禁であり、圧縮により積載効率が向上する。雨天時はシート養生または箱型車両の使用が必須である。

    \detailitem{剪定枝・草} 庭木剪定枝、草刈り残渣、落ち葉などを含む。かさばるため、破砕処理により積載効率が大幅に向上する。パッカー車による圧縮収集も有効である。

    \detailitem{家畜糞尿} 牛糞、豚糞、鶏糞などを含む。半固形から液状まで性状が多様であり、臭気対策が必須である。液状のものはバキューム車、半固形のものはダンプまたは密閉コンテナで運搬する。

    \detailitem{下水汚泥} 浄化槽汚泥、し尿などを含む。液状から泥状であり、専用車両(バキューム車)での運搬が法令により義務付けられている。

    \detailitem{廃食用油} 使用済み天ぷら油、業務用廃油などを含む。液体であるため、漏洩防止のため密閉容器(缶・ペットボトル)での回収を前提とする。容器ごと運搬することで、平ボディやウイング車でも運搬可能である。
\end{detaillist}

\vspace{1em}

これらの資源は、その物理的・化学的特性に応じて適切な運搬車両を選択する必要がある。次項では、これらの資源運搬に使用可能な車両について詳述する。

\subsubsection{運搬車両}

本システムでは、未利用資源の収集運搬に使用される14種類の車両を対象としている。各車両の特徴と主な用途を以下に分類して示す。

\paragraph{小型車両(2-3tクラス)}

\subparagraph{軽トラック}
最大積載量350kg程度、車両総重量2t未満の小型貨物車である。狭い道路での機動性に優れ、普通免許で運転可能である。少量の資源回収や農村部での巡回収集に適している。

\subparagraph{2tトラック(平ボディ)}
最大積載量2t、荷台がフラットな汎用トラックである。荷台寸法は約3.1m×1.6m×0.4m(長さ×幅×高さ)であり、様々な形状の資源を積載可能で、側面からの積み下ろしが容易である。小規模事業所からの資源回収や市街地での収集業務に使用される。

\subparagraph{2tダンプ}
油圧装置により荷台を傾斜させて荷下ろしする車両である。土砂、廃棄物など流動性のある資源に適しており、建設廃材、剪定枝、家畜糞尿などの運搬に使用される。

\paragraph{中型車両(4tクラス)}

\subparagraph{4t平ボディ}
最大積載量4t程度、車両総重量8t未満の中型トラックである。荷台寸法は約5.0-6.2m×2.1-2.2m×0.6mであり、運搬効率と機動性のバランスが良く、最も汎用性が高い。中規模事業所からの資源回収や中距離輸送に適している。

\subparagraph{4tダンプ}
2tダンプよりも大容量で、建設現場などでの大量運搬に適する。建設廃材、解体材、土砂類の大量運搬に使用される。

\subparagraph{4tユニック車(クレーン付きトラック)}
小型クレーンを搭載し、重量物の積み下ろしが可能な車両である。クレーン搭載により荷台容積は減少するが、重量物の単独作業が可能である。金属スクラップや建設廃材など重量物の運搬に使用される。

\subparagraph{4tウイング車}
側面が翼のように開閉する箱型トラックである。側面全開により積み下ろし作業性が極めて良好で、雨風から荷物を保護できる。古紙、廃プラスチック、食品廃棄物など多様な資源の運搬に使用される。

\subparagraph{4tパッカー車(塵芥車)}
圧縮機構を備えた廃棄物収集専用車両である。回転板やプレスにより廃棄物を圧縮し、積載効率が高い。食品廃棄物、廃プラスチック、古紙、剪定枝などの収集に使用される。

\subparagraph{4tアームロール車}
コンテナを脱着できる装置を備えた車両である。複数のコンテナを準備することで車両の稼働率を向上でき、様々な資源に対応可能である。建設廃材、産業廃棄物、多種類の資源の効率的収集に使用される。

\paragraph{大型車両(10tクラス)}

\subparagraph{10t平ボディ}
最大積載量10t程度、車両総重量25t未満の大型トラックである。荷台寸法は約9m×2.4m×0.5mであり、大量輸送に適し、長距離輸送でも効率的である。大規模施設からの資源回収、広域収集、長距離輸送に使用される。

\subparagraph{10tダンプ}
建設廃材や土砂の大量輸送に最適である。解体現場や大規模工事現場からの廃材運搬に使用される。

\subparagraph{10tウイング車}
大容量と作業性を両立し、最も効率的な大型運搬車両である。古紙、廃プラスチック、パレット化された資源の大量輸送に使用される。

\subparagraph{大型パッカー車}
圧縮機構を備えた大型廃棄物収集車である。大量の廃棄物を効率的に圧縮収集でき、広域での一般廃棄物収集や大規模イベントでの廃棄物回収に使用される。

\paragraph{特殊用途車両}

\subparagraph{バキューム車(4tクラス)}
真空ポンプにより液体・泥状物を吸引収集する専用車両である。液状・泥状の資源専用で、密閉タンク構造を持つ。家畜糞尿、下水汚泥、食品廃液の収集運搬に使用される。

\vspace{1em}

これらの車両の諸元(積載容積、燃費、購入費用など)は、実際のメーカーカタログおよび業界データに基づいて詳細に設定されている(付録A参照)。

\subsubsection{資源・車両対応表}

本システムでは、11種類の未利用資源と14種類の運搬車両の適合性を体系的に整理している。表\ref{tab:compatibility}に、各資源と車両の組み合わせにおける適合性を示す。

\begin{table}[H]
\centering
\caption{資源と運搬車両の適合性マトリックス}
\label{tab:compatibility}
\begin{small}
\begin{tabular}{lccccccccccc}
\toprule
\textbf{車両} & \rotatebox{90}{\textbf{建設廃材}} & \rotatebox{90}{\textbf{農業残渣}} & \rotatebox{90}{\textbf{林業残材}} & \rotatebox{90}{\textbf{食品廃棄物}} & \rotatebox{90}{\textbf{廃プラスチック}} & \rotatebox{90}{\textbf{金属スクラップ}} & \rotatebox{90}{\textbf{古紙・段ボール}} & \rotatebox{90}{\textbf{剪定枝・草}} & \rotatebox{90}{\textbf{家畜糞尿}} & \rotatebox{90}{\textbf{下水汚泥}} & \rotatebox{90}{\textbf{廃食用油}} \\
\midrule
軽トラック & ○ & ○ & ○ & × & ○ & ○ & ○ & ○ & × & × & ○ \\
2t平ボディ & ○ & ○ & ○ & × & ○ & ○ & ○ & ○ & × & × & ○ \\
2tダンプ & ○ & ○ & ○ & × & × & ○ & × & ○ & ○ & × & × \\
4t平ボディ & ○ & ○ & ○ & × & ○ & ○ & ○ & ○ & × & × & ○ \\
4tダンプ & ○ & ○ & ○ & × & × & ○ & × & ○ & ○ & × & × \\
4tユニック & ○ & × & ○ & × & × & ○ & × & × & × & × & × \\
4tウイング & ○ & ○ & ○ & ○ & ○ & × & ○ & ○ & ○ & × & ○ \\
4tパッカー & × & ○ & ○ & ○ & ○ & × & ○ & ○ & × & × & × \\
4tアームロール & ○ & ○ & ○ & ○ & ○ & ○ & ○ & ○ & ○ & × & ○ \\
10t平ボディ & ○ & ○ & ○ & × & ○ & ○ & ○ & ○ & × & × & ○ \\
10tダンプ & ○ & ○ & ○ & × & × & ○ & × & ○ & ○ & × & × \\
10tウイング & ○ & ○ & ○ & ○ & ○ & × & ○ & ○ & ○ & × & ○ \\
大型パッカー & × & ○ & ○ & ○ & ○ & × & ○ & ○ & × & × & × \\
バキューム車 & × & × & × & × & × & × & × & × & ○ & ○ & × \\
\bottomrule
\end{tabular}
\end{small}
\end{table}

\vspace{0.5em}
\noindent
\textbf{記号の意味}:○=適合、×=不適合

\paragraph{適合性判定の基準}

車両と資源の適合性は、以下の観点から総合的に評価されている:

\begin{itemize}
    \item \textbf{車両構造上の適合性}:開放型(平ボディ、ダンプ)、密閉型(ウイング、パッカー)、専用型(バキューム)などの車両構造が資源の性状に適しているか
    \item \textbf{法令遵守}:廃棄物処理法等の関連法規における運搬基準を満たしているか
    \item \textbf{衛生上の安全性}:汁漏れ、臭気、飛散等のリスクが適切に管理できるか
    \item \textbf{経済性}:当該車両で運搬することが経済的に合理的か
\end{itemize}

\paragraph{主な適合・不適合の理由}

\subparagraph{ダンプ車×食品廃棄物(不適合)}
食品廃棄物は水分含有率が高く、開放型のダンプ車では汁漏れリスクがあり、衛生上不適切である。

\subparagraph{パッカー車×金属スクラップ(不適合)}
金属スクラップは硬質で重量物が多く、パッカー車の圧縮機構を破損させるリスクが高いため不適合である。

\subparagraph{ユニック車×農業残渣(不適合)}
農業残渣は軽量でクレーンによる積み下ろしの必要がなく、クレーン付きユニック車を使用することは経済性を欠くため不適合とした。

\subparagraph{バキューム車×固体資源(不適合)}
バキューム車は液状・泥状物質専用の構造であり、固体資源の運搬には使用できない。下水汚泥と家畜糞尿(液状・泥状のもの)のみに適合する。

\subparagraph{ウイング車・アームロール車(高汎用性)}
4tウイング車と4tアームロール車は、密閉構造と高い作業性を兼ね備えており、多様な資源に対応可能である。特にアームロール車は10種類の資源に適合し、最も汎用性が高い。

\paragraph{条件付き適合}

一部の組み合わせでは、追加対策により適合化が可能である:

\begin{itemize}
    \item \textbf{平ボディ×古紙・段ボール}:雨天時はシート養生により適合化(追加コスト2-5円/km)
    \item \textbf{平ボディ×廃プラスチック}:防風シートにより飛散防止(追加コスト2-5円/km)
    \item \textbf{平ボディ×廃食用油}:密閉容器(缶・ペットボトル)使用により適合化(追加コスト5-15円/km)
\end{itemize}

本システムでは、これらの適合性データを基に、選択された資源に対して最適な車両を自動的に推奨する機能を実装している。

\subsubsection{最適化アルゴリズム}

本システムでは、未利用資源の収集運搬経路を最適化するために、車両経路問題(Vehicle Routing Problem, VRP)の解法を実装している。VRPは、複数の地点を訪問する最適な経路を決定する組合せ最適化問題であり、物流分野において広く研究されている。

\paragraph{問題の定式化}

本システムが扱う問題は、以下のように定式化される:

\begin{itemize}
    \item \textbf{入力}:
    \begin{itemize}
        \item デポ(車両の出発地点・帰着地点)
        \item 複数の回収地点(未利用資源の発生場所)
        \item 集積場所(資源の最終目的地)
        \item 道路ネットワーク(ノードとエッジのグラフ構造)
        \item 車両の容量制約
        \item 各回収地点での資源量
    \end{itemize}
    \item \textbf{目的}:
    \begin{itemize}
        \item 総走行距離の最小化
        \item 運搬費用の最小化
        \item 車両台数の最小化
    \end{itemize}
    \item \textbf{制約}:
    \begin{itemize}
        \item すべての回収地点を訪問すること
        \item 車両の積載容量を超えないこと
        \item すべての車両がデポから出発し、デポに帰着すること
        \item 資源は最終的に集積場所に運搬されること
    \end{itemize}
\end{itemize}

\paragraph{最短経路探索アルゴリズム}

本システムでは、グラフ理論に基づく最短経路探索アルゴリズムを採用している。

\subparagraph{Dijkstra法}
Dijkstra法は、単一始点からすべてのノードへの最短経路を求める古典的なアルゴリズムである。計算量はO((V + E) log V)(Vはノード数、Eはエッジ数)であり、負の重みがない場合に正確な解を得られる。

本システムでは、NetworkXライブラリの\texttt{dijkstra\_path}関数を使用して実装している。デポから各回収地点、回収地点間、回収地点から集積場所への最短経路を事前計算し、これを基に全体の経路を構築する。

\subparagraph{A*アルゴリズム}
A*アルゴリズムは、ヒューリスティック関数を用いてDijkstra法を拡張したアルゴリズムである。目標ノードまでの推定距離(ヒューリスティック値)を利用することで、探索の効率を向上させる。

本システムでは、ユークリッド距離をヒューリスティック関数として使用し、NetworkXの\texttt{astar\_path}関数により実装している。特に目標地点が明確な場合、Dijkstra法よりも高速に解を得られる。

\paragraph{車両経路問題の解法}

最短経路探索を基礎として、以下の手法により車両経路問題を解決している。

\subparagraph{最近傍法(Nearest Neighbor Heuristic)}
初期解の構築に使用される貪欲法である:
\begin{enumerate}
    \item デポから最も近い未訪問の回収地点を訪問
    \item 現在地から最も近い未訪問の回収地点を訪問
    \item 積載容量に達したら集積場所へ運搬
    \item すべての回収地点を訪問するまで繰り返す
\end{enumerate}

計算量はO(n²)(nは回収地点数)であり、高速に解を得られるが、最適解を保証しない。

\subparagraph{2-opt法による改善}
最近傍法で得られた初期解を改善するための局所探索法である:
\begin{enumerate}
    \item 経路中の2つのエッジを選択
    \item エッジを入れ替えた場合の経路長を計算
    \item 改善があれば入れ替えを採用
    \item 改善がなくなるまで繰り返す
\end{enumerate}

この手法により、交差する経路を解消し、経路長を短縮できる。

\subparagraph{容量制約の処理}
各車両には積載容量の制約があり、これを考慮した経路構築が必要である。本システムでは、以下の方法で容量制約を処理している:

\begin{itemize}
    \item 回収地点を訪問する際、現在の積載量に資源量を加算
    \item 積載量が車両容量を超える場合、その回収地点は訪問せず、次の候補を選択
    \item 積載量が容量に達したら、集積場所へ運搬し積載量をリセット
    \item すべての回収地点を訪問するまでこのプロセスを繰り返す
\end{itemize}

\paragraph{計算の効率化}

大規模な問題に対しても現実的な時間で解を得るため、以下の効率化を実施している:

\begin{itemize}
    \item \textbf{距離行列の事前計算}:すべてのノード間の最短距離を事前に計算し、キャッシュする
    \item \textbf{候補地点の絞り込み}:ヒューリスティックにより、明らかに非効率な経路を探索対象から除外
    \item \textbf{並列計算}:複数の車両の経路を独立に計算できる場合、並列処理を適用
    \item \textbf{早期終了条件}:一定時間経過後、または十分に良い解が得られた時点で探索を終了
\end{itemize}

これらのアルゴリズムにより、数十から数百の回収地点を持つ実用的な問題を、数秒から数分で解決できる。

\subsubsection{コスト計算}

本システムでは、未利用資源の運搬に関する総コストを詳細に推計する機能を実装している。コストは変動費と固定費に大別され、それぞれ複数の項目から構成される。

\paragraph{変動費(走行距離に比例するコスト)}

変動費は走行距離に応じて変化する費用であり、距離単価(円/km)で表される。以下の12項目から構成される。

\subparagraph{燃料費}
計算式:燃料費 = 燃料単価 ÷ 燃費

燃料単価は、軽油150円/L、ガソリン170円/L(2025年10月時点の全国平均)を使用している。燃費は車両タイプごとに設定されており、軽トラック13-16 km/L、2t-4tトラック5-11 km/L、10tトラック3.5-5 km/Lの範囲である。

例:4t平ボディ(燃費6 km/L)の場合、150円 ÷ 6 km/L = 25円/km

\subparagraph{運転手人件費}
計算式:人件費 = 時給 ÷ 平均速度

時給は厚生労働省「賃金構造基本統計調査」に基づき、軽トラック1,500-1,800円、2t-4tトラック1,800-2,200円、10tトラック2,000-2,500円を設定している。平均速度は市街地走行を想定し35-50 km/hである。

\subparagraph{高速道路料金}
車種区分別の基本料金を設定している:
\begin{itemize}
    \item 軽トラック・普通車:25-30円/km
    \item 中型車(2t-4t):30-50円/km
    \item 大型車(10t):60-75円/km
\end{itemize}

高速道路を使用しない場合は0円として計算される。ETC割引適用により実際の料金は変動する。

\subparagraph{タイヤ交換費}
計算式:タイヤ交換費 = (タイヤ本数 × 単価 + 工賃)÷ タイヤ寿命走行距離

軽トラック2円/km、2t-4tトラック3-5円/km、10tトラック10-15円/kmの範囲である。

\subparagraph{修理費}
年間平均修理費を年間走行距離で除算して算出する。軽トラック3-5円/km、2t-4tトラック6-12円/km、10tトラック15-25円/kmの範囲である。車両年式と使用状況により大きく変動する。

\subparagraph{作業時間人件費}
積み下ろし作業時間の人件費を平均運搬距離で除算して算出する。1回の運搬あたり積み下ろし各30分、計1時間を前提とし、軽トラック15-20円/km、2t-4tトラック25-35円/km、10tトラック30-45円/kmである。

\subparagraph{補助員人件費}
2名体制が必要な場合の追加人件費である。軽トラックは0円(1名で作業可能)、2t-4tトラックは0-30円/km(重量物の場合のみ2名体制)、10tトラックは30-45円/km(基本2名体制)である。

\subparagraph{回収容器費}
フレコンバッグ、プラスチックコンテナ等の償却費である。軽トラック5-8円/km、2t-4tトラック8-25円/km、10tトラック10-25円/kmである。アームロール車はコンテナ費用が高額となる。

\subparagraph{消耗品費}
作業用手袋、ロープ、シート、清掃用品等の費用である。軽トラック2-3円/km、2t-4tトラック3-8円/km、10tトラック8-12円/kmである。

\subparagraph{通信費}
携帯電話、業務無線、GPS利用料を走行距離で除算する。全車種で1-5円/kmである(月額通信費3,000-8,000円を想定)。

\subparagraph{マニフェスト費用}
産業廃棄物管理票の発行費用である。軽トラック3-5円/km、2t-4tトラック5-12円/km、10tトラック12-18円/kmである。産業廃棄物の場合のみ必要であり、一般廃棄物は不要である。

\subparagraph{処理費}
中間処理施設または最終処分場でのゲート料である。軽トラック20-30円/km、2t-4tトラック30-100円/km、10tトラック80-180円/kmである。資源の種類、地域、処理方法により大きく変動し、リサイクル可能資源は低額、焼却・埋立は高額となる。

\paragraph{固定費(年間で発生するコスト)}

固定費は走行距離に関わらず年間で発生する費用である。総コスト算出時は、固定費を年間走行距離で除算して距離単価に換算する。以下の13項目から構成される。

\subparagraph{車両購入費および減価償却費}
新車価格の市場相場(2025年10月時点)に基づき、軽トラック100-150万円、2tトラック300-500万円、4tトラック500-1,200万円、10tトラック2,000-2,800万円である。

減価償却費は定額法、耐用年数5年を前提として算出される(車両購入費 ÷ 5年)。税法上の貨物自動車の耐用年数は5年であるが、実際の使用年数は10-15年程度である。

\subparagraph{自動車税}
車両総重量に応じた年間税額であり、軽トラック5,000-11,000円、2tトラック15,000-22,000円、4tトラック25,000-35,000円、10tトラック40,000-60,000円である。

\subparagraph{重量税}
車検時(2年ごと)に納付する税金を年換算したものであり、軽トラック3,000-5,000円/年、2tトラック12,000-16,000円/年、4tトラック20,000-25,000円/年、10tトラック32,000-41,000円/年である。

\subparagraph{保険(自賠責・任意)}
法定義務の自賠責保険と任意保険の合計である。軽トラック5.8-20万円/年、2tトラック9.5-38万円/年、4tトラック17.5-60万円/年、10tトラック27.5-90万円/年である。任意保険は事故歴、運転者の年齢・経験により大きく変動する。

\subparagraph{車検費用および定期点検費用}
2年ごとの車検費用(法定費用除く)と法定点検(3ヶ月、6ヶ月、12ヶ月)の費用の合計を年換算したものである。軽トラック5-8万円/年、2tトラック8-13万円/年、4tトラック13-27万円/年、10tトラック22-40万円/年である。

\subparagraph{車庫賃料}
車両保管場所の年間賃料である。地域により大きく変動し、軽トラック12-24万円/年、2tトラック18-36万円/年、4tトラック24-48万円/年、10tトラック36-72万円/年である。都市部は高額、地方は低額であり、自社所有地の場合は0円である。

\subparagraph{許認可費用}
産業廃棄物収集運搬業許可等の申請・更新費用を年換算したものである。新規許可約10万円、更新約7万円(5年ごと)であり、講習会受講費も含む。

\subparagraph{システム利用料}
配車管理システム、GPS追跡システム、デジタルタコグラフ等の利用料であり、全車種で3-18万円/年である(月額3,000-15,000円)。事業規模により導入システムが異なる。

\subparagraph{福利厚生費}
健康診断、制服貸与、研修費用等であり、軽トラック15-25万円/年、2t-4tトラック20-50万円/年、10tトラック35-60万円/年である。

\subparagraph{社会保険料}
健康保険、厚生年金、雇用保険、労災保険の事業主負担分である。運転手の年収(300-500万円)の約25-30%を想定し、軽トラック80-120万円/年、2t-4tトラック100-200万円/年、10tトラック150-220万円/年である。

\paragraph{総コストの算出}

総コストは以下の式で計算される:

\begin{equation}
    \text{総コスト} = \text{変動費単価} \times \text{走行距離} + \frac{\text{年間固定費}}{\text{年間走行距離}} \times \text{走行距離}
\end{equation}

システムは、選択された車両タイプと走行距離に基づき、これらのコストを自動的に計算し、詳細な内訳とともにユーザーに提示する。これにより、運搬計画の経済性を事前に評価することが可能となる。



\newpage

% ========================================
% 3. 使い方
% ========================================
\section{使い方}

\subsection{システムの起動方法}

本システムは、配布されたフォルダ「ResouceCollection\_05」に含まれる\texttt{run\_app.bat}をダブルクリックするだけで起動できる。以下に、起動から運用までの手順を説明する。

\subsubsection{初回起動}

\paragraph{ステップ1:必要な環境の確認}

本システムの実行には、以下の環境が必要である:

\begin{itemize}
    \item \textbf{OS}:Windows 10以降(推奨:Windows 11)
    \item \textbf{Python}:3.8以降(インストールされていない場合は、\url{https://www.python.org/}からダウンロード)
    \item \textbf{インターネット接続}:初回起動時に必要なライブラリのダウンロードと地図表示のため
    \item \textbf{Webブラウザ}:Google Chrome、Firefox、Edge(最新版)
\end{itemize}

\paragraph{ステップ2:システムの起動}

\begin{enumerate}
    \item 配布フォルダ「\texttt{D:\textbackslash py\textbackslash Resource Collection\textbackslash ResouceCollection\_05}」を開く
    \item \texttt{run\_app.bat}ファイルをダブルクリックする
    \item コマンドプロンプトウィンドウが開き、システムの初期化が始まる
    \item 初回起動時は必要なPythonライブラリのインストールが自動的に行われる(数分かかる場合がある)
    \item インストールが完了すると、Webブラウザが自動的に起動し、システムのUIが表示される
    \item ブラウザのアドレスバーには「\texttt{http://localhost:8501}」と表示される
\end{enumerate}

\subsubsection{システムの操作手順}

システムのUIは、左側のサイドバーと右側のメイン画面で構成される。以下に、基本的な操作手順を示す。

\paragraph{1. 道路ネットワークの選択}

\begin{enumerate}
    \item サイドバーの「道路ネットワークの選択」セクションで、使用する道路データを選択する
    \item デフォルトでは「群馬県道路ネットワーク」が選択されている
    \item カスタムネットワークを使用する場合は、「カスタムファイルをアップロード」を選択し、JSON形式の道路データファイルを選択する
\end{enumerate}

\paragraph{2. 拠点の設定}

\begin{enumerate}
    \item サイドバーの「拠点設定」セクションで、「デポ(車庫)を設定」ボタンをクリックする
    \item メイン画面の地図上で、車両の出発・帰着地点をクリックする
    \item 設定された地点に赤色のマーカーが表示される
\end{enumerate}

\paragraph{3. 回収地点の設定}

\begin{enumerate}
    \item サイドバーの「回収地点設定」セクションで、「回収地点を追加」ボタンをクリックする
    \item 地図上で回収地点をクリックする
    \item ポップアップウィンドウが表示されるので、以下の情報を入力する:
    \begin{itemize}
        \item 資源種別(11種類から選択)
        \item 回収量(kg単位)
        \item 地点名(任意)
    \end{itemize}
    \item 「追加」ボタンをクリックして確定する
    \item 設定された地点に青色のマーカーが表示される
    \item 複数の回収地点を追加する場合は、この操作を繰り返す
\end{enumerate}

\paragraph{4. 集積場所の設定}

\begin{enumerate}
    \item サイドバーの「集積場所設定」セクションで、「集積場所を設定」ボタンをクリックする
    \item 地図上で資源の最終目的地をクリックする
    \item 設定された地点に緑色のマーカーが表示される
\end{enumerate}

\paragraph{5. 車両の選択}

\begin{enumerate}
    \item サイドバーの「車両選択」セクションで、使用する車両タイプを選択する
    \item システムが自動的に、選択された資源に適合する車両のみを表示する
    \item 適合する車両が複数ある場合、コスト試算を確認して選択できる
\end{enumerate}

\paragraph{6. 最適化の実行}

\begin{enumerate}
    \item すべての設定が完了したら、サイドバー下部の「最適化を実行」ボタンをクリックする
    \item システムが最適経路を計算する(通常数秒~数十秒)
    \item 計算が完了すると、最適化されたルートが地図上に表示される
\end{enumerate}

\subsubsection{結果の確認}

最適化計算が完了すると、以下の情報がメイン画面に表示される。

\paragraph{地図表示}

\begin{itemize}
    \item 最適化されたルートが青色の線で表示される
    \item 各区間をクリックすると、区間距離とコストがポップアップ表示される
    \item ズームイン・ズームアウト、地図の移動が可能
\end{itemize}

\paragraph{ルート情報}

\begin{itemize}
    \item 訪問順序と各地点間の距離
    \item 総走行距離(km)
    \item 推定所要時間
\end{itemize}

\paragraph{コスト詳細}

\begin{itemize}
    \item 変動費12項目の内訳と金額
    \item 固定費13項目の内訳と金額
    \item 総コスト(円)
    \item コスト単価(円/km)
\end{itemize}

\paragraph{エネルギー消費}

\begin{itemize}
    \item 燃料消費量(L)
    \item CO$_2$排出量(kg)
\end{itemize}

\subsubsection{データのエクスポート}

計算結果は、以下の形式でエクスポート可能である:

\begin{enumerate}
    \item サイドバーの「結果をエクスポート」セクションで、出力形式を選択する
    \item \textbf{CSV形式}:Excelで開けるデータファイル(ルート情報、コスト詳細)
    \item \textbf{JSON形式}:プログラムで読み込み可能な構造化データ
    \item \textbf{HTML形式}:地図を含む完全なレポート(印刷可能)
    \item \textbf{PDF形式}:印刷用の正式レポート(要追加ライブラリ)
    \item 「ダウンロード」ボタンをクリックして、ファイルを保存する
\end{enumerate}

\subsubsection{システムの終了}

\begin{enumerate}
    \item Webブラウザを閉じる
    \item コマンドプロンプトウィンドウで「Ctrl + C」キーを押す
    \item 確認メッセージが表示されたら「Y」を入力してEnterキーを押す
    \item コマンドプロンプトウィンドウが閉じ、システムが終了する
\end{enumerate}

\subsubsection{トラブルシューティング}

\paragraph{起動しない場合}

\begin{itemize}
    \item Pythonが正しくインストールされているか確認する
    \item コマンドプロンプトで「\texttt{python --version}」を実行し、バージョン3.8以降であることを確認する
    \item インターネット接続を確認する(初回起動時のライブラリインストールに必要)
    \item ウイルス対策ソフトによってブロックされていないか確認する
\end{itemize}

\paragraph{地図が表示されない場合}

\begin{itemize}
    \item インターネット接続を確認する
    \item ブラウザのキャッシュをクリアする
    \item ブラウザを変更してみる(Chrome推奨)
\end{itemize}

\paragraph{最適化が終わらない場合}

\begin{itemize}
    \item 回収地点が多すぎる可能性がある(10地点以下を推奨)
    \item 一度ブラウザをリロードして、設定をやり直す
    \item 道路ネットワークデータが正しく読み込まれているか確認する
\end{itemize}

\paragraph{エラーメッセージが表示される場合}

\begin{itemize}
    \item エラーメッセージの内容を記録する
    \item 入力データが正しいか確認する(座標、資源量など)
    \item システムを再起動する
    \item 問題が解決しない場合は、システム管理者に連絡する
\end{itemize}


\newpage

% ========================================
% 4. 結論
% ========================================
\section{結論}

\subsection{成果のまとめ}

本研究では、群馬県内における未利用資源の効率的な収集運搬を支援するシステムを開発した。主な成果を以下にまとめる。

\subsubsection{システムの実現}

交通安全環境研究所で開発中の「地域交通計画立案ツール」を基盤として、未利用資源運搬に特化した最適化システムを実現した。Streamlitフレームワークを用いたWebアプリケーションとして実装し、専門知識を持たないユーザーでも直感的に操作できるUIを提供している。

\subsubsection{体系的なデータ整備}

\begin{itemize}
    \item \textbf{未利用資源}:11種類の資源(建設廃材、農業残渣、林業残材、食品廃棄物、廃プラスチック、金属スクラップ、古紙・段ボール、剪定枝・草、家畜糞尿、下水汚泥、廃食用油)について、物理的・化学的特性と運搬上の注意点を整理した
    \item \textbf{運搬車両}:14種類の車両(軽トラックから10tクラスまで、および特殊車両)について、積載容積、燃費、コスト構造を詳細に調査・設定した
    \item \textbf{適合性マトリックス}:154組み合わせ(11資源×14車両)について、法令遵守、安全性、経済性の観点から適合性を判定し、体系的に整理した
\end{itemize}

\subsubsection{最適化アルゴリズムの実装}

\begin{itemize}
    \item グラフ理論に基づくDijkstra法とA*アルゴリズムによる最短経路探索を実装した
    \item 車両経路問題(VRP)の解法として、貪欲法による初期解生成と2-opt法による局所改善を組み合わせた
    \item 容量制約と資源適合性を考慮した現実的な最適化を実現した
\end{itemize}

\textbf{注意点}:本システムでは訪問順序の最適化に厳密解を求めるアプローチを採用しているため、回収地点数が増加すると計算時間が指数的に増大する。実用上は\textbf{10カ所程度がリミット}である。より多くの地点を扱う場合は、ヒューリスティック手法への切り替えが必要となる。

\subsubsection{詳細なコスト計算機能}

運搬費用を変動費12項目、固定費13項目に分解し、合計25項目の詳細なコスト計算を実現した。これにより、運搬計画の経済性を事前に評価し、コスト削減の余地を特定することが可能となった。

\subsubsection{実用的な提供形態}

配布フォルダに\texttt{run\_app.bat}を含めることで、複雑なセットアップ作業なしに、ダブルクリック一つでシステムを起動できる形態を実現した。計算結果はCSV、JSON、HTML形式でエクスポート可能であり、既存の業務フローへの統合が容易である。

\subsection{今後の展望}

本研究で開発したシステムを基盤として、以下の発展的研究の可能性が考えられる。

\subsubsection{貨客混載による地域内最適化への拡張}

現在のシステムは未利用資源の運搬のみを対象としているが、これを人流(旅客)と物流(貨物)を統合した貨客混載システムへ拡張することで、地域交通全体の最適化が可能となる。図\ref{fig:future_system}に、拡張システムの概念を示す。

\begin{figure}[H]
\centering
\fbox{
\begin{minipage}{0.9\textwidth}
\centering
\vspace{0.5cm}
\textbf{【現行システム】} \\
未利用資源運搬の最適化 \\
\vspace{0.3cm}
$\downarrow$ 拡張 \\
\vspace{0.3cm}
\textbf{【将来システム:貨客混載最適化】} \\
\vspace{0.3cm}
\begin{tabular}{cc}
\fbox{\begin{minipage}{0.4\textwidth}
\centering
\textbf{人流} \\
通勤・通学 \\
通院・買い物 \\
観光
\end{minipage}} &
\fbox{\begin{minipage}{0.4\textwidth}
\centering
\textbf{物流} \\
未利用資源 \\
農産物・加工品 \\
宅配便
\end{minipage}}
\end{tabular} \\
\vspace{0.3cm}
$\downarrow$ 統合最適化 \\
\vspace{0.3cm}
\textbf{総コスト最小化} \\
人流コスト + 物流コスト $<$ 独立運行コスト \\
\vspace{0.5cm}
\end{minipage}
}
\caption{貨客混載システムへの拡張概念}
\label{fig:future_system}
\end{figure}

\subsubsection{研究課題1:混載可能性の分類}

未利用資源を、旅客と同時運搬可能なものと不可能なものに分類する必要がある。

\paragraph{混載可能な資源}
\begin{itemize}
    \item 古紙・段ボール(梱包済み、無臭)
    \item 廃食用油(密閉容器入り)
    \item 廃プラスチック(洗浄済み、袋詰め)
    \item 農産物(出荷品質)
\end{itemize}

\paragraph{混載不可能な資源}
\begin{itemize}
    \item 家畜糞尿(臭気が強い)
    \item 下水汚泥(衛生上の問題)
    \item 食品廃棄物(腐敗リスク)
    \item 建設廃材(汚損・安全性の問題)
\end{itemize}

\subsubsection{研究課題2:混載不可資源の同時運搬}

混載不可能とされる資源同士でも、適切な工夫により同一車両で効率的に運搬できる可能性がある。

\paragraph{区画分離方式}
車両内部を物理的に区画分離することで、臭気や汚染の相互影響を防止する。例えば、家畜糞尿と建設廃材を同一車両で運搬する場合、密閉コンテナと開放エリアに分離する。

\paragraph{時間帯分離方式}
同一車両を時間帯によって用途分けし、洗浄・消毒を徹底することで、午前は旅客、午後は貨物といった運用が可能となる。

\paragraph{車両改造方式}
着脱式の内装(シートカバー、床マット等)や可動式隔壁を導入し、柔軟な用途変更を実現する。

\subsubsection{研究課題3:貨客混載の優位性検証}

貨客独立運搬と貨客混載のコストを詳細に比較し、混載の優位性を定量的に検証する必要がある。

\paragraph{比較対象}
\begin{equation}
\text{独立運搬コスト} = \text{人流運搬コスト} + \text{資源運搬コスト}
\end{equation}
\begin{equation}
\text{混載運搬コスト} = \text{統合運行コスト} + \text{追加設備費用}
\end{equation}

\paragraph{シミュレーション項目}
\begin{itemize}
    \item 運行パターン:デマンド型、定時定路線型、ハイブリッド型
    \item 需要変動:平日/休日、時間帯別の人流・物流需要
    \item 地域特性:市街地、郊外、中山間地域での比較
    \item 車両タイプ:小型・中型・大型、専用車・兼用車
    \item 環境負荷:CO$_2$排出量、エネルギー消費量
\end{itemize}

\paragraph{期待される効果}
\begin{itemize}
    \item 車両稼働率の向上(空車走行の削減)
    \item 人件費の削減(ドライバー数の削減)
    \item 環境負荷の低減(総走行距離の削減)
    \item 地域サービスの維持(採算性の改善)
\end{itemize}

\subsubsection{研究課題4:統合最適化システムの開発}

人流と物流を統合的に扱う最適化システムの開発が必要である。

\paragraph{技術的課題}
\begin{itemize}
    \item マルチモーダル需要予測:人流と物流の時空間需要の統合的予測
    \item 動的配車アルゴリズム:リアルタイムの需要変動に対応した配車最適化
    \item 制約条件の複雑化:旅客の乗車時間制約、貨物の時間指定配送
    \item 公平性の確保:旅客サービス品質と物流効率のトレードオフ
\end{itemize}

\paragraph{システム要件}
\begin{itemize}
    \item リアルタイム性:需要発生から配車決定まで数分以内
    \item スケーラビリティ:広域(市町村レベル)での運用に対応
    \item 利用者インターフェース:住民・事業者向けの予約・追跡機能
    \item データ連携:既存の公共交通・物流システムとの情報連携
\end{itemize}

\subsubsection{期待される社会的効果}

貨客混載システムの実現により、以下の社会的効果が期待される:

\begin{itemize}
    \item \textbf{持続可能な地域交通の実現}:過疎地域での交通サービス維持
    \item \textbf{未利用資源の利活用促進}:収集運搬コスト低減によるリサイクル推進
    \item \textbf{地域経済の活性化}:農産物等の輸送効率化による競争力向上
    \item \textbf{環境負荷の低減}:総走行距離削減によるCO$_2$排出量削減
    \item \textbf{ドライバー不足への対応}:車両・人員の効率的活用
\end{itemize}

\subsubsection{実証実験の提案}

理論的検討を経て、以下の段階的な実証実験を提案する:

\begin{enumerate}
    \item \textbf{フェーズ1(6ヶ月)}:混載可能資源の選定と小規模実験(1-2路線)
    \item \textbf{フェーズ2(12ヶ月)}:車両改造・運用ルールの確立と中規模実験(地区レベル)
    \item \textbf{フェーズ3(24ヶ月)}:統合システムの開発と広域実証(市町村レベル)
\end{enumerate}

本研究で開発したシステムは、これらの発展的研究の基盤として活用され、持続可能な地域交通システムの実現に貢献することが期待される。


\newpage

% ========================================
% 参考文献
% ========================================
\section*{参考文献}
\addcontentsline{toc}{section}{参考文献}

\subsection*{公的機関}

\begin{enumerate}
    \item 国土交通省「自動車燃費一覧」、2025年版
    \item 国土交通省「自動車諸元表」、2025年版
    \item 厚生労働省「賃金構造基本統計調査」、令和4年
    \item 全日本トラック協会「経営分析報告書」、2024年度版
    \item 全日本トラック協会「トラック運送事業の賃金・労働時間等の実態」、2024年度版
\end{enumerate}

\subsection*{業界団体・メーカー}

\begin{enumerate}
    \setcounter{enumi}{5}
    \item いすゞ自動車株式会社、車両カタログ、2025年版
    \item 日野自動車株式会社、車両カタログ、2025年版
    \item 三菱ふそうトラック・バス株式会社、車両カタログ、2025年版
    \item UDトラックス株式会社、車両カタログ、2025年版
    \item 社団法人プラスチック処理促進協会「燃料消費原単位」、2024年版
\end{enumerate}

\subsection*{中古車販売・情報サイト}

\begin{enumerate}
    \setcounter{enumi}{10}
    \item トラック王国、\url{https://www.55truck.com/}、2025年10月アクセス
    \item トラック流通センター、\url{https://www.kaitoriou.net/}、2025年10月アクセス
    \item TRUCK BIZ、\url{https://www.truck-five.com/tfbiz/}、2025年10月アクセス
    \item トラック市、\url{https://www.truck-ichi.co.jp/}、2025年10月アクセス
    \item バディトラック、\url{https://buddytruck.jp/}、2025年10月アクセス
\end{enumerate}

\subsection*{その他}

\begin{enumerate}
    \setcounter{enumi}{15}
    \item 求人ボックス「トラック運転の年収・時給」、2025年10月アクセス
    \item 運転ドットコム「トラック運転手給与情報」、2025年10月アクセス
    \item トラック運送事業者へのヒアリング調査(2024年9月-2025年10月実施)
\end{enumerate}

\subsection*{ソフトウェア・ライブラリ}

\begin{enumerate}
    \setcounter{enumi}{18}
    \item Streamlit Team, Streamlit: The fastest way to build data apps, \url{https://streamlit.io/}, v1.28.0
    \item NetworkX Developers, NetworkX: Network Analysis in Python, \url{https://networkx.org/}, v3.1
    \item Folium Contributors, Folium: Python Data, Leaflet.js Maps, \url{https://python-visualization.github.io/folium/}, v0.14.0
    \item Google OR-Tools Team, Google OR-Tools, \url{https://developers.google.com/optimization}, v9.7
    \item McKinney, W., pandas: powerful Python data analysis toolkit, \url{https://pandas.pydata.org/}, v2.1.0
\end{enumerate}

\subsection*{学術文献}

\begin{enumerate}
    \setcounter{enumi}{23}
    \item Dijkstra, E. W. (1959). A note on two problems in connexion with graphs. \textit{Numerische Mathematik}, 1(1), 269-271.
    \item Hart, P. E., Nilsson, N. J., \& Raphael, B. (1968). A formal basis for the heuristic determination of minimum cost paths. \textit{IEEE Transactions on Systems Science and Cybernetics}, 4(2), 100-107.
    \item Dantzig, G. B., \& Ramser, J. H. (1959). The truck dispatching problem. \textit{Management Science}, 6(1), 80-91.
    \item Toth, P., \& Vigo, D. (Eds.). (2014). \textit{Vehicle Routing: Problems, Methods, and Applications} (2nd ed.). SIAM.
    \item Croes, G. A. (1958). A method for solving traveling-salesman problems. \textit{Operations Research}, 6(6), 791-812.
\end{enumerate}


\end{document}
