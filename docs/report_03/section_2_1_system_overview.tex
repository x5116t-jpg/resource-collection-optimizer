\subsection{システムの全体構成}

本システムは、未利用資源の収集運搬計画を最適化するため、以下の主要コンポーネントから構成される(図\ref{fig:system_architecture})。

\begin{figure}[H]
\centering
\begin{minipage}{0.9\textwidth}
\centering

% ユーザー入力ボックス(システムの外側)
\fbox{
\begin{minipage}{0.85\textwidth}
\centering
\vspace{0.3cm}
\textbf{【ユーザー入力】} \\
\vspace{0.2cm}
資源回収地点・資源種別・回収量、\\
集積地点、車庫地点
\vspace{0.3cm}
\end{minipage}
}

\vspace{0.3cm}
$\downarrow$ \textit{入力}
\vspace{0.3cm}

% システム全体ボックス
\fbox{
\begin{minipage}{0.85\textwidth}
\centering
\vspace{0.3cm}
\textbf{【システム】}
\vspace{0.3cm}

\hrule
\vspace{0.3cm}

\textbf{【入力層】} \\
\textit{システムデフォルト}:\\
道路ネットワークデータ、マスタデータ\\
(運搬車両の諸元、未利用資源の特性、車両と資源の適合マトリクス)
\vspace{0.3cm}

$\downarrow$ \\
\vspace{0.3cm}

\textbf{【処理層】} \\
地点間の最短ルート探索 → 回収順序の最適化
\vspace{0.3cm}

$\downarrow$ \\
\vspace{0.3cm}

\textbf{【出力層】} \\
最適ルート情報、コスト詳細、\\
エネルギー消費量、地図表示 \\
\vspace{0.3cm}
\end{minipage}
}

\end{minipage}
\caption{システムの全体構成}
\label{fig:system_architecture}
\end{figure}

\subsubsection{システム構成要素}

システムは3層構造で構成されており、図\ref{fig:system_architecture}に示す各層の要素を以下に説明する。

\paragraph{ユーザー入力}
ユーザーが入力・選択する情報である:
\begin{itemize}
    \item 車庫地点:車両の出発地点・帰着地点の座標
    \item 資源回収地点:資源を回収する地点の座標
    \item 資源種別:回収する資源の種類(紙、プラスチック、ガラス等)
    \item 回収量:各地点での資源回収量
    \item 集積地点:資源の最終目的地の座標
\end{itemize}

\paragraph{入力層}
入力層では、システムが最適化計算を実行するために必要なデータを管理する。

\begin{detaillist}
    \detailitem{システムデフォルトのデータ} システムに予め組み込まれており、ユーザーの入力作業を軽減する基本データである。

    \textbf{道路ネットワークデータ}:OpenStreetMap(OSM)またはカスタムJSON形式で提供される道路ネットワーク情報である。ノード(交差点・地点)とエッジ(道路区間)の接続関係、各道路区間の距離情報を含む。

    \textbf{マスタデータ}:システムの基礎となる静的データであり、以下の3つから構成される。
    \begin{itemize}
        \item 14種類の運搬車両の諸元(積載容積、燃費、購入費用、固定費、変動費)
        \item 11種類の未利用資源の特性(比重、取扱い注意事項)
        \item 車両と資源の適合性マトリックス(154組み合わせ)
    \end{itemize}

    これらのデータはシステム内に標準装備されており、ユーザーは必要に応じて選択のみを行えばよい。
\end{detaillist}

\paragraph{処理層}
処理層では、入力されたデータを用いて最適化計算を実行する。

\begin{detaillist}
    \detailitem{地点間の最短ルート探索} 道路ネットワーク上で、すべての地点間の最短経路をDijkstra法で計算する。交差点や道路区間の接続関係を考慮し、実際の道路距離に基づいた最短ルートを求める。

    \detailitem{回収順序の最適化} 車両経路問題を解決し、最適な訪問順序を決定する。容量制約と資源適合性を考慮し、総走行距離やコストが最小となるルートを計算する。
\end{detaillist}

\paragraph{出力層}
出力層では、最適化結果をユーザーに分かりやすい形式で提供する。

\begin{detaillist}
    \detailitem{最適ルート情報} 車両ごとの訪問順序、各区間の距離、総走行距離、所要時間を表示する。

    \detailitem{コスト詳細} 変動費12項目と固定費13項目の内訳、項目別金額、総コストを詳細に表示する。

    \detailitem{エネルギー消費量} 燃料消費量、CO$_2$排出量を車両の燃費データから算出し表示する。

    \detailitem{地図表示} 最適化されたルートをインタラクティブな地図上に可視化する。経路は色分けされ、クリックすると詳細情報がポップアップ表示される。
\end{detaillist}

\subsubsection{データフロー}

システムのデータフローは以下の通りである:

\begin{enumerate}
    \item ユーザーが道路ネットワークを選択し、システムに読み込む
    \item 地図上で車庫地点、資源回収地点、集積地点を設定する
    \item 使用する車両タイプと運搬する資源タイプを選択する
    \item システムが車両・資源適合性を自動的に検証する
    \item 最適化エンジンが地点間の最短ルートを探索し、回収順序を最適化する
    \item 結果が地図上に可視化され、詳細レポートが生成される
    \item ユーザーは結果をCSV、JSON、HTML形式でエクスポート可能
\end{enumerate}

\subsubsection{技術スタック}

本システムは以下の技術により構築されている:

\begin{itemize}
    \item \textbf{グラフ処理}:NetworkX(道路ネットワークのグラフ表現と経路探索)
    \item \textbf{地図可視化}:Folium(インタラクティブマップ生成)
    \item \textbf{データ処理}:Pandas、NumPy(データ管理と数値計算)
    \item \textbf{最適化}:OR-Tools(制約付き最適化問題の解決)
    \item \textbf{データ形式}:JSON、CSV(データの入出力)
\end{itemize}

