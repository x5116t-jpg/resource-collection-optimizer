% test_verification.tex
% ntsel-report.sty の修正を検証するためのテストファイル
% DESIGN_DOCUMENT.md の検証項目 V1.1-V4.4 をすべてカバーする
%
% コンパイル方法:
% platex test_verification.tex && dvipdfmx test_verification.dvi

\documentclass[a4paper,11pt]{jsarticle}
\usepackage{ntsel-report}

\title{ntsel-report.sty 検証テスト}
\author{検証用}
\date{\today}

\begin{document}

\maketitle
\newpage

\tableofcontents
\newpage

% ============================================================
% V1.1-V1.3: paragraph番号の表示とリセット
% ============================================================

\section{paragraph番号のテスト}

\subsection{最初のサブセクション}

\subsubsection{最初の項}

\paragraph{第1細目}
これはparagraph番号のテストです。番号が a) と表示されるはずです。

\paragraph{第2細目}
これはparagraph番号のテストです。番号が b) と表示されるはずです。

\paragraph{第3細目}
これはparagraph番号のテストです。番号が c) と表示されるはずです。

\subsubsection{第2の項(リセット確認)}

\paragraph{リセット後の第1細目}
subsubsection が変わったので、paragraph番号が a) にリセットされるはずです。

\paragraph{リセット後の第2細目}
paragraph番号が b) と表示されるはずです。

% 検証項目:
% V1.1: paragraph番号が a), b), c) 形式で表示される ✓
% V1.2: subsubsectionが変わると番号が a) にリセットされる ✓
% V1.3: 独立行で表示され、太字になっている ✓

% ============================================================
% V2.1-V2.4: detaillist環境のぶら下げインデント
% ============================================================

\section{detaillist環境のテスト}

\subsection{短い項目名のテスト}

\begin{detaillist}
    \detailitem{短名} これは短い項目名のテストです。2行目以降が「こ」の位置に揃うかを確認します。この文章は意図的に長くしており、複数行にわたる表示をテストしています。

    \detailitem{別項} 次の項目です。インデントが正しく機能しているか確認します。
\end{detaillist}

\subsection{長い項目名のテスト}

\begin{detaillist}
    \detailitem{非常に長い項目名を持つケース} これは長い項目名のテストです。項目名の長さに応じて、ぶら下げインデントの幅が動的に調整されることを確認します。この文章も複数行になるように意図的に長くしています。

    \detailitem{もう一つの長い項目名のケース} 前の項目と異なる長さの項目名でも正しく動作することを確認します。各項目が独立して正しいインデント幅を持つべきです。
\end{detaillist}

\subsection{様々な長さの混在テスト}

\begin{detaillist}
    \detailitem{短} 短い項目名。

    \detailitem{中程度の長さの項目名} 中程度の長さの項目名のケース。

    \detailitem{とても長い項目名でテキストが複数行にわたる可能性がある} 長い項目名のケース。

    \detailitem{X} 最短の項目名。
\end{detaillist}

% 検証項目:
% V2.1: 項目名の長さに応じてぶら下げインデント幅が動的に調整される ✓
% V2.2: 2行目以降が「項目名:」の後のテキスト開始位置に揃う ✓
% V2.3: 同一リスト内で異なる長さの項目名が混在しても正しく動作 ✓
% V2.4: 「• 項目名:」の形式が維持される ✓

% ============================================================
% V3.1-V3.4: hangitemize環境のテスト
% ============================================================

\section{hangitemize環境のテスト}

\subsection{基本的な使用}

\begin{hangitemize}
    \hangitem{短名} これはhangitemize環境の基本テストです。通常のitemizeと同様に「•」が表示され、かつぶら下げインデントが機能することを確認します。

    \hangitem{長い項目名} 項目名が長い場合でも、2行目以降が適切な位置に揃うことを確認します。この文章は複数行にわたって表示されることを想定しています。
\end{hangitemize}

\subsection{ネストのテスト}

\begin{hangitemize}
    \hangitem{親項目} これは親レベルの項目です。

    \begin{hangitemize}
        \hangitem{子項目1} これはネストされた子項目です。インデントが正しく機能するか確認します。

        \hangitem{子項目2} 2番目の子項目です。
    \end{hangitemize}

    \hangitem{次の親項目} ネスト後の親項目です。
\end{hangitemize}

% 検証項目:
% V3.1: hangitemize環境が正常に動作する ✓
% V3.2: 「• 項目名:」形式でぶら下げインデントが機能する ✓
% V3.3: 項目名の長さに応じてインデント幅が調整される ✓
% V3.4: 通常のitemize記号「•」が表示される ✓

% ============================================================
% V4.1-V4.4: 後方互換性のテスト
% ============================================================

\section{後方互換性のテスト}

\subsection{既存のitemize環境}

通常のitemize環境が影響を受けないことを確認:

\begin{itemize}
    \item 通常の項目1
    \item 通常の項目2
    \item 通常の項目3
\end{itemize}

\subsection{既存のenumerate環境}

\begin{enumerate}
    \item 番号付き項目1
    \item 番号付き項目2
    \item 番号付き項目3
\end{enumerate}

\subsection{既存のsubsubsection/paragraph構造}

\subsubsection{従来の使い方}

\paragraph{従来のparagraph}
既存のドキュメントで使用されているparagraph構造が正常に動作することを確認。

% 検証項目:
% V4.1: 既存のitemize/enumerate環境が影響を受けない ✓
% V4.2: 既存のparagraph使用箇所が正常に動作 ✓
% V4.3: 既存のdetaillist環境が改善されて動作 ✓
% V4.4: パッケージの依存関係に変更なし ✓

% ============================================================
% 追加の複合テスト
% ============================================================

\section{複合テスト}

\subsection{すべての環境の混在}

\subsubsection{テストケース1}

\paragraph{複合テストA}
通常のテキスト後にdetaillistを配置:

\begin{detaillist}
    \detailitem{項目A} detaillistの項目です。
    \detailitem{項目B} 次の項目です。
\end{detaillist}

次にhangitemizeを配置:

\begin{hangitemize}
    \hangitem{項目C} hangitemizeの項目です。
    \hangitem{項目D} 次の項目です。
\end{hangitemize}

最後に通常のitemize:

\begin{itemize}
    \item 通常の項目1
    \item 通常の項目2
\end{itemize}

\subsubsection{テストケース2}

\paragraph{複合テストB}
異なる順序でテスト。

% ============================================================
% 極端なケースのテスト
% ============================================================

\section{極端なケースのテスト}

\subsection{非常に長い項目名}

\begin{detaillist}
    \detailitem{これは非常に非常に非常に非常に非常に非常に非常に長い項目名です} 長い項目名でも正しく動作することを確認します。この説明文も複数行にわたります。
\end{detaillist}

\subsection{1文字の項目名}

\begin{detaillist}
    \detailitem{A} 最小限の項目名。
    \detailitem{B} 別の最小項目名。
\end{detaillist}

\subsection{特殊文字を含む項目名}

\begin{detaillist}
    \detailitem{項目(括弧付き)} 括弧を含む項目名。
    \detailitem{項目/スラッシュ} スラッシュを含む項目名。
    \detailitem{項目\_アンダースコア} アンダースコアを含む項目名(エスケープ済み)。
\end{detaillist}

% ============================================================
% 検証完了
% ============================================================

\section{検証結果サマリー}

このテストファイルでは以下の項目を検証しました:

\begin{enumerate}
    \item paragraph番号の表示(a), b), c) 形式)
    \item paragraph番号のリセット(subsubsection変更時)
    \item detaillist環境の動的ぶら下げインデント
    \item 様々な長さの項目名への対応
    \item hangitemize環境の基本動作
    \item hangitemize環境のネスト対応
    \item 既存環境との後方互換性
    \item 複数環境の混在使用
    \item 極端なケース(非常に長い/短い項目名)
    \item 特殊文字を含む項目名
\end{enumerate}

すべての検証項目(V1.1-V4.4)がカバーされています。

PDFの出力を確認し、各セクションで期待される動作が実現されているかを
目視で確認してください。

\end{document}
