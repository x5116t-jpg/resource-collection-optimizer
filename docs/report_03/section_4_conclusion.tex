\subsection{成果のまとめ}

本研究では、群馬県内における未利用資源の効率的な収集運搬を支援するシステムを開発した。主な成果を以下にまとめる。

\subsubsection{システムの実現}

交通安全環境研究所で開発中の「地域交通計画立案ツール」を基盤として、未利用資源運搬に特化した最適化システムを実現した。専門知識を持たないユーザーでも直感的に操作できるUIを提供している。

\subsubsection{体系的なデータ整備}

\begin{itemize}
    \item \textbf{未利用資源}:11種類の資源(建設廃材、農業残渣、林業残材、食品廃棄物、廃プラスチック、金属スクラップ、古紙・段ボール、剪定枝・草、家畜糞尿、下水汚泥、廃食用油)について、物理的・化学的特性と運搬上の注意点を整理した
    \item \textbf{運搬車両}:14種類の車両(軽トラックから10tクラスまで、および特殊車両)について、積載容積、燃費、コスト構造を詳細に調査・設定した
    \item \textbf{適合性マトリックス}:154組み合わせ(11資源×14車両)について、法令遵守、安全性、経済性の観点から適合性を判定し、体系的に整理した
\end{itemize}

\subsubsection{最適化アルゴリズムの実装}

\begin{itemize}
    \item グラフ理論に基づくDijkstra法とA*アルゴリズムによる最短経路探索を実装した
    \item 車両経路問題(VRP)の解法として、貪欲法による初期解生成と2-opt法による局所改善を組み合わせた
    \item 容量制約と資源適合性を考慮した現実的な最適化を実現した
\end{itemize}

\textbf{注意点}:本システムでは訪問順序の最適化に厳密解を求めるアプローチを採用しているため、回収地点数が増加すると計算時間が指数的に増大する。実用上は\textbf{10カ所程度がリミット}である。より多くの地点を扱う場合は、ヒューリスティック手法への切り替えが必要となる。

\subsubsection{詳細なコスト計算機能}

運搬費用を変動費12項目、固定費13項目に分解し、合計25項目の詳細なコスト計算を実現した。これにより、運搬計画の経済性を事前に評価し、コスト削減の余地を特定することが可能となった。

\subsubsection{実用的な提供形態}

配布フォルダに\texttt{run\_app.bat}を含めることで、複雑なセットアップ作業なしに、ダブルクリック一つでシステムを起動できる形態を実現した。計算結果はCSV、JSON、HTML形式でエクスポート可能であり、既存の業務フローへの統合が容易である。

\subsection{今後の展望}

本研究で開発したシステムを基盤として、以下の発展的研究の可能性が考えられる。

\subsubsection{貨客混載による地域内最適化への拡張}

現在のシステムは未利用資源の運搬のみを対象としているが、これを人流(旅客)と物流(貨物)を統合した貨客混載システムへ拡張することで、地域交通全体の最適化が可能となる。図\ref{fig:future_system}に、拡張システムの概念を示す。

\begin{figure}[H]
\centering
\fbox{
\begin{minipage}{0.9\textwidth}
\centering
\vspace{0.5cm}
\textbf{【現行システム】} \\
未利用資源運搬の最適化 \\
\vspace{0.3cm}
$\downarrow$ 拡張 \\
\vspace{0.3cm}
\textbf{【将来システム:貨客混載最適化】} \\
\vspace{0.3cm}
\begin{tabular}{cc}
\fbox{\begin{minipage}{0.4\textwidth}
\centering
\textbf{人流} \\
通勤・通学 \\
通院・買い物 \\
観光
\end{minipage}} &
\fbox{\begin{minipage}{0.4\textwidth}
\centering
\textbf{物流} \\
未利用資源 \\
農産物・加工品 \\
宅配便
\end{minipage}}
\end{tabular} \\
\vspace{0.3cm}
$\downarrow$ 統合最適化 \\
\vspace{0.3cm}
\textbf{総コスト最小化} \\
人流コスト + 物流コスト $<$ 独立運行コスト \\
\vspace{0.5cm}
\end{minipage}
}
\caption{貨客混載システムへの拡張概念}
\label{fig:future_system}
\end{figure}

\subsubsection{研究課題1:混載可能性の分類}

未利用資源を、旅客と同時運搬可能なものと不可能なものに分類する必要がある。

\paragraph{混載可能な資源}
\begin{itemize}
    \item 古紙・段ボール(梱包済み、無臭)
    \item 廃食用油(密閉容器入り)
    \item 廃プラスチック(洗浄済み、袋詰め)
    \item 農産物(出荷品質)
\end{itemize}

\paragraph{混載不可能な資源}
\begin{itemize}
    \item 家畜糞尿(臭気が強い)
    \item 下水汚泥(衛生上の問題)
    \item 食品廃棄物(腐敗リスク)
    \item 建設廃材(汚損・安全性の問題)
\end{itemize}

本研究で開発したシステムは、これらの発展的研究の基盤として活用され、持続可能な地域交通システムの実現に貢献することが期待される。
