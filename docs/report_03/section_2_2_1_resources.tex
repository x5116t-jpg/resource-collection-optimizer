\subsubsection{資源}

本システムでは、群馬県内で発生する11種類の未利用資源を対象としている。各資源の特性と収集における注意点を以下に示す。

\begin{detaillist}
    \detailitem{建設廃材} 建設・解体工事から発生する木材、金属、コンクリートがらなどを含む。比重が大きく重量物が多いため、積載重量に注意が必要である。混合廃棄物の場合は分別処理が必要となる。

    \detailitem{農業残渣} 稲わら、もみ殻、麦わらなど農作物収穫後の残留物である。軽量でかさばる特性を持ち、乾燥状態と湿潤状態で性状が大きく異なる。運搬時は飛散防止対策が重要である。

    \detailitem{林業残材} 間伐材、枝葉、樹皮、製材端材などを指す。長尺物が多く、水分含有率により重量が大きく変動する。積載時は荷崩れ防止のための固定が必要である。

    \detailitem{食品廃棄物} 生ごみ、食品残渣、売れ残り食品などを含む。水分含有率が高く腐敗しやすいため、迅速な運搬が求められる。汁漏れ防止のため、密閉容器の使用または専用車両が必要である。

    \detailitem{廃プラスチック} 容器包装、フィルム、硬質プラスチックなどを含む。軽量でかさばり、風で飛散しやすいため、シート養生または密閉車両での運搬が推奨される。

    \detailitem{金属スクラップ} 鉄くず、非鉄金属、廃家電から回収した金属などを含む。比重が極めて大きいため、積載重量に特に注意が必要である。クレーン付き車両の使用が効率的な場合が多い。

    \detailitem{古紙・段ボール} 新聞紙、雑誌、段ボール箱などを含む。水濡れ厳禁であり、圧縮により積載効率が向上する。雨天時はシート養生または箱型車両の使用が必須である。

    \detailitem{剪定枝・草} 庭木剪定枝、草刈り残渣、落ち葉などを含む。かさばるため、破砕処理により積載効率が大幅に向上する。パッカー車による圧縮収集も有効である。

    \detailitem{家畜糞尿} 牛糞、豚糞、鶏糞などを含む。半固形から液状まで性状が多様であり、臭気対策が必須である。液状のものはバキューム車、半固形のものはダンプまたは密閉コンテナで運搬する。

    \detailitem{下水汚泥} 浄化槽汚泥、し尿などを含む。液状から泥状であり、専用車両(バキューム車)での運搬が法令により義務付けられている。

    \detailitem{廃食用油} 使用済み天ぷら油、業務用廃油などを含む。液体であるため、漏洩防止のため密閉容器(缶・ペットボトル)での回収を前提とする。容器ごと運搬することで、平ボディやウイング車でも運搬可能である。
\end{detaillist}

\vspace{1em}

これらの資源は、その物理的・化学的特性に応じて適切な運搬車両を選択する必要がある。次項では、これらの資源運搬に使用可能な車両について詳述する。
